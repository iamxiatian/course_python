\chapter{引言}

\section{什么是Python}

如果你打算让计算机做更多的事情,而不仅仅局限于使用他人已经编好的可运行程序,例如,
以某种你希望的方式对照片文件重新命名,或者信手写一个数独游戏以消磨时间,甚至是在
科学兴趣探索中,进行一些复杂的数字运算,那么Python将是一种非常适合你需求的高级编
程语言,其语法简单,安装测试方便,功能又异常强大,并拥有高效率的高层数据结构和面
向对象特征。付出时间成本投资于Python语言,你一定会取得满意的回报。

Python语言是一种少有的既简单又强大的高级编程语言,注重问题的解决而非编程语言的语
法和结构,正如其官方所介绍:

\begin{enumerate}
\item Python is a programming language that lets you work more quickly and
integrate your systems more effectively.

(Python是一种能够让你工作更便捷、系统集成更高效的编程语言)

\item Python is a clear and powerful object-oriented programming language,
comparable to Perl, Ruby, Scheme, or Java.

(相比于Perl、Ruby、Scheme或Java,Python是一种简单、强大的面向对象的编程语言)
\end{enumerate}

顺便说一句,Python语言的作者Guido van Rossum是根据英国广播公司的节目``Monty
Python's Flying Circus''(蟒蛇飞行马戏)给这个语言命名的,并非他本人特别喜欢蛇缠起它
们的长身躯碾死动物觅食。

目前,Python正处于有2.x版本到3.x版本过渡的过程之中,Python 3又称为Python
3000或Py3K,作为新一代Python语言,Python 3以上版本去除了之前版本中积累的一些老问
题,并使语言更为清晰,但另一方面,由于Python2.x是如此的成功,在大量系统之中得到了
广泛应用,出于移植成本考虑,短时间之内还无法将所有Python代码都升级为Python 3版本,
因此,Python 2和Python 3还将长时间并存\footnote{Python 2.7之后,Python 2.x版本将
  不会再有重大更新,新项目建议使用Python 3.x,到底选择Python2还是Python3,可阅
  读:\url{https://wiki.python.org/moin/Python2orPython3} },考虑到未来发展,本教
程将采用Python 3的语言规范进行讲解\footnote{如果你已有大量的 Python 2.x 代码,也
  可以使用工具将 2.x 转换成 3.x 的源代码,参
  考\url{http://docs.python.org/3.5/library/2to3.html}}。



\section{Python的安装}
如果你已经安装了 Python 2.x ,那么就没必要卸载后再安装 Python 3.x ,实际上,可以将它
们同时安装在电脑里。

\subsection{Linux用户}
如果你正在使用一个Linux的发行版,比如Ubuntu或Fedora,默认情况下,系统里面已经安
装了Python,要测试系统中是否已经安装了Python以及Python的版本,可以打开一个shell
程序,输入如下命令:
\shellbox{\$ python -V

Python 2.7.10
}

注意,参数V为大写形式,如果写成了小写的v,则代表以详细追踪方式显示Python的启动时
的执行内容,而不是显示Python的版本。

2016年以来发布的Linux版本,通常在装有python 2的同时,也安装有python3,以Ubuntu为
例,在命令行上运行python3 -V,可以运行python3,并得到其版本信息,类似于如下:

\shellbox{
\$ python3 -V

Python 3.4.3+ (default, Oct 14 2015, 16:03:50) 
}

\subsection{Mac用户}

Mac OS X 用户 会 发 现已 经 在 系统中 安 装了 Python 。打开 Terminal.app 运行
python -V ,接着参考上面关于 Linux 部分的建议。

\subsection{Windows用户}

访问 http://www.python.org 网站下载最新版,在写本书的时候是 3.0 beta 1 。仅
有 12.8MB ,与大多其它的语言或软件相比,是非常紧凑的。安装与其它的 Win-
dows 软件一样。

有趣的是,大多的 Python 下载是来自 Windows 用户的。当然,这并不能说明问
题,因为几乎所有的 Linux 用户已经在安装系统的时候默认安装了 Python 。


\section{如何运行Python程序}
我们将看一下如何用 Python 编写运行一个传统的“Hello World”程序。通过它,你将学会如
何编写、保存和运行 Python 程序。

有两种使用 Python 运行你的程序的方式 —— 使用交互式的带提示符的解释器或使用源文件。
我们将学习这两种方法。

\subsection{使用带提示符的解释器}
在 shell 提示符下,键入python命令启动解释器。对 Windows 用户,如果你已经配置好
了PATH变量,那么就可在命令行中启动解释器。

如果使用 IDLE ,点击 开始 $\rightarrow$ 程序 $\rightarrow$ Python 3.0
$\rightarrow$ IDLE (Python GUI)。键入 print('Hello World') ,按回车键。将会看
到 Hello World 字样的输出。

注意, Python会在下一行立即给出你输出!对于刚键入的Python语句print('Hello World'),
我们使用print(不要惊讶,这里的打印默认是``打印''到显示器上,即在显示器上输出结果)来
打印你提供给它的值,这里提供的文本是``Hello World'' ,它被立即打印在屏幕上。

至于如何退出解释器提示符,如果你使用的是Linux/BSD shell,那么按``Ctrl-d''退出提示
符,如果是在Windows命令行中,则按``Ctrl-z'',再按回车键退出。

\subsection{运行Python源代码文件}




\section{工作环境的配置}

工欲善其事必先利其器,搭建一个好的工作环境,对于提高生产效率是十分重要的。下面,
我们从源代码编辑器和增强命令行交互工具两个方面对Python工作环境进行介绍。

\subsection{源代码编辑器的选择}

支持Python的编辑器有很多,在一个由多个工具组成的环境中,最好只关注与Python代码编
写相关的编辑器。实际上,简单的代码编辑器和集成开发环境之间的边界并不明显,一些简
单的编辑器也会提供与系统交互和扩展的机制。对于代码编写来说,一个配置齐全的源代码
编辑器可以降低一些重复性工作,有效辅助代码的编写。

多年以来,Python源代码编辑器的最佳选择是vim或Emacs,初次接触这两个编辑器会觉得其
操作并不友好,甚至感觉完全无法使用,这是因为这两个编辑器都引入了大量键盘快捷键,
只有经过一段时间的熟悉之后,才会体会到它们的强大之处,如果你觉得自己还年轻,那么
投入精力去掌握它们,你的付出会让你终生受用。

Vim在多数Linux的发行版和苹果的笔记本操作系统中,默认都已经安装\footnote{部分系统
  默认安装的是vi,而vim是vi的升级版本,兼容vi的所有指令,并拥有大量新特性,如果系
  统默认安装的是vi,建议读者进一步安装vim。},可以直接使用,而Emacs通常需要单独安
装。相比于Vim或Emacs,新潮的开发人员或初学者,可能更喜欢以鼠标操作为主的编辑器,
其中,Sublime Text(http://www.sublimetext.com/)是目前非常流行的编辑器,对Python
的语法支持也很好。

综上,初学者可以先需用Sublime Text作为Python的源代码编辑器,并逐步学习使
用Vim或Emacs,提高熟练程度。

\subsection{扩展Python shell}

\subsubsection{ipython}

\subsubsection{bpython}
bpython是一个不错的Python解释器界面,它将集成开发环境的常见语法提示等常见功能封
装在在一个简单、轻量级的包里,可以在终端窗口里面直接运行,保持python shell的简洁、
快速和实用特点。

bpython可以利用pip快速安装:

\shellbox{
\$pip install bpython
}

bpython的功能十分丰富,包括:

\begin{itemize}
\item 内置的语法高亮 – 使用Pygments排版你敲出的代码,并合理配色
\item 根据你的行为,显示自动补全的建议
\item 为任何Python函数列出所期望的参数 – 可以显示你调用的任何函数的参数列表
\item “Rewind”功能会调出内存里的最后一行代码并重新执行
\item 可以将你输入的代码送到pastebin
\item 可以将你输入的代码保存到一个文件
\item 自动缩进
\item 支持Python 3
\end{itemize}

%%% Local Variables:
%%% mode: latex
%%% TeX-master: "../python"
%%% End:
