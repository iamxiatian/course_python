\begin{thebibliography}{999}
\bibitem {MAA2012} Mohammad Ali Abbasi, Sun-Ki Chai, Huan Liu, and Kiran. Sagoo, Real-world behavior analysis through a social media lens., Social Computing, Behavioral-Cultural Modeling and Prediction, Springer, 2012, pp. 18–26.

\bibitem {Abello2002} J. Abello, M. Resende, and S. Sudarsky, Massive quasi-clique detection., LATIN 2002: Theoretical Informatics (2002), 598–612.

\bibitem {Abrahamson1993} E. Abrahamson and L. Rosenkopf, Institutional and competitive band wagons: Using mathematical modeling as a tool to explore innovation diffusion., Academy of management review (1993), 487–517.

\bibitem {Adamic2003} L.A. Adamic and E. Adar, Friends and neighbors on the web., Social Networks 25 (2003), no. 3, 211–230.


\bibitem {Adomavicius2005} Gediminas Adomavicius and Alexander.Tuzhilin,Toward the nextgeneration of recommender systems: a survey of the state-of-the-art and possible extensions., IEEE Transactions on Knowledge and Data Engineering 17(2005), no.6, 734–749.

\bibitem {N. Agarwal2008} N. Agarwal, H. Liu, L. Tang, and P.S. Yu,Identifying the influential bloggers in a community, Proceedings of the International Conferenceon Web Search and Web Data Mining, ACM, 2008, pp. 207–218.

\bibitem {Orlin1995} R.K. Ahuja, T.L. Magnanti, J.B. Orlin, and K. Weihe, Network flows:theory, algorithms and applications., ZOR-Methods and Models of Operations Research 41 (1995), no. 3, 252–254.

\bibitem {Salem2006} [8] Mohammad Al Hasan, Vineet Chaoji, Saeed Salem, and Mohammed.Zaki, Link prediction using supervised learning., SDM’06: Workshop onLink Analysis, Counter-terrorism and Security, 2006.

\bibitem {Albert2000} R. Albert and A.L. Barab′asi, Topology of evolving networks: local eventsand universality., Physical review letters 85 (2000), no. 24, 5234–5237.

\bibitem {Anagnostopoulos2008} A. Anagnostopoulos, R. Kumar, and M. Mahdian, Influence and correlation in social networks, Proceedings of the 14th ACM SIGKD Dinternational conference on Knowledge Discovery and Data Mining,ACM, 2008, pp. 7–15.

\bibitem {Anderson1996} L.R. Anderson and C.A. Holt, Classroom games: information cascades.,The Journal of Economic Perspectives 10 (1996),no. 4, 187–193.

\bibitem {Charles1997} Lisa R. Anderson and Charles A. Holt , Information cascades in the laboratory., The American Economic Review (1997), 847–862.

\bibitem {Robert1992} Roy M Anderson, Robert M May, and B Anderson, Infectious diseasesof humans: dynamics and control., vol. 26, Wiley Online Library, 1992.

\bibitem {Ankerst1999} M. Ankerst, M.M. Breunig, H.P. Kriegel, and J. Sander, OPTICS:ordering points to identify the clustering structure., Proceedings of the 1999 ACM SIGMOD international conference on Management of Data (1999), 49–60.

\bibitem {Aral2009} S. Aral, L. Muchnik, and A. Sundararajan, Distinguishing influence-based contagion from homophily-driven diffusion in dynamic networks.,Proceedings of the National Academy of Sciences 106 (2009), no. 51,21544–21549.

\bibitem {Jaime2008} Jaime Arguello, Jonathan Elsas, Jamie Callan, and Jaime. Carbonell,Document representation and query expansion models for blog recommendation., Proceedings of the second international conference on Weblogs and Social Media (ICWSM), 2008.

\bibitem {Asch1956} S.E. Asch, Studies of independence and conformity: I. a minority of oneagainst a unanimous majority., Psychological Monographs: Generaland Applied 70 (1956), no. 9, 1–70.

\bibitem {Sitaram2010} Sitaram Asur and Bernardo A. Huberman, Predicting the future with social media. ieee international conference on, Web Intelligence and Intelligent Agent Technology, vol. 1, IEEE, 2010, pp. 492–499.

\bibitem {Backstrom2006} L. Backstrom, D. Huttenlocher, J.M. Kleinberg, and X. Lan, Groupformation in large social networks: membership, growth, and evolution,Proceedings of the 12th ACM SIGKDD international conference onKnowledge Discovery and Data Mining, ACM, 2006, pp. 44–54.

\bibitem {Marlow2010} L. Backstrom, E. Sun, and C. Marlow, Find me if you can: improving geographical prediction with social and spatial proximity., Proceedings of the 19th International conference on World Wide Web, ACM, 2010,pp. 61–70.

\bibitem {Bailey1975} N.T.J. Bailey, The mathematical theory of infectious diseases and its applications., Charles Griffin \& Company Ltd, 1975.

\bibitem {Bakshy2011} E. Bakshy, J.M. Hofman, W.A. Mason, and D.J. Watts, Everyone’s aninfluencer: quantifying influence on twitter, Proceedings of the fourth ACM international conference on Web Search and Data Mining,ACM, 2011, pp. 65–74.

\bibitem {Banerjee1992} A.V. Banerjee, A simple model of herd behavior., The Quarterly Journal of Economics 107 (1992), no. 3, 797–817.

\bibitem {Albert1999} A.L. Barab′asi and R. Albert, Emergence of scaling in random networks.,science 286 (1999), no. 5439, 509–512.

\bibitem {Gundech2013} Geoffrey Barbier, Zhuo Feng, Pritam Gundecha, and Huan. Liu, Morgan \& Claypool Publishers, 2013.

\bibitem {Gabriel2012} Geoffrey Barbier, Reza Zafarani, Huiji Gao, Gabriel Fung, and Huan.Liu, Maximizing benefits from crowd sourced data., Computational and Mathematical Organization Theory 18 (2012), no. 3, 257–279.

\bibitem {Barnes2004} S.J. Barnes and E. Scornavacca, Mobile marketing: the role of permissionand acceptance., International Journal of Mobile Communications 2(2004), no. 2, 128–139.

\bibitem {Alessandro2008} Alain Barrat, Marc Barthelemy, and Alessandro. Vespignani, Dynamical processes on complex networks., vol. 1, Cambridge UniversityPress, 2008.

\bibitem {Barwise2002} P. Barwise and C. Strong, Permission-based mobile advertising., Journal of Interactive Marketing 16 (2002), no. 1, 14–24.

\bibitem {Bass1969} F. Bass, A new product growth model for product diffusion., ManagementScience 15 (1969), 215–227.

\bibitem {Bell1924} W.G. Bell, The great plague in london in 1665., The Great Plague inLondon in 1665. (1924).

\bibitem {Bellma1956} R. Bellma, On a routing problem., NOTES 16 (1956), no. 1.

\bibitem {Bierlaire1998} M. Ben-Akiva, M. Bierlaire, H. Koutsopoulos, and R. Mishalani, Dy-namit: a simulation-based system for traffic prediction., DACCORS ShortTerm Forecasting Workshop, The Netherlands, 1998.

\bibitem {Berger2001} E. Berger, Dynamic monopolies of constant size., Journal of Combinatorial Theory, Series B 83 (2001), no. 2, 191–200.

\bibitem {Berkhin2006} P. Berkhin, A survey of clustering data mining techniques., Grouping Multidimensional Data (2006), no. c, 25–71.

\bibitem {Russell2012} H. Russell. Bernard, Social research methods: qualitative and quantitative approaches., Sage, 2012.

\bibitem {Bikhchandani1992} S. Bikhchandani, D. Hirshleifer, and I. Welch, A theory of fads, fashion, custom, and cultural change as informational cascades., Journal of Political Economy (1992), 992–1026.

\bibitem {Sharma2000} S. Bikhchandani and S. Sharma, Herd behavior in financial markets.,IMF Staff Papers (2000), 279–310.

\bibitem {Bishop1995} C.M. Bishop, Neural networks for pattern recognition., (1995).

\bibitem {Christopher2006} Christopher M. Bishop, Pattern recognition and machine learning., vol. 4, Springer,2006.

\bibitem {Bollob2001} B. Bollob′as, Random graphs., vol. 73, Cambridge University Press,2001.

\bibitem {Bonabeau1999} E. Bonabeau, M. Dorigo, and G. Theraulaz, Swarm intelligence: fromnatural to artificial systems., no. 1, Oxford University Press, 1999.

\bibitem {James2010} James Davidson, Benjamin Liebald, Junning Liu, Palash Nandy, Taylor Van Vleet, Ullas Gargi, Sujoy Gupta, Yu He, Mike Lambert, BlakeLivingston, and et al., The youtube video recommendation system., Proceedings of the fourth ACM conference on Recommender Systems,ACM, 2010, pp. 293–296.

\bibitem {Davies1979} D.L. Davies and D.W. Bouldin, A cluster separation measure., IEEE Transactions on Pattern Analysis and Machine Intelligence (1979),no. 2, 224–227.

\bibitem {Jarlais1994} D.C. Des Jarlais, S.R. Friedman, J.L. Sotheran, J. Wenston, M. Yan-covitz Marmor, Frank S.R., Beatrice B., and Mildvan D. S., Continuityand change within an hiv epidemic., JAMA: the journal of the American Medical Association 271 (1994), no. 2, 121–127.

\bibitem {Devenow1996} A. Devenow and I. Welch, Rational herding in financial economics.,European Economic Review 40 (1996), no. 3, 603–615.

\bibitem {Dia2001} H. Dia, An object-oriented neural network approach to short-term traffic forecasting., European Journal of Operational Research 131 (2001),no. 2, 253–261.

\bibitem {Diestel2005} R. Diestel, Graph theory. 2005., Graduate Texts in Math (2005).

\bibitem {Dietz1967} K. Dietz, Epidemics and rumours: a survey., Journal of the Royal Statistical Society. Series A (General) (1967), 505–528.

\bibitem {Dodds2004} P.S. Dodds and D.J. Watts, Universal behavior in a generalized model of contagion., Physical Review Letters 92 (2004), no. 21, 218701.

\bibitem {Drehmann2005} M. Drehmann, J. Oechssler, and A. Roider, Herding and contrarian behavior in financial markets – an internet experiment., (2005).

\bibitem {Richard2012} Richard O. Duda, Peter E. Hart, and David G. Stork, Pattern classification., Wiley-interscience, 2012.

\bibitem {Dunn1974} J.C. Dunn, Well-separated clusters and optimal fuzzy partitions., Journal of cybernetics 4 (1974), no. 1, 95–104.

\bibitem {Dye2003} C. Dye and N. Gay, Modeling the sars epidemic., Science 300 (2003),no. 5627, 1884–1885.

\bibitem {Nicholas2009} Nicholas A. Christakis and James H. Fowler, Connected: The surprising power of our social networks and how they shape our lives., Norskepidemiologi= Norwegian Journal of Epidemiology 19 (2009), no. 1,5.

\bibitem {Chung1997} F.R.K. Chung, Spectral graph theory., no. 92, American Mathematical Society, 1997.

\bibitem {Cialdini1998} R. B. Cialdini and M. R. Trost., Social influence: Social norms, conformity and compliance., (1998), 151.

\bibitem {Aaron2009} Aaron Clauset, Cosma Rohilla Shalizi, and Mark EJ. Newman, Power-law distributions in empirical data., SIAM Review 51 (2009), no. 4, 661–703.

\bibitem {Coleman1966} J.S. Coleman, E. Katz, and H. Menzel, Medical innovation: a diffusionstudy., Bobbs-Merrill Company, 1966.

\bibitem {Cont2000} R. Cont and J.P. Bouchaud, Herd behavior and aggregate fluctuations infinancial markets., Macroeconomic Dynamics 4 (2000), no. 02, 170–196.

\bibitem {Thomas2009} Thomas H. Cormen, Charles E. Leiserson, Ronald L. Rivest, and Clifford. Stein, Introduction to algorithms., 2009.

\bibitem {Currarini2009} S. Currarini, M.O. Jackson, and P. Pin, An economic model of friendship:Homophily, minorities, and segregation., Econometrica 77 (2009), no. 4,1003–1045.

\bibitem {Abhinandan2007} Abhinandan S. Das, Mayur Datar, Ashutosh Garg, and Shyam. Ra-jaram, Google news personalization: scalable online collaborative filtering.,Proceedings of the 16th international conference on World Wide Web,ACM, 2007, pp. 271–280.

\bibitem {Manoranjan1997} Manoranjan Dash and Huan. Liu, Feature selection for classification.,Intelligent Data Analysis 1 (1997), no. 3, 131–156.

\bibitem {Volker2000} Volker Roth and Tilman Lange, Feature selection for clustering., Knowledge Discovery and Data Mining. Current Issues and New Applications, Springer, 2000,pp. 110–121.350

\bibitem {Bondy1976} J.A. Bondy and U.S.R. Murty, Graph theory with applications., vol. 290,MacMillan London, 1976.

\bibitem {Stephen2004} Stephen Poythress Boyd and Lieven. Vandenberghe, Convex optimization., Cambridge University Press, 2004.

\bibitem {Ulrik2001} Ulrik. Brandes, A faster algorithm for betweenness centrality., Journal of Mathematical Sociology 25 (2001), no. 2, 163–177.

\bibitem {Kumar2000} Andrei Broder, Ravi Kumar, Farzin Maghoul, Prabhakar Raghavan,Sridhar Rajagopalan, Raymie Stata, Andrew Tomkins, and Janet.Wiener, Graph structure in the web., Computer networks 33 (2000),no. 1, 309–320.

\bibitem {Bryman2012} Alan. Bryman, Social research methods., Oxford University Press, 2012.

\bibitem {Burke2002} Robin. Burke, Hybrid recommender systems: Survey and experiments.,User Modeling and User-Adapted Interaction 12 (2002), no. 4, 331–370.

\bibitem {James2010} K. Selcuk Candan and Maria Luisa. Sapino, Data management formultimedia retrieval., Cambridge University Press, 2010.

\bibitem {Haddadi2010} M. Cha, H. Haddadi, F. Benevenuto, and K.P. Gummadi, Measuringuser influence in twitter: The million follower fallacy, AAAI Conferenceon Weblogs and Social Media, vol. 14, 2010, p. 8.

\bibitem {Soumen2003} Soumen. Chakrabarti, Mining the Web: discovering knowledge from hypertext data., Morgan Kaufmann, 2003.

\bibitem {Geyer2009} Jilin Chen, Werner Geyer, Casey Dugan, Michael Muller, and Ido.Guy, Make new friends, but keep the old: recommending people on social networking sites., Proceedings of the SIGCHI Conference on Human Factors in Computing Systems, ACM, 2009, pp. 201–210.

\bibitem {Chinese2004} S. Chinese and et al., Molecular evolution of the sars coronavirus duringthe course of the sars epidemic in china., Science 303 (2004), no. 5664,1666.

\bibitem {Christakis2007} N.A. Christakis and J.H. Fowler, The spread of obesity in a large socialnetwork over 32 years., New England Journal of Medicine 357 (2007),no. 4, 370–379.349

\bibitem {Easley2010} D. Easley and J.M. Kleinberg, Networks, crowds, and markets., Cambridge University Press, 2010.

\bibitem {Eberhart2001} R.C. Eberhart, Y. Shi, and J. Kennedy, Swarm intelligence., Elsevier,2001.

\bibitem {Edmonds1972} J. Edmonds and R.M. Karp, Theoretical improvements in algorithmic effciency for network flow problems., Journal of the ACM (JACM) 19(1972), no. 2, 248–264.

\bibitem {Nicole2007} Nicole B. Ellison and et al., Social network sites: definition, history,and scholarship., Journal of Computer-Mediated Communication 13(2007), no. 1, 210–230.

\bibitem {Engelbrecht2005} A.P. Engelbrecht, Fundamentals of computational swarm intelligence.,Recherche 67 (2005), 02.

\bibitem {Erdos1959} P. Erdos and A. R′enyi, On random graphs., Publicationes Mathematicae Debrecen 6 (1959), 290–297.

\bibitem {Erdos1960} P. Erdos and A. R′enyi, On the evolution of random graphs., Akad. Kiad′o, 1960.

\bibitem {Erdosi1961} P. Erdos and A. R′enyi, On the strength of connectedness of a random graph., Acta Mathematica Hungarica 12 (1961), no. 1, 261–267.

\bibitem {Ester1996} M. Ester, H.P. Kriegel, J. Sander, and X. Xu, A density-based algorithmfor discovering clusters in large spatial databases with noise., Proceedings of the second international conference on Knowledge Discovery andData Mining, Portland, OR, AAAI Press (1996), 226–231.

\bibitem {Faloutsos1999} M. Faloutsos, P. Faloutsos, and C. Faloutsos, On power-law relationships of the internet topology., ACM SIGCOMM Computer Communication Review, vol. 29, ACM, 1999, pp. 251–262.

\bibitem {Fisher1987} D. Fisher, Improving inference through conceptual clustering., Proceedings of the 1987 AAAI conference (1987), 461–465.

\bibitem {Floyd1962} R.W. Floyd, Algorithm 97: shortest path., Communications of the ACM5 (1962), no. 6, 345.

\bibitem {Fulkerson1956} L.R. Ford and D.R. Fulkerson, Maximal flow through a network., Canadian Journal of Mathematics 8 (1956), no. 3, 399–404.

\bibitem {Fortunato2010} S. Fortunato, Community detection in graphs., Physics Reports 486(2010), no. 3-5, 75–174.

\bibitem {Friedman2001} J. Friedman, T. Hastie, and R. Tibshirani, The elements of statistical learning., vol. 1, Springer Series in Statistics, 2001.

\bibitem {Gale1996} D. Gale, What have we learned from social learning, European Economic Review 40 (1996), no. 3, 617–628.

\bibitem {Barbier2011} H. Gao, G. Barbier, and R. Goolsby, Harnessing the crowdsourcingpower of social media for disaster relief., Intelligent Systems, IEEE 26(2011), no. 3, 10–14.

\bibitem {Tang2012} H. Gao, J. Tang, and H. Liu, Exploring social-historical ties on location-based social networks., Proceedings of the Sixth International Conference on Weblogs and Social Media, 2012.

\bibitem {spatio2012} H. Gao, J. Tang, and H. Liu, Mobile location prediction in spatio-temporal context., NokiaMobile Data Challenge Workshop (2012).

\bibitem {Jiliang2012} Huiji Gao, Jiliang Tang, and Huan. Liu, gscorr: modeling geo-social correlations for new checkins on location-based social networks., Proceedings of the 21st ACM international conference on Information andKnowledge Management, ACM, 2012, pp. 1582–1586.

\bibitem {Huiji2011} Huiji Gao, Xufei Wang, Georey Barbier, and Huan. Liu, Promoting coordination for disaster relief – from crowdsourcing to coordination.,Social Computing, Behavioral-Cultural Modeling and Prediction,Springer, 2011, pp. 197–204.

\bibitem {Gibson2005} D. Gibson, R. Kumar, and A. Tomkins, Discovering large dense sub-graphs in massive graphs, Discovering large dense subgraphs in massive graphs., VLDB Endowment, 2005, pp. 721–732.

\bibitem {Gilbert1959} E.N. Gilbert, Random graphs., The Annals of Mathematical Statistics30 (1959), no. 4, 1141–1144.

\bibitem {Girvan2002} M. Girvan and M.E.J. Newman, Community structure in social andbiological networks., Proceedings of the National Academy of Sciences99 (2002), no. 12, 7821.

\bibitem {Jennifer2006} Jennifer Golbeck and James. Hendler, Filmtrust: movie recommendations using trust in web-based social networks., Proceedings of the IEEE Consumer Communications and Networking Conference, vol. 96,Citeseer, 2006.

\bibitem {Goldberg1988} A.V. Goldberg and R.E. Tarjan, A new approach to the maximum-flow problem., Journal of the ACM (JACM) 35 (1988), no. 4, 921–940.

\bibitem {Golub2010} B. Golub and M.O. Jackson, Naive learning in social networks and the wisdom of crowds., American Economic Journal: Microeconomics 2(2010), no. 1, 112–149.

\bibitem {Goodchild2010} M.F. Goodchild and J.A. Glennon, Crowd sourcing geographic informa-tion for disaster response: a research frontier., International Journal ofDigital Earth 3 (2010), no. 3, 231–241.

\bibitem {Goodman732} L. Goodman and W. Kruskal, Measures of associations for cross-validations., Journal of the American Statistical Association 49, 732–764.

\bibitem {Goyal2010} A. Goyal, F. Bonchi, and L.V.S. Lakshmanan, Learning influence probabilities in social networks, Proceedings of the Third ACM international conference on Web Search and Data Mining, ACM, 2010, pp. 241–250.

\bibitem {Granovetter1978} M. Granovetter, Threshold models of collective behavior., American Journal of Sociology (1978), 1420–1443.

\bibitem {Granovetter1973} M.S. Granovetter, The strength of weak ties., American Journal of Sociology (1973), 1360–1380.

\bibitem {Gray1973} V. Gray, Innovation in the states: a diffusion study., The American Political Science Review 67 (1973), no. 4, 1174–1185.

\bibitem {Griliches1957} Z. Griliches, Hybrid corn: an exploration in the economics of technological change., Econometrica, Journal of the Econometric Society (1957),501–522.

\bibitem {Gruhl2004} D. Gruhl, R. Guha, D. Liben-Nowell, and A. Tomkins, Information diffusion through blogspace, Proceedings of the 13th international conference on the World Wide Web, ACM, 2004, pp. 491–501.

\bibitem {Guan2007} Y. Guan, H. Chen, K.S. Li, S. Riley, G.M. Leung, R. Webster, J.S.M.Peiris, and K.Y. Yuen, A model to control the epidemic of h5n1 influenzaat the source., BMC Infectious Diseases 7 (2007), no. 1, 132.

\bibitem {Pritam2011} Pritam Gundecha, Geoffrey Barbier, and Huan. Liu, Exploiting Vulnerability to Secure User Privacy on a Social Networking Site., Proceedings of the 17th ACM SIGKDD international conference on Knowledge Discovery and Data Mining, KDD, 2011, pp. 511–519.

\bibitem {Pritam2012} Pritam Gundecha and Huan. Liu, Mining social media: a brief introduction., Tutorials in Operations Research 1 (2012), no. 4.

\bibitem {Zwerdling2010} Ido Guy, Naama Zwerdling, Inbal Ronen, David Carmel, and Erel.Uziel, Social media recommendation based on people and tags., Proceedings of the 33rd international ACM SIGIR conference on Researchand Development in Information Retrieval, ACM, 2010, pp. 194–201.

\bibitem {Isabelle2006} Isabelle. Guyon, Feature extraction: foundations and applications., vol.207, Springer, 2006.

\bibitem {Hagerstrand1968} T. Hagerstrand and et al., Innovation diffusion as a spatial process.,Innovation diffusion as a spatial process. (1968).

\bibitem {Hamblin1973} R.L. Hamblin, R.B. Jacobsen, and J.L.L. Miller, A mathematical theory of social change., Wiley, 1973.

\bibitem {Micheline2006} Jiawei Han, Micheline Kamber, and Jian. Pei, Data mining: conceptsand techniques., Morgan Kaufmann, 2006.

\bibitem {Handcock2007} M.S. Handcock, A.E. Raftery, and J.M. Tantrum, Model-based clustering for social networks., Journal of the Royal Statistical Society: SeriesA (Statistics in Society) 170 (2007), no. 2, 301–354.

\bibitem {Nilsson1968} P.E. Hart, N.J. Nilsson, and B. Raphael, A formal basis for the heuristic determination of minimum cost paths., Systems Science and Cybernetics, IEEE Transactions on 4 (1968), no. 2, 100–107.

\bibitem {Simon1994} Simon. Haykin, Neural networks: a comprehensive foundation., PrenticeHall, 1994.

\bibitem {Hethcote1994} H.W. Hethcote, A thousand and one epidemic models., Lecture Notes in Biomathematics (1994), 504–504.

\bibitem {H2000} H.W. Hethcote, The mathematics of infectious diseases., SIAM review (2000),599–653.

\bibitem {Hethcote1981} H.W. Hethcote, H.W. Stech, and P. van den Driessche, Periodicity and stability in epidemic models: a survey., Differential Equations and Applications in Ecology, Epidemics and Population Problems (SNBusenberg and KL Cooke, eds.) (1981), 65–82.

\bibitem {Hirschman1980} E.C. Hirschman, Innovativeness, novelty seeking, and consumer creativity., Journal of Consumer Research (1980), 283–295.

\bibitem {Hirshleifer1997} D. Hirshleifer, Informational cascades and social conventions., University of Michigan Business School Working Paper No. 9705-10 (1997).

\bibitem {Raftery2002} P.D. Hoff, A.E. Raftery, and M.S. Handcock, Latent space approaches tosocial network analysis., Journal of the American Statistical Association97 (2002), no. 460, 1090–1098.

\bibitem {Hopcroft1973} J. Hopcroft and R. Tarjan, Algorithm 447: e?cient algorithms for graphmanipulation., Communications of the ACM 16 (1973), no. 6, 372–378.

\bibitem {Xia Hu2013} Xia Hu, Jiliang Tang, Huiji Gao, and Huan. Liu, Unsupervised sentiment analysis with emotional signals., Proceedings of the 22nd international conference on World Wide Web, WWW’13, ACM, 2013.

\bibitem {Xia Hu2013} Xia Hu, Lei Tang, Jiliang Tang, and Huan. Liu, Exploiting social relations for sentiment analysis in microblogging., Proceedings of the sixth ACM international conference on Web Search and Data Mining, 2013.

\bibitem {Jaccard1901} P. Jaccard, Distribution de la Flore Alpine: dans le Bassin des dranses etdans quelques r′egions voisines., Rouge, 1901.

\bibitem {Jackson2008} M.O. Jackson, Social and economic networks., Princeton UniversityPress, 2008.

\bibitem {Jain1988} A.K. Jain and R.C. Dubes, Algorithms for clustering data., Prentice-Hall, 1988.

\bibitem {Murty1999} A.K. Jain, M.N. Murty, and P.J. Flynn, Data clustering: a review., ACMComputing Surveys (CSUR) 31 (1999), no. 3, 264–323.

\bibitem {Anthony2007} Anthony Jameson and Barry. Smyth, Recommendation to groups., The Adaptive Web, Springer, 2007, pp. 596–627.

\bibitem {Dietmars2010} Dietmar Jannach, Markus Zanker, Alexander Felfernig, and Gerhard. Friedrich, Recommender systems: an introduction., Cambridge University Press, 2010.

\bibitem {Jensen1995} T.R. Jensen and B. Toft, Graph coloring problems., vol. 39, Wiley-Interscience, 1995.

\bibitem {Charles2012} Charles. Kadushin, Understanding Social Networks: theories, concepts,and findings: theories, concepts, and findings., Oxford University Press,USA, 2012.

\bibitem {Andreas2010} Andreas M. Kaplan and Michael. Haenlein, Users of the world, unite!the challenges and opportunities of social media., Business horizons 53(2010), no. 1, 59–68.

\bibitem {Karypis1999} G. Karypis, E.H. Han, and V. Kumar, Chameleon: hierarchical clustering using dynamic modeling., Computer 32 (1999), no. 8, 68–75.

\bibitem {Christakis2007} E. Katz and P.F. Lazarsfeld, Personal influence: the part played by peoplein the flow of mass communications., Transaction Pub, 2006.

\bibitem {Keeling2005} M.J. Keeling and K.T.D. Eames, Networks and epidemic models., Journalof the Royal Society Interface 2 (2005), no. 4, 295–307.

\bibitem {Keller2003} E. Keller and J. Berry, The influentials: One American in ten tells theother nine how to vote, where to eat, and what to buy., Free Press, 2003.

\bibitem {Kleinberg2003} D. Kempe, J.M. Kleinberg, and′E. Tardos, Maximizing the spread of influence through a social network, Proceedings of the ninth ACMSIGKDD international conference on Knowledge Discovery and Data Mining, ACM, 2003, pp. 137–146.

\bibitem {Kennedy2006} J. Kennedy, Swarm intelligence., Handbook of Nature-Inspired and Innovative Computing (2006), 187–219.

\bibitem {Kermack1932} W.O. Kermack and A.G. McKendrick, Contributions to the mathematical theory of epidemics. ii. the problem of endemicity., Proceedings of the Royal society of London. Series A 138 (1932), no. 834, 55–83.

\bibitem {Kietzmann2011} Jan H. Kietzmann, Kristopher Hermkens, Ian P. McCarthy, andBruno S. Silvestre, Social media? get serious! understanding the functional building blocks of social media., Business Horizons 54 (2011), no. 3,241–251.

\bibitem {Kleinberg1999} J.M. Kleinberg, Authoritative sources in a hyperlinked environment.,Journal of the ACM (JACM) 46 (1999), no. 5, 604–632.

\bibitem {Kleinberg2006} Jon Kleinberg and′Eva. Tardos, Algorithm design. addison wesley.,(2006).

\bibitem {Ioannis2009} Ioannis Konstas, Vassilios Stathopoulos, and Joemon M. Jose, Onsocial networks and collaborative recommendation., Proceedings of the32nd international ACM SIGIR conference on Research and Development in Information Retrieval, ACM, 2009, pp. 195–202.

\bibitem {Kosala2000} R. Kosala and H. Blockeel, Web mining research: a survey., ACMSigkdd Explorations Newsletter 2 (2000), no. 1, 1–15.

\bibitem {Kossinets2006} G. Kossinets and D.J. Watts, Empirical analysis of an evolving social network., Science 311 (2006), no. 5757, 88.

\bibitem {Krapivsky2000} P.L. Krapivsky, S. Redner, and F. Leyvraz, Connectivity of growing random networks., Physical review letters 85 (2000), no. 21, 4629–4632.

\bibitem {Kruskal1956} J.B. Kruskal, On the shortest spanning subtree of a graph and the traveling salesman problem., Proceedings of the American Mathematical Society7 (1956), no. 1, 48–50.

\bibitem {Kumar2010} R. Kumar, J. Novak, and A. Tomkins, Structure and evolution of online social networks., Link Mining: Models, Algorithms, and Applications(2010), 337–357.

\bibitem {Raghavan1999} R. Kumar, P. Raghavan, S. Rajagopalan, and A. Tomkins, Trawling theweb for emerging cyber-communities., Computer networks 31 (1999),no. 11-16, 1481–1493.

\bibitem {Zafarani2011} S. Kumar, R. Zafarani, and H. Liu, Understanding user migration patterns in social media, 25th AAAI Conference on Artificial Intelligence,2011.

\bibitem {Shamanth2013} Shamanth Kumar, Fred Morstatter, Reza Zafarani, and Huan. Liu,Whom should i follow? identifying relevant users during crises., Proceedings of the 24th ACM Conference on Hypertext and Social Media,2013.

\bibitem {Neville2010} T. La Fond and J. Neville, Randomization tests for distinguishing social influence and homophily effects, Proceedings of the 19th international conference on the World Wide Web, ACM, 2010, pp. 601–610.

\bibitem {Lancichinetti2009} A. Lancichinetti and S. Fortunato, Community detection algorithms: a comparative analysis., Physical Review E 80 (2009), no. 5, 056117.

\bibitem {Langley1996} P. Langley, Elements of machine learning., Morgan Kaufmann, 1996.

\bibitem {Charles1995} Charles L. Lawson and Richard J. Hanson, Solving least squares problems, vol. 15, SIAM, 1995.

\bibitem {Leibenstein1950} H. Leibenstein, Bandwagon, snob, and veblen effects in the theory ofconsumers’ demand., The Quarterly Journal of Economics 64 (1950),no. 2, 183–207.

\bibitem {Leicht2006} E.A. Leicht, P. Holme, and M.E.J. Newman, Vertex similarity in net-works., Physical Review E 73 (2006), no. 2, 026120.

\bibitem {Kleinberg2005} J. Leskovec, J.M. Kleinberg, and C. Faloutsos, Graphs over time: densification laws, shrinking diameters and possible explanations, Proceedings of the 11th ACM SIGKDD international conference on Knowledge Discovery in Data Mining, ACM, 2005, pp. 177–187.

\bibitem {Leskovec2010} J. Leskovec, K.J. Lang, and M. Mahoney, Empirical comparison of algorithms for network community detection, Proceedings of the 19th international conference on the World Wide Web, ACM, 2010, pp. 631–640.

\bibitem {McGlohon2007} J. Leskovec, M. McGlohon, C. Faloutsos, N. Glance, and M. Hurst,Cascading behavior in large blog graphs., Arxiv preprint arXiv:0704.2803(2007).

\bibitem {Backstrom2009} Jure Leskovec, Lars Backstrom, and Jon. Kleinberg, Meme-trackingand the dynamics of the news cycle., Proceedings of the 15th ACMSIGKDD international conference on Knowledge Discovery and Data Mining, ACM, 2009, pp. 497–506.

\bibitem {Lewis2009} T.G. Lewis, Network Science: theory and Applications., Wiley Publishing, 2009.

\bibitem {Kleinberg2007} D. Liben-Nowell and J.M. Kleinberg, The link-prediction problem for social networks., Journal of the American society for information science and technology 58 (2007), no. 7, 1019–1031.

\bibitem {Lietsala2008} Katri Lietsala and Esa. Sirkkunen, Social media. introduction to the tools and processes of participatory economy, tampere university., (2008).

\bibitem {Liu2007} B. Liu, Web data mining: exploring hyperlinks, contents, and usage data.,Springer Verlag, 2007.

\bibitem {Huan Liu1998} Huan Liu and Hiroshi. Motoda, Feature extraction, construction and selection: a data mining perspective., Springer, 1998.

\bibitem {Huan Liu2005} Huan Liu and Lei. Yu, Toward integrating feature selection algorithms for classification and clustering., IEEE Transactions on Knowledge andData Engineering 17 (2005), no. 4, 491–502.

\bibitem {Jiahui Liu2010} Jiahui Liu, Peter Dolan, and Elin Ronby. Pedersen, Personalized news recommendation based on click behavior., Proceedings of the 15th international conference on Intelligent User Interfaces, ACM, 2010,pp. 31–40.

\bibitem {Lorrain1971} F. Lorrain and H.C. White, Structural equivalence of individuals in socialnetworks., Journal of Mathematical Sociology 1 (1971), no. 1, 49–80.

\bibitem {Linyuan2011} Linyuan Lu and Tao. Zhou, Link prediction in complex networks: asurvey., Physica A: Statistical Mechanics and its Applications 390(2011), no. 6, 1150–1170.

\bibitem {Michael2009} Hao Ma, Michael R. Lyu, and Irwin. King, Learning to recommend with trust and distrust relationships., Proceedings of the third ACM conference on Recommender Systems, ACM, 2009, pp. 189–196.

\bibitem {Haixuan2008} Hao Ma, Haixuan Yang, Michael R. Lyu, and Irwin. King, Sorec:social recommendation using probabilistic matrix factorization., Proceedings of the 17th ACM conference on Information and Knowledge Management, ACM, 2008, pp. 931–940.

\bibitem {Hao Ma2011} Hao Ma, Dengyong Zhou, Chao Liu, Michael R. Lyu, and Irwin.King, Recommender systems with social regularization., Proceedings of the fourth ACM international conference on Web Search and Data Mining, ACM, 2011, pp. 287–296.

\bibitem {Macy1991} M.W. Macy, Chains of cooperation: threshold effects in collective action.,American Sociological Review (1991), 730–747.

\bibitem {Willer2002} M.W. Macy and R. Willer, From factors to actors: computational sociology and agent-based modeling., Annual Review of Sociology (2002), 143–166.

\bibitem {Mahajan1985} V. Mahajan, Models for innovation diffusion., no. 48, Sage Publications,1985.

\bibitem {Muller1982} V. Mahajan and E. Muller, Innovative behavior and repeat purchase diffusion models, Proceedings of the American Marketing Educators Conference, vol. 456, 1982, p. 460.

\bibitem {Peterson1978} V. Mahajan and R.A. Peterson, Innovation diffusion in a dynamic potential adopter population., Management Science (1978), 1589–1597.

\bibitem {Mansfield1961} E. Mansfield, Technical change and the rate of imitation., Econometrica:Journal of the Econometric Society (1961), 741–766.

\bibitem {Martino1993} J.P. Martino, Technological forecasting for decision making., McGraw-Hill, 1993.

\bibitem {Paolo2004} Paolo Massa and Paolo. Avesani, Trust-aware collaborative filtering for recommender systems., On the Move to Meaningful Internet Systems2004: CoopIS, DOA, and ODBASE, Springer, 2004, pp. 492–508.

\bibitem {McKay1981} B.D. McKay, Practical graph isomorphism, Practical Graph Isomorphism., vol. 1, Utilitas Mathematica, 1981, p. 45.

\bibitem {Cook2001} M. McPherson, L. Smith-Lovin, and J.M. Cook, Birds of a feather:homophily in social networks., Annual Review of Sociology (2001),415–444.

\bibitem {Dowling1978} D.F. Midgley and G.R. Dowling, Innovativeness: the concept and its measurement., Journal of Consumer Research (1978), 229–242.

\bibitem {Milgram2009} S. Milgram, Obedience to authority: an experimental view., Harper Perennial Modern Classics, 2009.

\bibitem {Berkowitz1969} S. Milgram, L. Bickman, and L. Berkowitz, Note on the drawing power of crowds of different size., Journal of Personality and Social Psychology;Journal of Personality and Social Psychology 13 (1969), no. 2, 79.

\bibitem {Cooper1985} G.W. Milligan and M.C. Cooper, An examination of procedures for determining the number of clusters in a data set., Psychometrika 50 (1985),no. 2, 159–179.

\bibitem {Mirkin2005} B.G. Mirkin, Clustering for data mining: a data recovery approach., Chap-man \& Hall/CRC, 2005.

\bibitem {Mislove2007} Alan Mislove, Massimiliano Marcon, Krishna P. Gummadi, Peter Dr-uschel, and Bobby. Bhattacharjee, Measurement and analysis of onlinesocial networks., Proceedings of the 7th ACM SIGCOMM conferenceon Internet measurement, ACM, 2007, pp. 29–42.

\bibitem {Mitchell1997} T.M. Mitchell, Machine learning. wcb., Mac Graw Hill (1997), 368.

\bibitem {Pinelli2009} A. Monreale, F. Pinelli, R. Trasarti, and F. Giannotti, Wherenext: alocation predictor on trajectory pattern mining., Proceedings of the 15thACM SIGKDD international conference on Knowledge Discovery and Data Mining, ACM, 2009, pp. 637–646.

\bibitem {Moore2000} C. Moore and M.E.J. Newman, Epidemics and percolation in small-world networks., Physical Review E 61 (2000), no. 5, 5678.

\bibitem {Morris2000} S. Morris, Contagion., The Review of Economic Studies 67 (2000),no. 1, 57–78.

\bibitem {Morstatter2013} Fred Morstatter, Jurgen Pfeffer, Huan Liu, and Kathleen M. Carley,Is the sample good enough? comparing data from twitter’s streaming api with twitter’s firehose., Proceedings of ICWSM (2013).

\bibitem {Raghavan2010} R. Motwani and P. Raghavan, Randomized algorithms., Chapman \& Hall/CRC, 2010.

\bibitem {Myung2003} I.J. Myung, Tutorial on maximum likelihood estimation., Journal of Mathematical Psychology 47 (2003), no. 1, 90–100.

\bibitem {Nelson2007} M.I. Nelson and E.C. Holmes, The evolution of epidemic influenza.,Nature reviews genetics 8 (2007), no. 3, 196–205.

\bibitem {Nemhauser1988} George L. Nemhauser and Laurence A. Wolsey, Integer and combinatorial optimization., vol. 18, Wiley New York, 1988.

\bibitem {Wasserman1996} John Neter, William Wasserman, Michael H. Kutner, and et al., Applied linear statistical models., vol. 4, Irwin Chicago, 1996.

\bibitem {Newman2003} M.E.J. Newman, Mixing patterns in networks., Physical Review E 67(2003), no. 2, 026126.

\bibitem {M2003} M.E.J. Newman, Random graphs as models of networks., Handbook of graphsand networks (2003), 35–68.

\bibitem {M2006} M.E.J. Newman, Modularity and community structure in networks., Proceedings of the National Academy of Sciences 103 (2006), no. 23, 8577.

\bibitem {E2010} M.E.J. Newman, Networks: an introduction., Oxford University Press, 2010.

\bibitem {Barabasi2006} M.E.J. Newman, A.L. Barabasi, and D.J. Watts, The structure and dynamics of networks., Princeton University Press, 2006.

\bibitem {Forrest2002} M.E.J. Newman, S. Forrest, and J. Balthrop, Email networks and thespread of computer viruses., Physical Review E 66 (2002), no. 3, 035101.

\bibitem {Girvan2003} M.E.J. Newman and M. Girvan, Mixing patterns and community structure in networks., Statistical Mechanics of Complex Networks (2003),66–87.

\bibitem {Strogatz2007} M.E.J. Newman, S.H. Strogatz, and D.J. Watts, Random graphs with arbitrary degree distributions and their applications., Physical Review E64, 026118.

\bibitem {Watts2002} M.E.J. Newman, D.J. Watts, and S.H. Strogatz, Random graph modelsof social networks., Proceedings of the National Academy of Sciencesof the United States of America 99 (2002), no. Suppl 1, 2566.

\bibitem {Ng1994} R.T. Ng and J. Han, Efficient and Effective Clustering Methods for Spatial Data Mining., Proceedings of the 20th International Conference onVery Large Data Bases (1994), 144–155.

\bibitem {Nocedal2006} Jorge Nocedal and S. Wright, Numerical optimization, series in operations research and financial engineering., Springer (2006).

\bibitem {Clarke1926} J. Nohl, C.H. Clarke, and et al., The Black Death. A Chronicle of thePlague. westholme., The Black Death. A Chronicle of the Plague. (1926).

\bibitem {Brendan2010} Brendan O’Connor, Ramnath Balasubramanyan, Bryan R. Routledge, and Noah A. Smith, From tweets to polls: linking text sentimentto public opinion time series., Proceedings of the International AAAI Conference on Weblogs and Social Media, 2010, pp. 122–129.

\bibitem {Donovan2005} John O’Donovan and Barry. Smyth, Trust in recommender systems.,Proceedings of the 10th international conference on Intelligent UserInterfaces, ACM, 2005, pp. 167–174.

\bibitem {Onnela2010} J.P. Onnela and F. Reed-Tsochas, Spontaneous emergence of social influence in online systems., Proceedings of the National Academy ofSciences 107 (2010), no. 43, 18375–18380.

\bibitem {Page1999} L. Page, S. Brin, R. Motwani, and T. Winograd, The pagerank citation ranking: bringing order to the web, (1999).

\bibitem {Palla2005} G. Palla, I. Der′enyi, I. Farkas, and T. Vicsek, Uncovering the overlapping community structure of complex networks in nature and society., Nature435 (2005), no. 7043, 814–818.

\bibitem {Barab2007} Gergely Palla, Albert-L′aszl′o Barab′asi, and Tam′as. Vicsek, Quantifying social group evolution., Nature 446 (2007), no. 7136, 664–667.

\bibitem {Lillian2008} Bo Pang and Lillian. Lee, Opinion mining and sentiment analysis., Foundations and Trends in Information Retrieval 2 (2008), no. 1-2, 1–135.

\bibitem {Papadimitriou1998} Christos H. Papadimitriou and Kenneth. Steiglitz, Combinatorial optimization: algorithms and complexity., Courier Dover Publications,1998.

\bibitem {Vespignani2011} R. Pastor-Satorras and A. Vespignani, Epidemic spreading in scale-freenetworks., Physical review letters 86 (2001), no. 14, 3200–3203.

\bibitem {Patterson2002} K.B. Patterson and T. Runge, Smallpox and the native american., The American Journal of the Medical Sciences 323 (2002), no. 4, 216.

\bibitem {Pattillo2012} J. Pattillo, N. Youssef, and S. Butenko, Clique relaxation models in social network analysis., Handbook of Optimization in Complex Networks(2012), 143–162.

\bibitem {Peleg1997} D. Peleg, Local majority voting, small coalitions and controlling monopolies in graphs: A review, Proceedings of the third Colloquiumon Structural Information and Communication Complexity, 1997,pp. 152–169.

\bibitem {Prim1957} R.C. Prim, Shortest connection networks and some generalizations., BellSystem Technical Journal 36 (1957), no. 6, 1389–1401.

\bibitem {Quinlan1986} J.R. Quinlan, Induction of decision trees., Machine Learning 1 (1986),no. 1, 81–106.

\bibitem {1993} J.R. Quinlan, C4. 5: programs for machine learning., Morgan Kaufmann,1993.

\bibitem {Rand1971} W.M. Rand, Objective criteria for the evaluation of clustering methods.,Journal of the American Statistical Association (1971), 846–850.

\bibitem {Resnick1997} Paul Resnick and Hal R. Varian, Recommender systems., Communications of the ACM 40 (1997), no. 3, 56–58.

\bibitem {Robertson1967} T.S. Robertson, The process of innovation and the diffusion of innovation.,The Journal of Marketing (1967), 14–19.

\bibitem {Rogers1995} E.M. Rogers, Diffusion of innovations., Free Press, 1995.

\bibitem {Rohlfs2003} J.H. Rohlfs and H.R. Varian, Bandwagon effects in high-technology industries., The MIT Press, 2003.

\bibitem {Rousseeuw1987} P.J. Rousseeuw, Silhouettes: a graphical aid to the interpretation andvalidation of cluster analysis., Journal of computational and applied mathematics 20 (1987), 53–65.

\bibitem {Ryan1943} B. Ryan and N.C. Gross, The diffusion of hybrid seed corn in two iowacommunities., Rural sociology 8 (1943), no. 1, 15–24.

\bibitem {Salton1975} G. Salton, A. Wong, and C.S. Yang, A vector space model for automaticindexing., Communications of the ACM 18 (1975), no. 11, 613–620.

\bibitem {Michael1986} Gerard Salton and Michael J. McGill, Introduction to modern information retrieval, mcgraw-hill., (1986).

\bibitem {Ester1998} J. Sander, M. Ester, H.P. Kriegel, and X. Xu, Density-based clusteringin spatial databases: the algorithm GDBSCAN and its applications., Data Mining and Knowledge Discovery 2 (1998), no. 2, 169–194.

\bibitem {Sarwar2001} Badrul Sarwar, George Karypis, Joseph Konstan, and John. Riedl,Item-based collaborative filtering recommendation algorithms., Proceedings of the 10th international conference on World Wide Web, ACM,2001, pp. 285–295.

\bibitem {Scellato2011} S. Scellato, M. Musolesi, C. Mascolo, V. Latora, and A. Campbell,Nextplace: a spatio-temporal prediction framework for pervasive systems.,Pervasive Computing (2011), 152–169.

\bibitem {Schafer2007} J Ben Schafer, Dan Frankowski, Jon Herlocker, and Shilad. Sen, Collaborative filtering recommender systems., The Adaptive Web, Springer,2007, pp. 291–324.

\bibitem {Konstan1999} J Ben Schafer, Joseph Konstan, and John Riedl, Recommender systemsin e-commerce., Proceedings of the first ACM conference on Electronic Commerce, ACM, 1999, pp. 158–166.

\bibitem {Scharfstein1990} D.S. Scharfstein and J.C. Stein, Herd behavior and investment., The American Economic Review (1990), 465–479.

\bibitem {Schelling1971} T.C. Schelling, Dynamic models of segregation., Journal of Mathematical Sociology 1 (1971), no. 2, 143–186.

\bibitem {Schelling2006} T.C. Schelling, Micromotives and macrobehavior., WW Norton \& Company,2006.

\bibitem {Scott1988} John. Scott, Social network analysis., Sociology 22 (1988), no. 1, 109–127.

\bibitem {Jesse2009} Shilad Sen, Jesse Vig, and John. Riedl, Tagommenders: connecting usersto items through tags., Proceedings of the 18th international conferenceon World Wide Web, ACM, 2009, pp. 671–680.

\bibitem {Shalizi2011} C.R. Shalizi and A.C. Thomas, Homophily and contagion are generically confounded in observational social network studies., Sociological Methods \& Research 40 (2011), no. 2, 211–239.

\bibitem {Shiller1995} R.J. Shiller, Conversation, information, and herd behavior., The American Economic Review 85 (1995), no. 2, 181–185.

\bibitem {Roelof2008} B¨orkur Sigurbj¨ornsson and Roelof. Van Zwol, Flickr tag recommendation based on collective knowledge., Proceedings of the 17th international conference on World Wide Web, ACM, 2008, pp. 327–336.

\bibitem {Simmel1949} G. Simmel and E.C. Hughes, The sociology of sociability., American Journal of Sociology (1949), 254–261.

\bibitem {Simon1954} H.A. Simon, Bandwagon and underdog effects and the possibility of election predictions., Public Opinion Quarterly 18 (1954), no. 3, 245–253.

\bibitem {Simon1955} H.A. Simon, On a class of skew distribution functions., Biometrika 42 (1955),no. 3/4, 425–440.

\bibitem {Snijders2007} T.A.B. Snijders, C.E.G. Steglich, and M. Schweinberger, Modeling the co-evolution of networks and behavior. in, Longitudinal models in the behavioral and related sciences (2007), 41–71.

\bibitem {Solomonoff1951} R. Solomonoff and A. Rapoport, Connectivity of random nets., Bulletinof Mathematical Biology 13 (1951), no. 2, 107–117.

\bibitem {Spaccapietra2008} S. Spaccapietra, C. Parent, M.L. Damiani, J.A. De Macedo, F. Porto,and C. Vangenot, A conceptual view on trajectories., Data and knowledge engineering 65 (2008), no. 1, 126–146.

\bibitem {Borgatti1993} Martin G. Everett. Stephen P. Borgatti, Two algorithms for computing regular equivalence., Social Networks 15 (1993), no. 4, 361–376.

\bibitem {Stevens1946} S.S. Stevens, On the theory of scales of measurement, science, 1946.

\bibitem {Strang1998} D. Strang and S.A. Soule, Diffusion in organizations and social movements: from hybrid corn to poison pills., Annual Review of Sociology(1998), 265–290.

\bibitem {Strehl2003} A. Strehl, J. Ghosh, and C. Cardie, Cluster ensembles-a knowledge reuse framework for combining multiple partitions., Journal of Machine Learning Research 3 (2003), no. 3, 583–617.

\bibitem {Taghi2009} Xiaoyuan Su and Taghi M. Khoshgoftaar, A survey of collaborative filtering techniques., Advances in Artificial Intelligence 2009 (2009), 4.

\bibitem {Faloutsos2007} J. Sun, C. Faloutsos, S. Papadimitriou, and P.S. Yu, Graphscope:parameter-free mining of large time-evolving graphs, Proceedings of the13th ACM SIGKDD international conference on Knowledge Discovery and Data Mining, ACM, 2007, pp. 687–696.

\bibitem {Steinbach2006} P.N. Tan, M. Steinbach, V. Kumar, and et al., Introduction to datamining., Pearson Addison Wesley Boston, 2006.

\bibitem {Jiliang2013} Jiliang Tang, Huiji Gao, Xia Hu, and Huan. Liu, Exploiting homophily effect for trust prediction., WSDM, 2013.

\bibitem {Huiji2012} Jiliang Tang, Huiji Gao, and Huan. Liu, mtrust: discerning multi-faceted trust in a connected world., WSDM, 2012.

\bibitem {Huan2012} Jiliang Tang, Huiji Gao, Huan Liu, and Atish Das. Sarma, eTrust:understanding trust evolution in an online world., KDD, 2012.

\bibitem {Tang2013} Jiliang Tang, Xia Hu, Huiji Gao, and Huan. Liu, Exploiting local and global social context for recommendation., IJCAI, 2013.

\bibitem {Liu2012} Jiliang Tang and Huan. Liu, Feature selection with linked data in socialmedia., SDM, 2012.

\bibitem {Jiliang2012} Jiliang Tang and Huan. Liu, Unsupervised feature selection for linked social media data., KDD,2012.

\bibitem {Huan2013} Jiliang Tang and Huan. Liu, Coselect: Feature selection with instance selection for social mediadata., SDM, 2013.

\bibitem {Tang2014} L. Tang and H. Liu, Community detection and mining in social media., Synthesis Lectures on Data Mining and Knowledge Discovery2 (2010), no. 1, 1–137.

\bibitem {Huan2009} Lei Tang and Huan. Liu, Relational learning via latent social dimensions.,Proceedings of the 15th ACM SIGKDD international conference on Knowledge Discovery and Data Mining, ACM, 2009, pp. 817–826.

\bibitem {Xufei2012} Lei Tang, Xufei Wang, and Huan. Liu, Community detection via heterogeneous interaction analysis., Data Mining and Knowledge Discovery(DMKD) 25 (2012), no. 1, 1– 33.

\bibitem {Tarde1907} G. Tarde, Las leyes de la imitaci′on: Estudio sociol′ogico., Daniel Jorro,1907.

\bibitem {Thanh2007} N. Thanh and T.M. Phuong, A gaussian mixture model for mobile location prediction., 2007 IEEE International Conference on Research,Innovation and Vision for the Future, IEEE, 2007, pp. 152–157.

\bibitem {Trotter1916} W. Trotter, Instincts of the Herd in War and Peace., 1916.

\bibitem {Ugander4503} Johan Ugander, Brian Karrer, Lars Backstrom, and Cameron. Marlow,The anatomy of the facebook social graph, arxiv preprint arxiv:1111.4503.,Structure 5, 6.

\bibitem {Valente1995} T.W. Valente, Network models of the diffusion of innovations, 1995.

\bibitem {Valente1996} T.W. Valente, Network models of the diffusion of innovations., Computational \& Mathematical Organization Theory 2 (1996), no. 2, 163–164.

\bibitem {Thomas1996} Thomas W. Valente, Social network thresholds in the diffusion of innovations., SocialNetworks 18 (1996), no. 1, 69–89.

\bibitem {Veblen1965} T. Veblen, The Theory of the Leisure Class., Houghton Mifflin Boston,1965.

\bibitem {Huang2010} F. Wang and Q.Y. Huang, The importance of spatial-temporal issues forcase-based reasoning in disaster management., 2010 18th International Conference on Geoinformatics, IEEE, 2010, pp. 1–5.

\bibitem {Kwon2010} S.S. Wang, S.I. Moon, K.H. Kwon, C.A. Evans, and M.A. Stefanone,Face o?: implications of visual cues on initiating friendship on facebook.,Computers in Human Behavior 26 (2010), no. 2, 226–234.

\bibitem {Shamanth2011} Xufei Wang, Shamanth Kumar, and Huan. Liu, A study of tagging behavior across social media., In SIGIR Workshop on Social Web Searchand Mining (SWSM), 2011.

\bibitem {Lei2010} Xufei Wang, Lei Tang, Huiji Gao, and Huan. Liu, Discovering overlapping groups in social media., the 10th IEEE International Conferenceon Data Mining (ICDM2010) (Sydney, Australia), December 14 - 172010.

\bibitem {Warshall1962} S. Warshall, A theorem on boolean matrices., Journal of the ACM (JACM)9 (1962), no. 1, 11–12.

\bibitem {Wasserman1994} S. Wasserman and K. Faust, Social network analysis: Methods and applications., Cambridge University Press (1994).

\bibitem {Watts1999} D.J. Watts, Networks, dynamics, and the small-world phenomenon.,American Journal of Sociology 105 (1999), no. 2, 493–527.

\bibitem {Duncan2002} Duncan J. Watts, A simple model of global cascades on random networks., Proceedings of the National Academy of Sciences 99 (2002), no. 9, 5766–5771.

\bibitem {Dodds2007} D.J. Watts and P.S. Dodds, Influentials, networks, and public opinion formation., Journal of Consumer Research 34 (2007), no. 4, 441–458.

\bibitem {Strogatz1998} D.J. Watts and S.H. Strogatz, Collective dynamics of small-world net-works., nature 393 (1998), no. 6684, 440–442.

\bibitem {Welch1992} I. Welch, Sequential sales, learning, and cascades., Journal of Finance(1992), 695–732.

\bibitem {Lim2010} J. Weng, E.P. Lim, J. Jiang, and Q. He, Twitterrank: finding topic-sensitive influential twitterers, Proceedings of the third ACM international conference on Web Search and Data Mining, ACM, 2010,pp. 261–270.

\bibitem {West2001} D.B. West, Introduction to graph theory., vol. 2, Prentice Hall UpperSaddle River, NJ.:, 2001.

\bibitem {White1980} D.R. White, Structural equivalences in social networks: concepts andmeasurement of role structures, Research Methods in Social NetworkAnalysis Conference, 1980, pp. 193–234.

\bibitem {Reitz1984} White and Reitz, Regge: a regular graph equivalence algorithm for computing role distances prior to blockmodeling., Unpublished manuscript, Universityof California, Irvine (1984).

\bibitem {Frank2011} I.H. Witten, E. Frank, and M.A. Hall, Data Mining: practical machinelearning tools and techniques., Morgan Kaufmann, 2011.

\bibitem {Wunsch2005} R. Xu and D. Wunsch, Survey of clustering algorithms., Neural Net-works, IEEE Transactions on 16 (2005), no. 3, 645–678.

\bibitem {Leskovec2010} J. Yang and J. Leskovec, Modeling information di?usion in implicit net-works, IEEE 10th International Conference on Data Mining, IEEE,2010, pp. 599–608.

\bibitem {Young2001} H.P. Young, Individual strategy and social structure: an evolutionarytheory of institutions., Princeton University Press, 2001.

\bibitem {Yule1925} G.U. Yule, A mathematical theory of evolution, based on the conclusionsof dr. j.c. willis, frs., Philosophical Transactions of the Royal Societyof London. Series B, Containing Papers of a Biological Character 213(1925), 21–87.

\bibitem {Zachary1977} W.W. Zachary, An information flow model for conflict and fission in small groups., Journal of Anthropological Research (1977), 452–473.

\bibitem {Zafarani2010} Reza Zafarani, William D. Cole, and Huan. Liu, Sentiment propagation in social networks: a case study in livejournal., Advances in SocialComputing, Springer, 2010, pp. 413–420.

\bibitem {Zafarani2009} Reza Zafarani and Huan. Liu, Connecting corresponding identitiesacross communities., ICWSM, 2009.

\bibitem {Reza2013} Reza Zafarani and Huan Liu, Connecting users across social media sites: a behavioral-modelingapproach., Proceedings of the 19th ACM SIGKDD international con-ference on Knowledge Discovery and Data Mining, KDD, 2013.

\bibitem {Alan2011} Zheng Alan Zhao and Huan. Liu, Spectral feature selection for datamining., Chapman \& Hall/CRC, 2011.

\end{thebibliography}
