\chapter*{序言}

Python 是一门简单易学、功能强大的编程语言。它具有高效的高级数据结构和简单而有效的面向对象编程的特性。Python 优雅的语法和动态类型、 以及其解释性的性质,使它在许多领域和大多数平台成为脚本编写和快速应用程序开发的理想语言。

从 Python 网站 https://www.python.org/可以免费获得所有主要平台的源代码或二进制形式的Python 解释器和广泛的标准库,并且可以自由地分发。网站还包含许多免费的第三方 Python 模块、 程序、工具以及附加文档的发布包和链接。

Python 解释器可以用 C 或 C+ +  (或调用C的其他语言)中轻松的对新的函数和数据类型进行扩展Python 也适合作为可定制应用程序的一种扩展语言。

本教程通俗地向读者介绍 Python 语言及其体系的基本概念和功能。 随手使用Python 解释器来亲自动手实践是很有帮助的,并且由于所有示例都是自成体系的,所以本教程也可以离线阅读。

有关标准对象和模块的说明,请参阅Python 标准库。 Python语言参考 给出了Python语言的更正式的定义。要编写 C 或 C+ + 的扩展,请阅读扩展和嵌入Python解释器 与Python/C API参考手册。也有几本书深度地介绍了Python 。

本教程不会尝试全面地涵盖每一个单独特性,甚至即使它是常用的特性。相反,它介绍了许多 Python 的值得注意的特性,从而能让你很好的把握这门语言的特性。经过学习,你将能够阅读和编写Python 的模块和程序,并可以更好的学会 Python 标准库 中描述的各种 Python 库模块。


\subsection*{图书网站和相关资源}

本书的网站和附带资源可以通过以下网址获得:

\url{http://dmml.asu.edu/smm}

网站同时提供了配套课件、练习题和示范工程,同时给出了与社会媒体挖掘相关的公共资源入口。

\subsection*{教师须知}

本书按照高年级本科生或研究生一个学期的学习课程进行设计,主要面向拥有计算机科学背景的学生,但对于掌握概率论、统计和线性代数基础知识的读者,也很容易理解本书内容。本书部分章节用于回顾一些基础知识,学生已掌握该部分内容时可忽略这些章节。例如,如果学生已经学过了数据挖掘或机器学习课程,可略过第5章。如果时间受限,第6章至第8章应深入讨论,但第9章和第10章可以简要讨论或者作为阅读材料部分。

~\\
~\\
~\\

\begin{flushright}
\emph{
	Reza Zafarani \\
	Mohammad Ali Abbasi \\
	Huan Liu
}

2013年8月于美国 亚利桑那州 坦佩
\end{flushright}





