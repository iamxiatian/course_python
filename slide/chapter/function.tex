\section{函数}

\begin{frame}[fragile]{CH4 函数}
  \begin{enumerate}
     \item 函数的创建
     \item 函数参数
     %\item 参数魔法
     \item 作用域
     \item 名称与docstring
     %\item 注解
     \item 递归
  \end{enumerate}
\end{frame}

\begin{frame}[fragile]{函数介绍}
  \begin{easylist} \easyitem
    & 函数的类型
    && 全局函数
    && 局部函数
    && lambda函数: 特殊表达式,比普通函数有更多限制
    && 方法
    & 根据功能的实现者不同
    && 内置函数
    && 第三方函数
    && 自己编写的函数
  \end{easylist}
\end{frame}

\subsection{函数的创建}
\begin{frame}[fragile]{函数的创建}
  \begin{python}
    def functionName(parameters):
        suite
  \end{python}

  \begin{easylist}
    & parameters是可选的,如果有多个,则用英文逗号分隔
    & 函数参数可以有默认值
    & 每个函数都有返回值,如果未明确指定返回,则返回None
  \end{easylist}
\end{frame}


\begin{frame}[fragile]{函数创建示例}
  计算前top\_n个元素的平均值
  \begin{python}
    def avg_top(lst, top_n):
        total = 0
        for e in lst[:top_n]:
            total += e
        print(total/top_n)

    avg_top([1,3,9,7,6], 4)
  \end{python}
\end{frame}

\subsection{函数参数}
\begin{frame}[fragile]{函数参数}
  \begin{easylist}
    & 参数可以有默认值
  \end{easylist}

  \begin{python}
    def avg_top(lst, top_n=4):
        total = 0
        for e in lst[:top_n]:
            total += e
        print(total/top_n)

    avg_top([1,3,9,7,6])
  \end{python}
\end{frame}

\begin{frame}[fragile]{形式参数与实际参数}
  \begin{easylist}
    & 写在def语句中函数名称后面的变量叫函数的形式参数
    & 函数调用时提供的值称为实际参数
    & 字符串、数字等普通类型默认为传值调用
    & 列表、元组等默认为传递引用方式
  \end{easylist}
\end{frame}

\begin{frame}[fragile]{传值调用示例}
  \begin{python}
    def change(name):
        name = 'Summer'

    name = 'Spring'
    change(name)
    print(name)
  \end{python}

  此时,输出的name为Spring还是Summer?
\end{frame}


\begin{frame}[fragile]{传递引用调用示例}
  \begin{python}
    def change_list(lst):
        lst.append('Summer')
    
    x = ['Spring']
    change_list(x)
    print(x)
  \end{python}

  此时,输出的x结果是['Spring', 'Summer'],还是['Spring']?
\end{frame}


\begin{frame}[fragile]{位置参数与关键字参数}
  \begin{easylist}
    & 函数默认按照声明位置依次传入参数,即位置参数
    & 当参数顺序难以记忆时,可以使用关键字参数方式,即指定参数名称和参数值的方式
    传递参数
  \end{easylist}

  \begin{python}
    def hello(name, greeting):
        print("%s, %s!" % (name, greeting))
    
    hello("悟空", "欢迎你")
    hello(greeting="欢迎你", name="悟空" )
    hello(name="悟空", greeting="欢迎你")
  \end{python}
  
  后面三条语句的输出结果分别是什么?
\end{frame}


\begin{frame}[fragile]{收集参数}
  \begin{easylist}
    & 有时候允许用户提供任意数量的参数非常有用,如实现任意数量的数字的加法
    & Python允许在一个形式参数名称前面加上星号*,表示收集该位置开始的任意数量的
    参数
    & 任意多个关键字参数的收集:**修饰符 (自学)
  \end{easylist}

  \begin{python}
    def add(*numbers):
        total = 0
        for n in numbers:
            total += n
        return total

    add(1,2,3)
  \end{python}
\end{frame}


\subsection{Lambda函数}
\begin{frame}[fragile]{Lambda函数}
  \begin{easylist}
    & 一种快速定义单行的最小函数
    & 按照如下语法创建的匿名函数
    && lambda parameters: expression
    & 特点
    && parameters是可选的
    && epxression不能包含分支或循环,但可以使用条件表达式
    && 不能包含return, yield语句
    && 结果为一个匿名函数
  \end{easylist}
\end{frame}

\begin{frame}[fragile]{Lambda函数示例}
  \begin{python}
def f(x,y):
    return x*y

g = lambda x,y: x*y
print(f(3,4))
print(g(3,4))
  \end{python}

  \begin{easylist}
    & 上例中函数f和g的效果相同,但g的定义更快捷
    & 区别:def 是语句而lambda是表达式
  \end{easylist}
\end{frame}


%\subsection{参数魔法}


\subsection{作用域}
\begin{frame}[fragile]{作用域}
  \begin{easylist}
    & 全局变量
    && 函数内部可以直接调用之前声明的全局变量,但不建议使用这种方式
    & 局部变量
    && 在函数(类)内部定义的变量
  \end{easylist}

  \begin{python}
    def f(): 
        x = 10
    
    x = 1
    f()
    print(x)
  \end{python}
  结果为:?
\end{frame}

\begin{frame}[fragile]{在函数内部绑定全局变量}
  \begin{easylist}
    & 通过global关键字在函数内部将变量绑定为全局变量
  \end{easylist}

  \begin{python}
    def f():   
        global x
        x = 10
    
    x = 1
    f()
    print(x)
  \end{python}

  结果为:10
\end{frame}



\subsection{名称与docstring}

\begin{frame}[fragile]{参数的命名建议}
  \begin{easylist}
    & 对函数及其参数合理命名有助于代码理解
    && 使用命名框架,保持一致性
    && 所有名称避免使用缩略(除非是标准化且广泛使用的)
    && 合理地使用变量与参数名
    &&& x: 适合作为坐标参数
    &&& i: 适合用于简单的循环计数
    && 函数名与方法名应可以表现其行为或返回值
    &&& E.g. 查找在列表中的位置
    \begin{python}
      def find(l, s, i=0)
      def linear_search(l, s, i = 0)
      def first_index_of(sorted_name_list, name, start=0) #GOOD!
    \end{python}
  \end{easylist}
\end{frame}

\begin{frame}[fragile]{docstring}
  \begin{easylist}
    & 对于上面的函数first\_index\_of,如果没有找到对应的name该怎么处理,是返回$-1$,
    还是产生异常?这类信息可以通过docstring提供给函数使用者
    & docstring第一行是对函数的一个简短的描述,紧接着一个空白行,然后是完整描述,
    如果是交互式输入再执行的程序,还会给出一些示例
    & 例子
  \end{easylist}

\end{frame}

\begin{frame}[fragile]{docstring示例}
  \tiny
  \lstinputlisting[keywordstyle=\ttfamily,
  basicstyle=\rmfamily\small]{src/finder.py} 
\end{frame}

\subsection{注解}

\subsection{递归}
\begin{frame}[fragile]{递归}
  \begin{easylist}
    & 函数不仅可以调用其他函数,还可以调用自身
    & 递归的典型特点是一个函数不断调用自身,当到达指定条件时,再逐级返回
    & 普通的幂函数实现
  \end{easylist}

  \begin{python}
def power(x, n):
    result = 1
    for i in range(n):
        result = result*x
    return result

power(2,3) #应输出8
  \end{python}
\end{frame}

\begin{frame}[fragile]{幂函数的递归实现}
  \begin{easylist}
    & 对于任意数字x,power(x,0) = 1
    & 对于任意大于0的数n来说,power(x,n)是x乘以power(x, n-1)的结果
  \end{easylist}

  \begin{python}
def power(x, n):
    if(n==0): 
        return 1
    else:
        return power(x, n-1)*x
    
power(2,3)#应输出8
  \end{python}
\end{frame}


\begin{frame}[fragile]{课堂练习}
  \begin{easylist}
    & 利用递归算法解决汉诺塔(Hanoi Tower)问题
  \end{easylist}

  \begin{center}
    \includegraphics[width=0.6\textwidth]{figure/hanoi.jpg}
  \end{center}

\end{frame}


\begin{frame}[fragile]{Hanoi Tower}
  \begin{python}
def hanoi(a, b, c, n):
    if n==1:
        print("move {0} from {1} to {2}".format(n, a, c))
    else:
        hanoi(a, c, b, n-1)
        print("move {0} from {1} to {2}".format(n, a, c))
        hanoi(b, a, c, n-1)
        
hanoi('a', 'b', 'c', 3)    
  \end{python}
\end{frame}


\begin{frame}[fragile]{求组合$n \choose m$}
  c(n,m)=c(n-1,m-1)+c(n-1,m)
等式左边表示从n个元素中选取m个元素,而等式右边表示这一个过程的另一种实现方法:任意选择n中的某个备选元素为特殊元素,从n中选m个元素可以由此特殊元素的分成两类情况,即m个被选择元素包含了特殊元素和m个被选择元素不包含该特殊元素。
\end{frame}


\subsection{函数式编程}
\begin{frame}[fragile]{函数式编程}
  \begin{easylist}
    & 面向过程的程序设计
    && 把复杂任务分解成简单的任务,简单任务通常以函数表示
    & 函数式编程(Functional Programming)
    && 也可以归结到面向过程的程序设计,但其思想更接近数学计算。
    && 就是越低级的语言,越贴近计算机,抽象程度低,执行效率高,比如C语言;越高级
    的语言,越贴近计算,抽象程度高,执行效率低,比如Lisp语言。
    && 函数式编程就是一种抽象程度很高的编程范式,纯粹的函数式编程语言编写的函数
    没有变量。任意一个函数,只要输入是确定的,输出就是确定的,这种没有副作用的函
    数称为纯函数。
    && 允许把函数本身作为参数传入另一个函数,还允许返回一个函数!
  \end{easylist}
\end{frame}

\begin{frame}[fragile]{高阶函数:Higher-order function}
  \begin{easylist}
    & 变量可以指向函数
    & 函数名也是变量
    & 传入函数
    & 
  \end{easylist}

\end{frame}


\begin{frame}[plain]
  \begin{center}
    \Simley{1}{10}{1}

    \Huge ---END---
  \end{center}
\end{frame}

%%% Local Variables:
%%% mode: latex
%%% TeX-master: "../python"
%%% End:
