\section{Python数据类型}

\begin{frame}[fragile]{CH2 数据类型}
  \begin{easylist} \easyitem
    & 2.1 标识符与关键字
    & 2.2 基本数据类型
    && 2.2.1 整数
    && 2.2.2 布尔
    && 2.2.3 浮点
    && 2.2.4 字符
    & 2.3 组合数据类型
    && 2.3.1 序列类型:元组与列表
    && 2.3.2 集合
    && 2.3.3 字典(映射)
  \end{easylist}
\end{frame}

\subsection{2.1 标识符与关键字}

\begin{frame}[fragile]{2.1 标识符与关键字}
  \begin{easylist}
    & 标识符:任意长度的非空字符序列
    && 要求
    &&& 引导字符:Unicode编码字母、ASCII字符、\_,不能是数字!
    &&& 不能是关键字
  \end{easylist}

  \begin{table}
    \begin{center}
      \begin{tabular}{|l | l | l |  l| l|}
        \toprule \hline
        and & as & assert & break & class \\ \hline
        continue & def & del & eif & else \\ \hline
        except & False & finally & for & from \\ \hline
        global & if & import & in & is \\ \hline
        lambda & None & nonlocal & not & or \\ \hline
        pass & raise & return & True & try \\ \hline
        while & with & yield & & \\ \hline
        \bottomrule
      \end{tabular}
    \end{center}
    \caption{Python关键字列表}
  \end{table}
\end{frame}

\begin{frame}[fragile]{标识符命名约定}
  \begin{easylist}
    & 尽量不要使用Python内置函数名与异常名, 如int, float...
    & 避免使用名称的开始和结尾都是下划线
    && 在Python解释器中,输入如下语句,观察输出:

    $>>>$ max.\_\_doc\_\_

    注意:前后各是两个下划线"\_"
  \end{easylist}
\end{frame}

\begin{frame}[fragile]{测试}
  $>>>$ hello-world = 1 \\ Error!

  $>>>$ 5miles = 5 \\ Error!

  $>>>$ int = 5 \\ 正常运行,但不建议

  $>>>$ hello\_world = 'Hello world!' \\ OK
\end{frame}


\subsection{2.2 基本数据类型}
\begin{frame}[fragile]{2.2 基本数据类型}
  \begin{easylist}
    &  基本数据类型
    && 整数
    && 布尔
    && 浮点
    && 字符
  \end{easylist}
\end{frame}


\subsubsection{2.2.1 整数}
\begin{frame}[fragile]{2.2.1 整数}
  \begin{easylist}
    & 默认为10进制形式 \\
    $>>>$ 12345

    & 二进制形式:0bxxxx \\
    $>>>$ 0b1111 (15)

    & 八进制形式:0oxxxx \\
    $>>>$ 0o10 (8)

    & 十六进制形式:0xABCD \\
    $>>>$ 0xFF (255)

    & 前导字符大小写均可以,如0XFF
  \end{easylist}
\end{frame}


\begin{frame}[fragile]{数值型操作符与函数}
  \begin{center}
    \begin{tabular}{l | l}
      \toprule
      语法 & 描述 \\ 
      \midrule
      $x // y$ & 整除 \\ \hline
      $x \% y $ & 取模 \\ \hline
      $x ** y$ & x的y次幂 \\ \hline
      abs(x) & 取绝对值 \\ \hline
      pow(x, y) & x的y次幂 \\ \hline
      round(x, n) & 四舍五入,n指小数位保留几位 \\
      \bottomrule
    \end{tabular}
  \end{center}
\end{frame}

\begin{frame}[fragile]{整数转换函数}
  \begin{center}
    \begin{tabular}{l | l}
      \toprule
      语法 & 描述 \\
      \midrule
      bin(i) & 返回整数i的二进制形式 \\ \hline
      hex(i) & 返回整数i的十六进制形式 \\ \hline
      int(x) & 将x转换为整数 \\ \hline
      oct(i) & 返回i的八进制形式 \\ 
      \bottomrule
    \end{tabular}
  \end{center}
\end{frame}

\begin{frame}[fragile]{整数位移操作符}
  \begin{center}
    \begin{tabular}{l | l}
      \toprule
      语法 & 描述 \\ 
      \midrule
      $i | j$  & 逻辑OR运算 \\ \hline
      $i \mathbin{\char`\^} j$ & XOR \\ \hline
      $i \& j$ & AND \\ \hline
      $i<<j$ & i左移j位,类似于$i*(2**j)$,但不带溢出检查 \\ \hline 
      $i>>j$ & i右移j位,类似于$i//(2**j)$,但不带溢出检查 \\ \hline
      $\textasciitilde i$ & 反转i的每一位 \\
      \bottomrule
    \end{tabular}
  \end{center}
\end{frame}

\begin{frame}[fragile]{位逻辑操作练习}
  $>>>$ i = 0b1010 \\
  $>>>$ j = 0b11000 \\

  $>>>$ $print(i,j,i|j, i\&j, i>>2, i<<2, \textasciitilde i)$

  手工计算一下结果是多少?并利用Python解释器进行验证,注意,可以利用bin()函数,
  查看结果的二进制形式。

  \pause

  输出结果:10 24 26 8 2 40 -11

  \pause

  \begin{easylist}
    & \small 计算机内部通常采用补码表示整数,补码对于正数就是本身,负数补码等于源
    码的反码加一,正数取反后在数值上体现为$-|x+1|$,负数的是$|x|-1$。

    & \small 如果要表示多个布尔值,可以利用一个整数进行表示,如文件的属性有``读、写和执
    行''三种,可以用一个三位的二进制数字表示。
  \end{easylist}
\end{frame}


\begin{frame}[fragile, allowframebreaks]{原码 $\rightarrow$ 反码 $\rightarrow$ 补码}
  \h 无符号数字:只有正数,没有负数的概念。

  \begin{center}
    \begin{tabular}{| c | c |}
      \hline
      ~~~十进制~~~ & ~~~~~~~二进制~~~~~~~ \\ \hline
      0 & 0000 \\ \hline
      1 & 0001 \\ \hline
      2 & 0010 \\ \hline
      3 & 0011 \\ \hline
      4 & 0100 \\ \hline
      $\vdots$ & $\vdots$ \\ \hline
    \end{tabular}
  \end{center}

    \newpage
    ~\\
    \begin{easylist}
      & 神说,要有光,就有了光,要有正数和负数,就有了正数和负数
      && 为表示正数和负数,人们发明了``原码'':
      &&& 左边第一位存放符号,正用0来表示,负用1来表示
    \end{easylist}

    \begin{center}
      \begin{columns}[totalwidth=0.5\textwidth,t]
        \begin{column}{.3\textwidth}
          \centering
          \begin{tabular}{| c | c |}
            \hline
            ~ & 正数 \\ \hline
            0 & 0000 \\ \hline
            1 & 0001 \\ \hline
            2 & 0010 \\ \hline
            3 & 0011 \\ \hline
            4 & 0100 \\ \hline
            5 & 0101 \\ \hline
            6 & 0110 \\ \hline
            7 & 0111 \\ \hline
          \end{tabular}
        \end{column}
        
        \begin{column}{.3\textwidth}
          \centering
          \begin{tabular}{| c | c |}
            \hline
            ~ & 负数 \\ \hline
            -0 & 1000 \\ \hline
            -1 & 1001 \\ \hline
            -2 & 1010 \\ \hline
            -3 & 1011 \\ \hline
            -4 & 1100 \\ \hline
            -5 & 1101 \\ \hline
            -6 & 1110 \\ \hline
            -7 & 1111 \\ \hline
          \end{tabular}
        \end{column}
      \end{columns}
    \end{center}

    \newpage
    ~ \\
    \begin{easylist}
      & 易于理解,但不利于计算机处理
      && $1+(-1) = 0001_b + 1001_b = 1010_b = -2=$
      && 存在正0($0000_b$)和负0($1000_b$)
      && 正负相加不等于0
    \end{easylist}

    \newpage
    ~\\
    \begin{easylist}
      & 反码
      && 用来处理负数的,符号位置不变,其余位置相反
    \end{easylist}

    \begin{center}
      \begin{columns}[totalwidth=0.5\textwidth,t]
        \begin{column}{.3\textwidth}
         \centering
          正数:\\
          \begin{tabular}{| c | c |}
            \hline
            ~ & 正数 \\ \hline
            0 & 0000 \\ \hline
            1 & 0001 \\ \hline
            2 & 0010 \\ \hline
            3 & 0011 \\ \hline
            4 & 0100 \\ \hline
            5 & 0101 \\ \hline
            6 & 0110 \\ \hline
            7 & 0111 \\ \hline
          \end{tabular}
        \end{column}
        
        \begin{column}{.3\textwidth}
          \centering
          \color{red} 反码:\\
          \begin{tabular}{| c | c |}
            \hline
            ~ & 负数 \\ \hline
            -0 & 1111 \\ \hline
            -1 & 1110 \\ \hline
            -2 & 1101 \\ \hline
            -3 & 1100 \\ \hline
            -4 & 1011 \\ \hline
            -5 & 1010 \\ \hline
            -6 & 1001 \\ \hline
            -7 & 1000 \\ \hline
          \end{tabular}
        \end{column}

        \begin{column}{.3\textwidth}
          \centering
          原码:\\
          \begin{tabular}{| c | c |}
            \hline
            ~ & 负数 \\ \hline
            -0 & 1000 \\ \hline
            -1 & 1001 \\ \hline
            -2 & 1010 \\ \hline
            -3 & 1011 \\ \hline
            -4 & 1100 \\ \hline
            -5 & 1101 \\ \hline
            -6 & 1110 \\ \hline
            -7 & 1111 \\ \hline
          \end{tabular}
        \end{column}
      \end{columns}
    \end{center}
    
    \newpage
    ~\\
    \begin{easylist}
      & 原码变成反码后,解决了正负相加等于$0$的问题
      & 过去的$+1$和$-1$相加,变成了$0001_b+1101_b=1111_b$,刚好反码表示方式中,
      $1111_b$象征$-0$
      & 新问题:存在两个零,$+0$和$-0$
      & 从原来"反码"的基础上,补充一个新的代码,负数加1,因此,$-0$由$1111_b$变
      为$10000_b$,舍去溢出的高位,变为$0000_b$
    \end{easylist}
    
    \newpage
    ~\\
    \begin{center}
      \begin{columns}[t]
        \begin{column}{.25\textwidth}
         \centering
          正数:\\
          \begin{tabular}{| c | c |}
            \hline
            ~ & 正数 \\ \hline
            0 & 0000 \\ \hline
            1 & 0001 \\ \hline
            2 & 0010 \\ \hline
            3 & 0011 \\ \hline
            4 & 0100 \\ \hline
            5 & 0101 \\ \hline
            6 & 0110 \\ \hline
            7 & 0111 \\ \hline
          \end{tabular}
        \end{column}
        
        \begin{column}{.25\textwidth}
          \centering
          \color{blue} 补码:\\
          \begin{tabular}{| c | c |}
            \hline
            ~ & 负数 \\ \hline
            -0 & 0000 \\ \hline
            -1 & 1111 \\ \hline
            -2 & 1110 \\ \hline
            -3 & 1101 \\ \hline
            -4 & 1100 \\ \hline
            -5 & 1011 \\ \hline
            -6 & 1010 \\ \hline
            -7 & 1001 \\ \hline
            -8 & 1000 \\ \hline
          \end{tabular}
        \end{column}

        \begin{column}{.25\textwidth}
          \centering
          \color{red} 反码:\\
          \begin{tabular}{| c | c |}
            \hline
            ~ & 负数 \\ \hline
            -0 & 1111 \\ \hline
            -1 & 1110 \\ \hline
            -2 & 1101 \\ \hline
            -3 & 1100 \\ \hline
            -4 & 1011 \\ \hline
            -5 & 1010 \\ \hline
            -6 & 1001 \\ \hline
            -7 & 1000 \\ \hline
          \end{tabular}
        \end{column}

        \begin{column}{.25\textwidth}
          \centering
          原码:\\
          \begin{tabular}{| c | c |}
            \hline
            ~ & 负数 \\ \hline
            -0 & 1000 \\ \hline
            -1 & 1001 \\ \hline
            -2 & 1010 \\ \hline
            -3 & 1011 \\ \hline
            -4 & 1100 \\ \hline
            -5 & 1101 \\ \hline
            -6 & 1110 \\ \hline
            -7 & 1111 \\ \hline
          \end{tabular}
        \end{column}
      \end{columns}
    \end{center}
    
    \begin{easylist}
      & 解决了$+0$和$-0$同时存在的问题
      & 另外"正负数相加等于0"的问题
      & 多了一个$-8$
    \end{easylist}

\end{frame}




\subsubsection{2.2.2 布尔类型}
\begin{frame}[fragile]{2.2.2 布尔类型}
  ~ \\
  \begin{columns}
    \begin{column}{.3\textwidth}
      基本用法:

      \pyinline{t=True} 

      \pyinline{f = False} 

      \pyinline{t and f} \\
      False 

      \pyinline{t and True} \\
      True 
    \end{column}
    \pause
    \begin{column}{.6\textwidth}
      布尔类型可以看作是一种特殊的整数类型,False为0,True为1,
      0为False, 非0值为True,例如:

      \vspace{0.5cm}
      \pyinline{False + 1} \\ $1$

      \pyinline{True - 2} \\ $-1$
    \end{column}
  \end{columns}
\end{frame}


\subsubsection{2.2.3 浮点类型}
\begin{frame}[fragile]{2.2.3 浮点类型}
  \textbf{float}:  计算机以二进制表示浮点数,是近似表示
  
  \pyinline{0.0, 5.4, 1e-2} \\
  (0.0, 5.4, 0.01)
  

  \pyinline{x = 5.59} \\
  \pyinline{int(x)} \\5 \\    
  \pyinline{s=x.hex()}\\'0x1.65c28f5c28f5cp+2', 此处指数以p表示,而不是e,why? \\
  \pyinline{y=float.fromhex(s)}
\end{frame}

\begin{frame}[fragile]{其他数字类型及操作}
  \begin{easylist}
    & 复数 \\
    \pyinline{z = 1 + 2j}
    & 扩展包
    && math \\
    ~\pyinline{import math} \\
    ~\pyinline{math.pi} \\
    ~\pyinline{math.______}
    && decimal:精度可控 

    \begin{python}
      import decimal
      a = decimal.Decimal(1234)
      b = decimal.Decimal("1234567890000.1234567")
      a + b
    \end{python}
    Result: Decimal('1234567891234.1234567')
  \end{easylist}
\end{frame}

\subsection{2.2.4 字符串}
\begin{frame}[fragile]{2.2.4 字符串}
  \begin{easylist}
    & 可以使用单引号或双引号,但两端必须相同
    & 三个引号包含的字符串
    && 可以直接在里面使用换行,而不需要进行转义处理
    & 转义符
    && 用于处理特殊字符,如回车、换行、引号等
    &&& $\backslash r$, $\backslash t$, $\backslash n$, $\backslash '$,
    $\backslash "$, $\cdots$
    &&& $\backslash xhh$: 对应十六进制数字的字符
    && 例子 \\
    ~\pyinline{s = 'hello\\x20world\\n\\tI\\'m fine.'} \\
    ~\pyinline{print(s)} \\
    \pause
    输出结果:\\
    hello world\\
    ~~~~I'm fine.
  \end{easylist}
\end{frame}

\begin{frame}[fragile]{常见函数}
  \begin{easylist}
    & Python通过ord和chr两个内置函数,用于字符与ASCII码之间的转换
    && ord()函数 \\
    求特定字符的Unicode编码,以十进制表示返回结果 
    && chr()函数\\
    返回整数所对应的字符 
  \end{easylist}

  \pyinline{ord('A')}  \\65 \\
  \pyinline{hex(ord('A'))} \\'0x41' \\
  \pyinline{hex(ord('中'))} \\'0x4e2d' \\
  \pyinline{chr(0x4e2d)} \\ '中'
\end{frame}

\begin{frame}[fragile]{字符串比较}
  内部安装字符串的ASCII码值进行比较

  \pyinline{"RUC" > "IRM"} \\ True

  \pyinline{"China" > "Canda"} \\True
\end{frame}

\begin{frame}[fragile]{字符串分片与步距}
  \begin{center}
    \begin{tikzpicture}[c/.style={minimum width=1.2cm, minimum height=1cm, draw}]
      \draw node[c] (1) {W} node[c, right=0 of 1] (2) {h} node[c, right=0 of 2] (3) {o} node[c, right=0 of 3] (4) { } node[c, right=0 of 4] (5) {a}  node[c, right=0 of 5] (6) {m}  node[c, right=0 of 6] (7) { }  node[c, right=0 of 7] (8) {I}  node[c, right=0 of 8] (9) {?};
      \draw node[above=0 of 1] {s[-9]} node[above=0 of 2] {s[-8]} node[above=0 of 3] {s[-7]} node[above=0 of 4] {s[-6]} node[above=0 of 5] {s[-5]} node[above=0 of 6] {s[-4]} node[above=0 of 7] {s[-3]} node[above=0 of 8] {s[-2]} node[above=0 of 9] {s[-1]};
      \draw node[below=0 of 1] {s[0]} node[below=0 of 2] {s[1]} node[below=0 of 3] {s[2]} node[below=0 of 4] {s[3]} node[below=0 of 5] {s[4]} node[below=0 of 6] {s[5]} node[below=0 of 7] {s[6]} node[below=0 of 8] {s[7]} node[below=0 of 9] {s[8]};
    \end{tikzpicture}
  \end{center}

  \begin{easylist}
    & 语法
    && s[start]
    && s[start:end]
    && s[start:end:step]
  \end{easylist}

  \pyinline{s = 'Who am I?'} \\
  \pyinline{s[0:6:2], s[0:3]} \\
  'Woa', 'Who'
\end{frame}


\begin{frame}[fragile]{字符串步距}
  步距默认为1,但可以为负

  \pyinline{s[-1:]} \\ '?'

  \pyinline{s[-1::-1]} \\'?I ma ohW'
\end{frame}


\begin{frame}[fragile,allowframebreaks]{字符串操作方法}
  \begin{center}
    \scriptsize
    \begin{longtable}{p{0.3\textwidth} | p{0.6\textwidth}}
      \toprule
      语法 & 描述 \\ 
      \midrule
      s.capitalize() & 返回s的副本,并将首字母变为大小 \\ \hline
      s.center(width,char) & 返回s中间部分的一个子字符串,长度为width,并使用空格或可选的char(长度为1的字符串)进行填充。参考str.ljust()、str.rjust()与str.format() \\ \hline
      s.count(t,start,end) & 返回字符串s中(或在s的start:end分片中)子字符串t出现的次数 \\ \hline
      s.encode(encoding,err) & 返回一个bytes对象,该对象使用默认的编码格式或制定的编码格式来表示该字符串 \\ \hline
      s.endswith(x,start,end) & 如果s(或在s的start:end分片中)一字符串x(或元组中的任意字符串)结尾,就返回true,否则返回False,参考str.find() \\ \hline
      s.expandtabs(size) & 返回s的一个副本,其中的制表符使用8个货指定数量的空格替换 \\ \hline
      s.find(t,star,end) & 返回t在s中(或在s的start:end分片中)的最左位置,如果没有找到,就返回-1;使用str.rfind()则可以发现相应的最右边位置:参考str.index() \\ \hline
      s.formal(...) & 返回按给定参数进行格式化后的字符串副本,这一方法及其参数将在下一小节进行讲解 \\ \hline
      s.index(t,start,end) & 返回t在s中的最左边位置(或在s的start:end分片中),如果没有找到,就产生ValuError异常。使用sr.rindex()可以从右面开始搜索。参见str.find() \\ \hline
      s.isalnum() & 如果s非空,并且其中的每个字符串都是字母数字的,就返回True \\ \hline
      s.isalpha() & 如果s非空,并且其中每个字符都是字母的,就返回Ture \\ \hline
      s.isdecimal() & 如果s非空,并且其中的每个字符都是Uicode的基数为10的数字,就返回True, \\ \hline
      s.isdigit() & 如果s非空,并且每个字符都是一个ASCII数字,就返回Ture \\ \hline
      s.isidentifier() & 如果s非空,并且是一个邮箱的标识符,就返回Ture \\ \hline
      s.islower() & 如果s至少有一个可小写的字符,并且所有可小写的字符都是小写的,就返回True,参见str.isupper() \\ \hline
      s.isnumeric() & 如果s非空,并且其中的每个字符都是数值型的Uicode字符,比如数字或小数,就返回True \\ \hline
      s.isprintable() & 如果s非空,并且其中的每一个字符被认为死可打印的,包括空格,但不包括换行,就返回True \\ \hline
      s.isspace() & 如果s非空,并且其中的每一个字符都是空白字符,就返回Ture \\ \hline
      s.istitle() & 如果s是一个非空的首字母大写的字符串,就返回True,参见str.title() \\ \hline
      s.isupper() & 如果s至少有一个可大写的字符,并且所有可大写的字符都是大写的,就返回True,参见str.islower() \\ \hline
      s.join(seq) & 返回序列seq中每个项连接起来的后果,并以s(可以为空)在每两项之间分隔 \\ \hline
      s.ljust(width,char) & 返回长度为width的字符串(使用空格或可选的char(长度为1的字符串)进行填充)中左对齐的字符串s的一个副本,使用str.rjust()可以右对齐,str.center()可以中间对齐,参考str.format() \\ \hline
      s.lower() & 将s中的字符变为小写,参见str.upper() \\ \hline
      s.maketrans() & 与str.translate()类似,参见正文谅解详细资料 \\ \hline
      s.partition(t) & 返回包含3个字符串的元组————字符串s在t的最左边之前的部分、t、字符串s在t之后的部分。如果t不在s内,则返回s与两个空字符串。使用str.rpartition()可以在t最右边部分进行分区 \\ \hline
      s.replace(t,u,n) & 返回s的一个副本,其中每个(或最多n个,如果给定)字符串t使用u替换 \\ \hline
      s.split(t,n) & 返回一个字符串列表。要求在字符串t处至多分割n次,如果没有给定n,就分割尽可能多次,如果t没有给定,就在空白处分割。使用str.rsplit()可以从右边进行分割————只有在给定n并且n小于可能分割的最大次数时才能起作用 \\ \hline
      s.splitlines(f) & 返回在行终结符处进行分割产生的列表,并剥离行终结符(除非f为True) \\ \hline
      s.starswith(x,start,end) & 如果s(或s的start:end分片)以字符串x开始(或以元组x中的任意字符串开始),就返回True,否则返回False,参考str.endswith() \\ \hline
      s.strip(chars) & 返回s的一个副本,并将开始处与结尾处的空白字符(或字符串chars中的字符)移除,str.lstrip()仅剥离起始处的相应字符,str.rstrip()只剥离结尾处的相应字符 \\ \hline
      s.swapcase() & 返回s的辅副本,并将其中大写字符变为小写,小写字符变大写,参考str.lower()与str.upper() \\ \hline
      s.title() & 返回s的副本,并将每个单词的首字母变为大写,其他字母都变为小写,参考str.istitle() \\ \hline
      s.translate() & 与str.maketrans()类似,参考正文了解详细资料 \\ \hline
      s.upper() & 返回s的大写化版本,参考str.lower() \\ \hline
      s.zfill(w) & 返回s的副本,如果比w短,就在开始处添加0,使其长度为w \\ \hline
    \end{longtable}
  \end{center}
\end{frame}

\begin{frame}[fragile]{join函数}
  \pyinline{countries=["China", "USA", "Japan"]} 

  \pyinline{" ".join(countries)}  \\
  'China USA Japan' 

  \pyinline{" <-> ".join(countries)} \\
  'China <-> USA <-> Japan'
\end{frame}

\begin{frame}[fragile]{字符串复制 \emph{*}}
  \pyinline{s = 'ab' * 3} \\
  'ababab'

  \pyinline{s *= 2} \\
  'abababababab'
\end{frame}

\begin{frame}[fragile]{format()函数}
  \pyinline{'{0} is {1} years old.'.format("Tom", 5)} \\
  'Tom is 5 years old.'

\end{frame}

\subsection{2.3 组合数据类型}
\begin{frame}[fragile]{2.3 组合数据类型}
  \begin{easylist}
    & 列表
    & 元组
    & 字典
    & 集合
  \end{easylist}
\end{frame}



\begin{frame}[fragile]{列表}
  \h list是一种有序的集合,可以随时添加和删除其中的元素

  \pyinline{classmates = ['Michael', 'Bob', 'Tracy']} 

  \pyinline{classmates} \\
  $['Michael', 'Bob', 'Tracy']$

\end{frame}

\begin{frame}[fragile]{列表——索引}
  \h 用索引来访问list中每一个位置的元素,索引是从$0$开始的,最后一个元素的索引是$-1$

  \pyinline{classmates[0]} \\  
  'Michael'

  \pyinline{classmates[1]} \\  
  'Bob'  

  \pyinline{classmates[-1]} \\
  'Tracy'

  \pyinline{classmates[-2]} \\
  'Bob'
\end{frame}


\begin{frame}[fragile]{列表——追加}
    \h list是一个可变的有序表,可以往list中追加元素到末尾

    \pyinline{classmates.append('Adam')} 

    \pyinline{classmates} \\  
    $['Michael', 'Bob', 'Tracy', 'Adam']$
    

    \h 可以把元素插入到指定的位置,比如索引号为1的位置

    \pyinline{classmates.insert(1, 'Jack')} 

    \pyinline{classmates} \\ 
    $['Michael', 'Jack', 'Bob', 'Tracy', 'Adam']$
\end{frame}

\begin{frame}[fragile]{列表——删除}
  \h 用pop()方法删除list末尾的元素

  \pyinline{classmates.pop()} \\ 
  'Adam'

  \pyinline{classmates} \\ 
  $['Michael', 'Jack', 'Bob', 'Tracy']$


  \h 用pop(i)方法删除指定位置的元素

  \pyinline{classmates.pop(1)} \\ 
  'Jack'

  \pyinline{classmates} \\ 
  $['Michael', 'Bob', 'Tracy']$
\end{frame}

\begin{frame}[fragile]{列表——替换}
  \h 赋值给对应的索引位置把某个元素替换成别的元素

  \pyinline{classmates[1] = 'Sarah'} 

  \pyinline{classmates} \\ 
  $['Michael', 'Sarah', 'Tracy']$
\end{frame}


\begin{frame}[fragile]{列表——数据类型}
  \h list里面的元素的数据类型可以不同

  \pyinline{L = ['Apple', 123, True]} 


  \h list元素也可以是另一个list

  \pyinline{p = ['asp', 'php']} 

  \pyinline{s = ['python', 'java', p, 'scheme']} 

  \pyinline{s[2][1]} \\ 
  'php'
\end{frame}

\begin{frame}[fragile]{元组}
  \h 元组tuple是一种有序的集合,一旦初始化就不能修改,使代码更安全

  \pyinline{classmates = ('Michael', 'Bob', 'Tracy')} 
  
\end{frame}

\begin{frame}[fragile]{元组——属性}
  \h tuple不能改变,不可以追加、删除、替换

  \pyinline{t = (1, 2)} 

  \pyinline{t} \\  
  (1, 2)

  
  tuple的“可变性”

  \pyinline{t = ('a', 'b', ['A', 'B'])} 

  \pyinline{t[2][0] = 'X'} 

  \pyinline{t[2][1] = 'Y'} 

  \pyinline{t} \\
  ('a', 'b', ['X', 'Y'])
\end{frame}


\begin{frame}[fragile]{字典}
  \h 字典是一种可变的无序的数据组合类型,使用键-值(key-value)储存,进行极快的
  查找 

  \h 键必须是唯一的,必须是可哈希的

  \pyinline{d = {'Michael': 95, 'Bob': 75, 'Tracy': 85}} 

  \pyinline{d['Michael']} \\  
  95
  
\end{frame}

\begin{frame}[fragile]{字典——原理}
  \begin{easylist}
    & 给定一个名字,比如'Michael',dict在内部就可以直接计算出Michael对应的存放成绩的
    “页码” 
    & 也就是95这个数字存放的内存地址,直接取出来,所以速度非常快
    & 在放进去的时候,必须根据key算出value的存放位置,取的时候才能根据key直接拿到
    value
  \end{easylist}
\end{frame}

\begin{frame}[fragile]{字典——对应性}
  \h 一个key只能对应一个value,多次对一个key放入value,后面的值会把前面的值冲掉

  \pyinline{>>> d['Jack'] = 90}

  \pyinline{d['Jack']} \\ 
  90

  \pyinline{>>> d['Jack'] = 88} 

  \pyinline{d['Jack']} \\ 
  88

\end{frame}

\begin{frame}[fragile]{字典——更新和删除}
  \h 字典是可变的,所以可以用update进行更新 

  \pyinline{d = {'Michael': 95, 'Bob': 75, 'Tracy': 85}} 

  \pyinline{d1 = {'Jack':80}} 

  \pyinline{d.updata(d1)} 

  \pyinline{d} 
  $\{'Jack':80 ,'Michael': 95, 'Bob': 75, 'Tracy': 85\}$


  \h 如果要删除一个key,用pop(key)方法
  
  \pyinline{d.pop('Bob')} \\
  75

  \pyinline{d} \\
  $\{'Jack':80 ,'Michael': 95, 'Tracy': 85\}$
\end{frame}

\begin{frame}[fragile]{集合}
    \h 集合是一个无序的、不重复的元素集,包含两种类型:可变集合(set)和不可变集合
    (frozen set)
    
    ~ (1) 可变集合(set):一组key的集合,不储存value,可添加和删除元素,非可哈希的,
    不能用作字典的键,也不能做其他集合的元素

    \pyinline{s = set([1, 1, 2, 2, 3, 3])} 
  
    \pyinline{s} \\  
    {1, 2, 3}
    
    ~ (2) 不可变集合(frozenset):不可添加和删除元素,可哈希的,能用作字典的键,能做其他集合的元素
\end{frame}

\begin{frame}[fragile]{集合——添加和删除}
    \h set集合是可变的,可以用add(key)进行添加 

    ~ \pyinline{s.add(4)} 

    ~ \pyinline{s} \\ 
    {1, 2, 3, 4}

    \h 如果要删除一个key,用remove(key)方法 

    ~ \pyinline{s.remove(4)} 

    ~ \pyinline{s} \\
    {1, 2, 3}
\end{frame}

\begin{frame}[fragile]{集合——交集和并集}
  \h 两个set可以做数学意义上的交集、并集等操作

  \pyinline{s1 = set([1, 2, 3])} 

  \pyinline{s2 = set([2, 3, 4])} 

  \pyinline{s1 \& s2} \\ 
  {2, 3}

  \pyinline{s1 | s2} \\
  {1, 2, 3, 4}
  
\end{frame}

\begin{frame}[fragile]{练习}
  \begin{easylist}
    & 给定一个字符串变量s,编写Python代码统计字符串中每个字符的出现频度。
    && 例如:
    
    $s = 'hello'$,则输出:

    e:1, h:1, l:2, o:1
  \end{easylist}

 \begin{python}
s = 'hello world'
d = {}
for ch in s:
     if d.get(ch)==None:
         d[ch] = 1
     else:
         d[ch] += 1
 \end{python}
\end{frame}


\begin{frame}[plain]
  \begin{center}
    \Simley{1}{10}{1}

    \Huge ---END---
  \end{center}
\end{frame}

%%% Local Variables:
%%% mode: latex
%%% TeX-master: "../python"
%%% End:
