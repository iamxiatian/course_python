\section{模块}

\begin{frame}[fragile]{CH5 模块}
  \begin{enumerate}
     \item 模块介绍
     \item 模块的导入方式
     \item 模块的作用域
     \item 模块的测试
     \item 模块导入的路径搜索
     \item 包
  \end{enumerate}
\end{frame}

\begin{frame}[fragile]{模块介绍}
  \begin{easylist} \easyitem
    & 随着程序代码越写越多,在一个文件里代码就会越来越长,越来越不容易维护。
    & 为了编写可维护的代码,把函数分组放到不同的文件里,这样,每个文件包含的代码
    就相对较少,很多编程语言都采用这种组织代码的方式。
    & 在Python中,一个.py文件就称之为一个模块(Module)。
    && 模块:把多个函数组织到一起,方便其他程序调用
    && 提高了代码的可维护性
    && 编写代码不必从零开始。当一个模块编写完毕,就可以被其他地方引用。

    & 之前我们编写的程序也保存在.py文件中,程序和模块的区别在于:
    && 程序的设计目标是运行
    && 模块的设计目标是由其他程序导入并使用
  \end{easylist}
\end{frame}


\begin{frame}[fragile]{模块的导入方式}
  \begin{easylist}
    & import importable
    & import importable1, importable2, ..., importableN
    & import importable as preferred\_name
    & from importable import *
  \end{easylist}
\end{frame}

\begin{frame}[fragile]{标准库中的模块使用示例}
  \begin{python}
import sys
from pprint import pprint
pprint(sys.path)
  \end{python}

['', \\
 '/usr/lib/python3.4', \\
 '/usr/lib/python3.4/plat-x86\_64-linux-gnu', \\
 '/usr/lib/python3.4/lib-dynload', \\
 '/usr/local/lib/python3.4/dist-packages', \\
 '/usr/lib/python3/dist-packages']

\end{frame}


\begin{frame}[fragile, allowframebreaks]{自定义模块}
  \lstinputlisting[keywordstyle=\ttfamily,
  basicstyle=\rmfamily\normalsize]{src/ch5/hello.py}
  
  \newpage
  ~\\
  \begin{easylist}
    & 行1的注释可以让hello.py文件直接在Unix/Linux/Mac上运行
    & 行2的注释表示.py文件本身使用标准UTF-8编码
    & 第4到6行是一个字符串,表示模块的文档注释,任何模块代码的第一个字符串都被视
    为模块的文档注释
    & 第8行导入了引用的模块
    & 第10行使用\_\_author\_\_变量把作者写进去
  \end{easylist}

\end{frame}


\begin{frame}[fragile]{如何运行}
  \begin{easylist}
    & 方式1:
    && 保存到hello.py文件中
    && 进入命令行,通过cd命令进入hello.py文件所在的目录
    &&& python3 hello.py
    &&& python3 hello.py Tom    
    & 方式2:
    && 启动python交互环境
    &&& \pyinline{import hello}
    &&& \pyinline{hello.sayHi()}
    &&& \pyinline{help(hello)}
    & 观察sys.argv是否包含了模块对应的文件名称
  \end{easylist}

\end{frame}


\begin{frame}[fragile]{模块的作用域}
  \begin{easylist}
    & 在一个模块中,我们可能会定义很多函数和变量,但有的函数和变量我们希望给别人
    使用,有的函数和变量我们希望仅仅在模块内部使用。
    && 通过\_前缀定义的函数和变量只能在模块内部访问
    && 其他函数和变量则是公开可访问的
    && \_\_xxx\_\_ 这样的变量可以被直接引用,但通常有特殊含义
    && 如\_\_name\_\_,  \_\_author\_\_
    & private函数和变量“不应该”被直接引用,而不是“不能”被直接引用,是因为Python
    并没有一种方法可以完全限制访问private函数或变量
  \end{easylist}
\end{frame}


\begin{frame}[fragile, allowframebreaks]{作用域示例: hello2.py}
    \lstinputlisting[keywordstyle=\ttfamily,
  basicstyle=\rmfamily\normalsize]{src/ch5/hello2.py}
 
 \begin{easylist}
    & 请分别用命令行和Python交互环境进行测试
    & 问题:能够在交互环境中通过hello2.\_sayInChinese()访问私有方法?
    & 实验:添加代码,使得程序能够根据命令行传入的参数,决定源代码26行处是调用
  \_sayInChinese()还是\_sayInEnglish()
  && 假设命令行传入的第一个有效参数用于指定语言,中文对应为zh,英文对应为en
  \end{easylist}
  
\end{frame}


\begin{frame}[fragile, allowframebreaks]{hello\_lang.py}
    \lstinputlisting[keywordstyle=\ttfamily,
  basicstyle=\rmfamily\normalsize]{src/ch5/hello_lang.py}
 
 \begin{easylist}
    & python3 hello\_lang.py zh
    & python3 hello\_lang.py en
    & python3 hello\_lang.py zh Tom
    & python3 hello\_lang.py en Tom
  \end{easylist}
\end{frame}


\begin{frame}[fragile]{模块的测试}
  \begin{easylist}
    & 模块本身用于定义函数、类及其他一些内容
    & 在模块中添加一些检查模块本身是否正常工作的测试代码非常有用
    \lstinputlisting[keywordstyle=\ttfamily,
    basicstyle=\rmfamily\normalsize]{src/ch5/hello3.py}   
    & 打开Python交互环境测试
    && \pyinline{import hello3}
    && \pyinline{hello3.hello()}
  \end{easylist}
\end{frame}


\begin{frame}[fragile]{\_\_name\_\_}
  \begin{easylist}
    & \pyinline{hello3.__name__}
    & \pyinline{__name__}
    & 因此,在测试模块时,可以通过如下方式:
    \begin{python}
      if __name__ == '__main__':
          test_suite...
    \end{python}
    && 此时,如果将模块作为独立的程序,条件判断将会满足,继续执行测试代码
    && 如果是import引入模块,则条件表达式不成立,测试代码被忽略
  \end{easylist}
\end{frame}


\begin{frame}[fragile]{如何让Python找到自定义的模块}
  \begin{enumerate}
  \item 在源代码目录下执行python
  \item  设置sys.path
  \end{enumerate}

  \lstinputlisting[keywordstyle=\ttfamily,
    basicstyle=\rmfamily\normalsize]{src/ch5/path_test.py} 
\end{frame}

\subsection{包}
\begin{frame}[fragile]{包的处理}
  \begin{easylist}
    & 包是一个有层次的文件目录结构,由模块和子包组成。
    && 为平坦的名称空间加入了有层次的组织结构
    && 允许程序员把有联系的模块组织到一起
    && 允许分发者使用目录结构而非一大堆文件
    && 有助于解决模块名称冲突问题
  \end{easylist}

\end{frame}


\begin{frame}[fragile]{\_\_init\_\_.py文件}
  \begin{easylist}
    & python的每个模块的包中,都有一个\_\_init\_\_.py文件,有了这个文件,我们才
    能导入这个目录下的module
    && 该文件可以为空
    && 我们在导入一个包时,实际上导入了它的\_\_init\_\_.py文件
  \end{easylist}

\end{frame}

\begin{frame}[fragile]{包的示例}
  graphics/ \\
  ~~~~\_\_init\_\_.py \\
  ~~~~primitive/ \\
  ~~~~~~~~\_\_init\_\_.py \\
  ~~~~~~~~line.py \\
  ~~~~~~~~fill.py \\
  ~~~~~~~~text.py \\
  ~~~~formats/ \\
  ~~~~~~~~\_\_init\_\_.py \\
  ~~~~~~~~png.py \\
  ~~~~~~~~jpg.py \\
\end{frame}

\begin{frame}[fragile]{以上包结构的引用方式}
  \begin{python}
    import graphics.primitive.line
    from graphics.primitive import line
    from graphics.primitive import *
    import graphics.formats.jpg as jpg
  \end{python}
  
  \begin{easylist}
    & 第1行在使用line中的方法时,只能用全名称引用:
    && graphics.primitive.line.xxxx
    & 第2行和第3行的方式,则可以直接使用如下方式
    && line.xxxx, fill.xxxx
    && jpg.xxxx
  \end{easylist}

\end{frame}

\begin{frame}[fragile]{练习}
  \begin{easylist}
    & 实现以上包结构
    && line.py, fill.py 和 text.py中分别写一个draw()方法,在该方法中输入一个简单
    的字符串
    && 在png.py和jpg.py中添加open()方法和close()方法,方法本身只输出一个提示字符
    串即可
  \end{easylist}
\end{frame}


\begin{frame}[fragile]{\_\_init\_\_.py中的\_\_all\_\_}
  \begin{easylist}
    & 通过from p1.p2 import *,可以一次性引入包里面的所有子模块
    & 如果想限定默认引入的子模块集合,可以通过设置\_\_init\_\_.py,例如:
    && 在\_\_init\_\_.py中添加如下内容:\\
    \_\_all\_\_ = ['line', 'fill']
    && from graphics.primitive import *
    && 请完善代码并测试能够直接调用line.xxx()和text.xxxx(),并分析原因
  \end{easylist}
\end{frame}


\begin{frame}[plain]
  \begin{center}
    \Simley{1}{10}{1}

    \Huge ---END---
  \end{center}
\end{frame}

%%% Local Variables:
%%% mode: latex
%%% TeX-master: "../python"
%%% End:
