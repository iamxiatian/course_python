\section{CH7 面向对象编程}

\begin{frame}[fragile]{CH7 面向对象编程}
  \begin{easylist} \easyitem
    & 简介
    & 类和实例
    & 数据封装
    & 继承和多态
    & Duck Typing
    & 获取对象信息
    & 实例属性和类属性
    & 高级特性
    & 小结
  \end{easylist}
\end{frame}

\subsection{简介}
\begin{frame}[fragile]{\newsec 面向对象编程简介}
  \begin{easylist}
    & 面向对象编程OOP(Object Oriented Programming)
    && 一种程序设计思想,OOP把对象作为程序的基本单元,一个对象包含了数据和操作
    数据的函数。
    && 面向过程的程序设计把计算机程序视为一系列的命令集合,即一组函数的顺序执
    行
    && OOP通过把大块函数切割为小块函数来降低系统的复杂度
    && OOP把计算机程序看做是一组对象的集合,程序执行就是一些列消息在各个对象之
    间传递,给对象发消息实际上就是调用对象对应的关联函数
    & OOP特点
    && 封装、继承、多态
    & 类(Class)与实例(Instance)
  \end{easylist}
\end{frame}

\begin{frame}[fragile]{举例: 面向过程方式}
  处理学生成绩并打印输出,面向过程设计方式:

  \begin{python}
std1 = {'name': '韩寒', 'score':85}    
std2 = {'name': '黄锤', 'score':90}

def print_score(std):
    print('%s: %s' % (std['name'], std['score']))

print_score(std1)
print_score(std2)
  \end{python}
\end{frame}

\begin{frame}[fragile]{举例: 面向对象方式}
  \begin{python}
class Student(object):
    def __init__(self, name, score):
        self.name = name
        self.score = score

    def print_score(self):
        print('%s: %s' % (self.name, self.score))

if __name__ == '__main__':
    han = Student('韩寒', 85)
    huang = Student('黄锤', 90)
    han.print_score()
    huang.print_score()
  \end{python}
\end{frame}


\begin{frame}[fragile]{例子解释}
  \begin{easylist}
    & class关键字
    & object
    & self
  \end{easylist}
\end{frame}


\subsection{类和实例}
\begin{frame}[fragile]{\newsec 类和实例}
  \begin{easylist}
    & 类是抽象的模板,比如Student类
    & 实例是根据类创建出来的一个个具体的“对象”
    & 每个对象都拥有相同的方法,但各自的数据可能不同。
  \end{easylist}
\end{frame}

\begin{frame}[fragile]{类的定义}
  通过class关键字定义类

  \begin{python}
class Student(object):
    pass
  \end{python}

  \begin{easylist}
    & class后面紧接着是类名
    & 类名通常是大写开头的单词
    & 类名后紧接着是(object),表示该类是从哪个类继承下来的
    && object是所有类最终都会继承的类
  \end{easylist}
\end{frame}


\begin{frame}[fragile]{类的实例化}
  类名+()实现, 例如:

  \begin{python}
    xiaoming = Student('小明', 85)
    print(xiaoming)    
    print(Student)
  \end{python}

  <\_\_main\_\_.Student object at 0x7f16e99ee710> \\
  <class '\_\_main\_\_.Student'>
\end{frame}

\begin{frame}[fragile]{实例的变量绑定}
  可以自由地给一个实例变量绑定属性和方法,比如,给实例xiaoming绑定一个name属性,
  和speak()方法:

\begin{python}
def speak(something):
    print('speak:' , something)

xiaoming.name = 'xiaoming'
xiaoming.speak = speak
print(xiaoming.name)
xiaoming.speak('hello')
\end{python}
\end{frame}


\begin{frame}[fragile]{实例的初始化}
  通过定义特殊的\_\_init\_\_方法,在创建实例的时候,可以绑定期望的属性

\begin{python}
  class Student(object):

    def __init__(self, name, score):
        self.name = name
        self.score = score
\end{python}

\begin{easylist}
  & \_\_init\_\_方法的第一个参数永远是self,表示创建的实例本身
  & 在\_\_init\_\_方法内部,可以把各种属性绑定到self,因为self就指向创建的实例本
  身。

\end{easylist}

\end{frame}


\begin{frame}[fragile]{\_\_init\_\_()解释}
  \begin{easylist}
    & 有了\_\_init\_\_方法,在创建实例的时候,就不能传入空的参数了,必须传入与该
    方法匹配的参数,但self不需要传,Python解释器自己会把实例变量传进去
  
    \pyinline{xiaoli = Student('小李', 80)}
    & 和普通的函数相比,在类中定义的函数只有一点不同,就是第一个参数永远是实例变
    量self,并且,调用时,不用传递该参数。除此之外,类的方法和普通函数没有什么区
    别,所以,仍然可以用默认参数、可变参数、关键字参数和命名关键字参数。

  \end{easylist}
\end{frame}

\subsection{数据封装}
\begin{frame}[fragile]{\newsec 数据封装}
  \begin{easylist}
    & 实例中拥有的属性、方法,被封装到实例中
    & 在实例中使用本身的属性或方法,用self.xxx方式
    & 在类中定义方法时,第一个参数必须为self
    & 在外部调用方法时,无需指定self,类会自动把self传入
    & 封装的优点是调用简单,无需关心内部实现细节
  \end{easylist}
\end{frame}


\subsection{继承和多态}
\begin{frame}[fragile]{\newsec 继承和多态}
  \begin{block}{}
    在OOP程序设计中定义一个class的时候,可以从某个现有的class继承,新的class称为
    子类(Subclass),而被继承的class称为基类、父类或超类(Base class、Super
    class)。
  \end{block}

  \begin{python}
    class Postgraduate(Student):
        '''post graduate student'''
        pass
  \end{python}

  \color{blue}继承可以把父类的所有功能都直接拿过来,这样就不必从零做起,子类只需要新增自己特
  有的方法,也可以把父类不适合的方法覆盖重写(多态)。

\end{frame}


\begin{frame}[fragile]{多态}
  \begin{easylist}
    & 多态即多种状态
    & 同一个实体同时具有多种形式, 对基类的引用指向子类的对象
    && 例子中,子类和父类都存在相同的sayHi()方法,此时,子类的sayHi()覆盖了父类的
    sayHi(),在代码运行的时候,总是会调用子类的sayHi()
  \end{easylist}
\end{frame}


\begin{frame}[fragile, allowframebreaks]{多态示例}
  \lstinputlisting[keywordstyle=\ttfamily]{src/ch7/student.py}
\end{frame}


\begin{frame}[fragile]{isinstance}
  \begin{easylist}
    & 可以用isinstance判断实例的类型
  \end{easylist}

\begin{python}
  wang = Postgraduate('王二小', 80)
  isinstance(wang, Postgraduate) #True or False?
  isinstance(wang, Student) #True or False?
\end{python}
\end{frame}


\subsection{Duck Typing}
\begin{frame}[fragile]{\newsec Duck Typing\footnote{\url{https://en.wikipedia.org/wiki/Duck\_typing}}}
  \begin{easylist}
    & 对于静态语言(如Java),如果需要传入Student类型,则传入的对象必须是Student类
    型或者它的子类  
    & 对于Python这样的动态语言,只需要保证传入的对象有sayHi()方法就可以
    & 这种设计方法称之为鸭子类型:
    && \color{blue} 一个对象有效的语义,不是由继承自特定的类或实现特定的接口,而是由当前方法
    和属性的集合决定。
  \end{easylist}

\begin{python}
class Robot():
    def sayHi(self):
        print("I'm robot")
        
meet(Robot())    
\end{python}
\end{frame}

\begin{frame}[fragile]{Duck Typing}
  \begin{block}{Duck Test}
    {\huge "} When I see a bird that walks like a duck and swims like a duck and quacks
    like a duck, I call that bird a duck.{\huge "}

    \begin{flushright}
      \scriptsize{--- Indiana poet James Whitcomb Riley (1849--–1916)}
    \end{flushright}
  \end{block}
\end{frame}

\begin{frame}[fragile, allowframebreaks]{Duck Typing示例}
  \lstinputlisting{src/ch7/duck.py}
\end{frame}


\subsection{获取对象信息}
\begin{frame}[fragile]{\newsec 获取对象信息}
  \begin{easylist}
    & 当我们拿到一个对象的引用时,如何知道这个对象是什么类型、有哪些方法呢?
    && isinstance(): 判断对象是否为某个类的实例 
    && type(): 判断对象的类型
    && dir(): 获得对象的所有属性和方法
  \end{easylist}
\end{frame}

\begin{frame}[fragile]{type()}
  \begin{python}
    type(123) == type(456) 
    type(123) == int 
    type('hello') == str 

    import types
    type(abs) == types.BuiltinFunctionType
    type(lambda x:x) == types.LambdaType
    type((x for x in range(10)))==types.GeneratorType

    def hi():
        pass
    type(hi) == types.FunctionType
  \end{python}
\end{frame}

\begin{frame}[fragile]{dir()}
  \begin{easylist}
    & 获得一个对象的所有属性和方法,可以使用dir()函数,它返回一个包含字符串的list
  \end{easylist}

  \begin{python}
    wang = Postgraduate('王小二', 80)
    dir(wang)

    ['__class__', '__delattr__', '__dict__', '__dir__', '__doc__', '__eq__', '__format__', '__ge__', '__getattribute__', '__gt__', '__hash__', '__init__', '__le__', '__lt__', '__module__', '__ne__', '__new__', '__reduce__', '__reduce_ex__', '__repr__', '__setattr__', '__sizeof__', '__str__', '__subclasshook__', '__weakref__', 'name', 'print_score', 'sayHi', 'score']
  \end{python}
\end{frame}


\subsection{实例属性和类属性}
\begin{frame}[fragile]{\newsec 实例属性和类属性}
  \begin{easylist}
    & 实例属性隶属于实例
    && 通过self或实例绑定属性
    & 类属性隶属于类,其所有实例均可以访问到
    && 直接在class中定义的属性
  \end{easylist}
\end{frame}

\begin{frame}[fragile, allowframebreaks]{实例属性和类属性示例}
  \begin{python}
#实例属性示例
class Student(object):
    def __init__(self, name):
        self.name = name #通过self绑定属性
        
s = Student('Bob')
s.score = 90 #通过实例绑定属性
print("%s : %s" % (s.name,  s.score))
\end{python}

\newpage

\begin{python}
#类属性示例
class Student(object):
    name = 'Student'
\end{python}

  $>>>$ s = Student() \# 创建实例s \\
  $>>>$ print(s.name) \# 打印name属性,因为实例并没有name属性,所以会继续查找class的
  name属性 \\
  Student \\
  $>>>$ print(Student.name) \# 打印类的name属性 \\
  Student \\
  $>>>$ s.name = 'Michael' \# 给实例绑定name属性 \\

  \newpage

  $>>>$ print(s.name) \# 由于实例属性优先级比类属性高,因此,它会屏蔽掉类的name属性 \\
  Michael \\
  $>>>$ print(Student.name) \# 但是类属性并未消失,用Student.name仍然可以访问 \\
  Student \\
  $>>>$ del s.name \# 如果删除实例的name属性 \\
  $>>>$ print(s.name) \# 再次调用s.name,由于实例的name属性没有找到,类的name属性就显
  示出来了 \\
  Student    
\end{frame}


\subsection{小结}
\begin{frame}[fragile]{\newsec 面向对象编程小结}
  \begin{easylist}
    & 类是创建实例的模板,而实例则是一个一个具体的对象,各个实例拥有的数据都互相独立,互不影响;

    & 方法就是与实例绑定的函数,和普通函数不同,方法可以直接访问实例的数据;
    & 通过在实例上调用方法,我们就直接操作了对象内部的数据,但无需知道方法内部的实现细节。
    & 和静态语言不同,Python允许对实例变量绑定任何数据
    && 也就是说,对于两个实例变量,虽然它们都是同一个类的不同实例,但拥有的变量
    名称都可能不同
    & Duck Typing
    & 实例属性与类属性
  \end{easylist}
\end{frame}

\begin{frame}[fragile]{练习}
  \begin{easylist}
    & 假设磁盘中存在一个学生成绩的文本文件,每行格式如下:
    && 123,成工,高等数学,80
    && 123,成工,线性代数,85
    && 124,王小二,线性代数,80
    && $\cdots$
    & 利用OOP设计思想,实现以下功能:
    && 按照学号或姓名查找满足条件的学生的所有成绩信息,并输出其平均分
    && 根据课程名称输出所有学生的成绩,并输出平均成绩、最高分和最低分
  \end{easylist}

\end{frame}


\begin{frame}[fragile, allowframebreaks]{Test}
  \begin{python}
class Query():
    def __init__(self, filename):
        slef.filename = filename
        
    def _parse_line(self, line):
        return 123, 'zhang', 'math', 80
        
    def find_by_number(self, num):
        scores = {}
        f = open(self.filename, 'r')
        for line in f:
            n, name, course, score = self._parse_line(line)
            if n == num:
                scores[course] = score
                
        total = 0
        for course, score in scores.items():
            print(course, '==>', score)
            total += score
        print('avg', total/len(scores))    

    if __name__ == '__main__':
        q = Query('some_file.txt')
        q.find_by_number(123)
  \end{python}
\end{frame}



\subsection{Python面向对象的高级特性}
\begin{frame}[fragile]{Python面向对象的高级特性}
  \begin{easylist}
    & \_\_slots\_\_
    & @property
    & 多重继承
    & 定制类
    & 枚举类
    & 元类
  \end{easylist}
\end{frame}


\subsubsection{\_\_slots\_\_}
\begin{frame}[fragile, allowframebreaks]{\_\_slots\_\_的引入}
  实例绑定与类的绑定示例
  \begin{python}
class Student(object):
    pass

s = Student()
s.name = 'Lucy'
print(s.name) # 输出Lucy

def set_age(self, age):
    self.age = age

from types import MethodType
s.set_age = MethodType(set_age, s) # 给实例绑定方法
s.set_age(25) # 调用实例方法
print(s.age) # 测试结果

s2 = Student() # 创建新的实例
s2.set_age(25) # 尝试调用方法
  \end{python}

  调用s2.set\_age(25)会给出错误提示:对象Student没有属性set\_age,此时,为了给所有
  实例都绑定方法,可以给class绑定方法,如下:
\begin{python}
def set_score(self, score):
    self.score = score

Student.set_score = set_score # 在类级别上绑定方法

s.set_score(100)
print(s.score)
s2.set_score(99)
print(s2.score)
\end{python}

\begin{easylist}
  & 新问题:
  && 如果要限制对实例的属性随意赋值,该怎么处理?
  && \_\_slots\_\_
\end{easylist}
\end{frame}



\begin{frame}[fragile]{\_\_slots\_\_}
  \_\_slots\_\_规定了class能添加的属性

\begin{python}
class Student(object):
    __slots__ = ('name', 'age')

s = Student()
s.age = 25
s.score = 90 # Error!
\end{python}

AttributeError: 'Student' object has no attribute 'score'

注意:\_\_slots\_\_定义的属性仅对当前类实例起作用,对继承的子类是不起作用的
\end{frame}


\subsubsection{@property}
\begin{frame}[fragile, allowframebreaks]{@property的引入}
  \begin{easylist}
    & 在绑定属性时,如果我们直接把属性暴露出去,虽然写起来很简单,但是,没办法检
    查参数
    && 例如,成绩score,我们希望把成绩的范围限制到0--100之间
    && 思路:增加get\_score()和set\_score()对进行读取和修改
  \end{easylist}

  \begin{python}
class Student(object):
    def get_score(self):
        return self._score

    def set_score(self, value):
        if not isinstance(value, int):
            raise ValueError('score必须是整数!')    
        if value < 0 or value > 100:
            raise ValueError('score必须是一个0到100之间的数字')
        self._score = value
  \end{python}

  \newpage
  \begin{python}
s = Student()
s.set_score(60)
print(s.get_score())
s.set_score(999) # Error!
  \end{python}

  \begin{easylist}
    & 上面的调用方法又略显复杂
    & 能否仍然使用s.score = 某个数字, 同时还能判断赋予的数值是否满足要求?
    & Python的@property装饰器负责把一个方法变成属性调用
  \end{easylist}
\end{frame}

\begin{frame}[fragile, allowframebreaks]{@property示例}
  \begin{python}
class Student(object):
    |\color{red}@property|
    def score(self):
        return self._score

    |\color{red}@score.setter|
    def score(self, value):
        if not isinstance(value, int):
            raise ValueError('score必须是整数!')    
        if value < 0 or value > 100:
            raise ValueError('score必须是一个0到100之间的数字')
        self._score = value   
\end{python}

\newpage
\begin{python}
s = Student()
s.score = 60
print(s.score)
|\color{red}s.score = 999| # Error!
\end{python}
\end{frame}

\subsubsection{多重继承}
\begin{frame}[fragile]{多重继承}
  \begin{easylist}
    & 继承是面向对象编程的一个重要的方式,因为通过继承,子类就可以扩展父类的功能。
    & 假设我们要实现以下4种动物:
    && Dog - 狗
    && Bat - 蝙蝠
    && Parrot - 鹦鹉
    && Ostrich - 鸵鸟
  \end{easylist}

\end{frame}

\begin{frame}[fragile]{设计思路}
  \begin{columns}
    \begin{column}{.5\textwidth}
      按照哺乳动物和鸟类归类:


    \end{column}
    
    \begin{column}{.5\textwidth}
      按照“能跑”和“能飞”来归类
      
    \end{column}
  \end{columns}
\end{frame}

\begin{frame}[fragile]{设计思路}
  \begin{python}
class Animal(object):
    pass

# 大类:
class Mammal(Animal):
    pass

class Bird(Animal):
    pass

# 各种动物:
class Dog(Mammal):
    pass

class Bat(Mammal):
    pass

class Parrot(Bird):
    pass

class Ostrich(Bird):
    pass    
  \end{python}
\end{frame}


\begin{frame}[fragile]{定义Runnable和Flyable}
  \begin{python}
class Runnable(object):
    def run(self):
        print('Running...')

class Flyable(object):
    def fly(self):
        print('Flying...')
  \end{python}
\end{frame}

\begin{frame}[fragile]{多重继承示例}
  \begin{python}
class Dog(Mammal, Runnable):
    pass

class Bat(Mammal, Flyable):
    pass    
  \end{python}

  \begin{easylist}
    & 通过多重继承,一个子类就可以同时获得多个父类的所有功能。
  \end{easylist}
\end{frame}

\begin{frame}[fragile]{MixIn}
  \begin{easylist}
    & 在设计类的继承关系时,通常,主线都是单一继承下来的
    && 如:Ostrich继承自Bird
    & 如果需要“混入”额外的功能,通过多重继承就可以实现
    && 比如,让Ostrich除了继承自Bird外,再同时继承Runnable
    & 这种设计方法通常称之为MixIn
    && 为了更好地看出继承关系,通常采用MixIn作为功能性的父类名称的后缀
  \end{easylist}
\end{frame}

\begin{frame}[fragile]{MixIn实例}
  \begin{python}
class RunnableMixIn(object):
    def run(self):
        print('Running...')

class FlyableMixIn(object):
    def fly(self):
        print('Flying...')

class Dog(Mammal, RunnableMixIn):
    pass    
  \end{python}
\end{frame}

\subsubsection{对类进行定制}
\begin{frame}[fragile]{对类进行定制}
  \begin{easylist}
    & \_\_len\_\_
    & \_\_str\_\_
    & \_\_repr\_\_
    & \_\_iter\_\_ 与 \_\_next\_\_
    & \_\_getitem\_\_
    & \_\_getattr\_\_
    & \_\_call\_\_
  \end{easylist}
\end{frame}

\begin{frame}[fragile]{\_\_len\_\_}
  \begin{python}
    class Panda(object):
    def __init__(self):
        pass

    def __len__(self):
        return 10

panda = Panda()
len(panda)
  \end{python}
\end{frame}

\begin{frame}[fragile]{\_\_str\_\_与\_\_repr\_\_}
  \begin{python}
class Panda(object):
    def __init__(self):
        pass

    def __str__(self):
        return '熊猫'

panda = Panda()
print(panda) # 熊猫
panda    # <__main__.Panda at 0x7f20e414db38>
  \end{python}
  
  \begin{easylist}
    & 如何解决非print时输出地址的问题?
  \end{easylist}
\end{frame}


\begin{frame}[fragile]{\_\_str\_\_与\_\_repr\_\_}
  \begin{python}
class Panda(object):
    def __init__(self):
        pass

    def __str__(self):
        return '熊猫'

    __repr__ = __str__
panda = Panda()
print(panda) # 熊猫
panda    # 熊猫
  \end{python}  
\end{frame}


\begin{frame}[fragile]{\_\_iter\_\_ 与 \_\_next\_\_}
  \begin{easylist}
    & 方便通过for循环对对象维持的数据进行遍历
    && 例如:假设拥有一个斐波那契数列的类Fib,通过以下语句循环输出:
    \begin{python}
for n in Fib(10):
     print(n)
    \end{python}
    && 我们希望输出1, 1, 2, 3, 5, 8, 13, 21, 34, 55
    && 怎么实现Fib? 
  \end{easylist}
\end{frame}

\begin{frame}[fragile, allowframebreaks]{Fib}
  \lstinputlisting{src/ch7/fib.py}
\end{frame}

\begin{frame}[fragile]{扩展练习}
  \begin{easylist}
    & 用Python编写程序,对本机指定目录下的文件进行扫描,把所有指定后缀名的文档
    名称统一输出到一个文本文件中。
  \end{easylist}
\end{frame}


\begin{frame}[fragile]{END}
  ~
\end{frame}

%%% Local Variables:
%%% mode: latex
%%% TeX-master: "../python"
%%% End:
