\section{Web处理}

\begin{frame}[fragile]{Web服务器}
  \begin{easylist} \easyitem
    & 最简单的Python Web服务器
    & Flask
  \end{easylist}
\end{frame}

\begin{frame}[fragile]{Simple Http Server}
  \begin{easylist}

    & 最轻便的Web服务器\footnote{参数m的作用参考:
      \url{http://www.tuicool.com/articles/jMzqYzF}}:

    & 启动方式:
    && python -m http.server
  \end{easylist}
\end{frame}

\begin{frame}[fragile]{Flask}
  \begin{easylist}
    & Flask是一种简便的基于Python语言的Web应用程序开发框架
    && Flask is a microframework for Python based on Werkzeug, Jinja 2 and good intentions. And before you ask: It's BSD licensed!
    & 相关中文文档可在线参考:
    &&
    \scriptsize{\url{http://dormousehole.readthedocs.io/en/latest/quickstart.html}
    }
    & 安装:
    && pip install Flask
    & 文档
    && http://flask.pocoo.org/docs/0.10/.latex/Flask.pdf
  \end{easylist}
\end{frame}


\begin{frame}[fragile]{Flask简单例子}
  \lstinputlisting[keywordstyle=\ttfamily]{src/web/flask1.py}
\end{frame}

\begin{frame}[fragile, allowframebreaks]{Flask简单例子}
  \lstinputlisting[keywordstyle=\ttfamily]{src/web/flask2.py}
\end{frame}

\begin{frame}[fragile]{练习}
  实现一个Web程序,在网页上输入一个url地址,提交后,通过浏览器显示该地址所包含的所有图片。
\end{frame}


\begin{frame}[fragile]{科研人员在线社交媒体行为分析}
  \begin{easylist}
    & 编写一个Web应用程序,记录每个高校在社交媒体上的活动信息
    && 第一步完成基本信息的收集管理
    & 基本信息包括
    && 学校名称, 学院, 专业, 姓名, 性别, 出生年, 简介, 微博UID, 微博注册日期, 最
    后访问日期
  \end{easylist}

  \begin{easylist}
    & 基本信息管理功能:
    && 基本信息录入
    && 重复检测(根据学校、学院、姓名判断重复,或者根据微博UID判断重复)
    && 检索统计:
  \end{easylist}
\end{frame}

%%% Local Variables:
%%% mode: latex
%%% TeX-master: "../python"
%%% End:
