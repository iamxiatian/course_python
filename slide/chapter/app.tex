\section{Python应用开发}

\begin{frame}[fragile]{CH9 Python应用开发}
  \begin{easylist} \easyitem
    & Web服务器
    && python -m http.server
    && Flask
    & 数据库
    && MySQL
  \end{easylist}

\end{frame}


\subsection{Web服务器}
\begin{frame}[fragile]{Web服务器}
  \begin{easylist} \easyitem
    & 最简单的Python Web服务器
    & Flask
  \end{easylist}
\end{frame}

\begin{frame}[fragile]{Simple Http Server}
  \begin{easylist}

    & 最轻便的Web服务器\footnote{参数m的作用参考:
      \url{http://www.tuicool.com/articles/jMzqYzF}}:

    & 启动方式:
    && python -m http.server
  \end{easylist}
\end{frame}

\begin{frame}[fragile]{Flask}
  \begin{easylist}
    & Flask是一种简便的基于Python语言的Web应用程序开发框架
    && Flask is a microframework for Python based on Werkzeug, Jinja 2 and good intentions. And before you ask: It's BSD licensed!
    & 相关中文文档可在线参考:
    &&
    \scriptsize{\url{http://dormousehole.readthedocs.io/en/latest/quickstart.html}
    }
    & 安装:
    && pip install Flask
    & 文档
    && http://flask.pocoo.org/docs/0.10/.latex/Flask.pdf
  \end{easylist}
\end{frame}


\begin{frame}[fragile]{Flask简单例子}
  \lstinputlisting[keywordstyle=\ttfamily]{src/web/flask1.py}
\end{frame}

\begin{frame}[fragile, allowframebreaks]{Flask简单例子}
  \lstinputlisting[keywordstyle=\ttfamily]{src/web/flask2.py}
\end{frame}


\subsection{操作数据库}
\begin{frame}[fragile]{操作数据库}
  \begin{easylist}
    & 商业数据库
    && Oracle
    && Microsoft SQL Server
    && IBM DB2
    && Sybase
    & 开源数据库
    && \textit{MySQL}
    && PostgreSQL
    && Sqlite: 嵌入式数据库
  \end{easylist}
\end{frame}


\begin{frame}[fragile]{MySQL}
  \begin{easylist}
    & MySQL是Web世界中使用最广泛的数据库服务器
    & 我们选择5.5以上版本进行测试
  \end{easylist}
\end{frame}


\begin{frame}[fragile]{MySQL的编码设置}
  \begin{easylist}
    & 把数据库默认的编码全部改为UTF-8,防止出现乱码问题
    & 配置文件的位置:
    && /etc/my.cnf或者/etc/mysql/my.cnf
    & Windows下,安装时请选择UTF-8编码
  \end{easylist}

  \begin{tcolorbox}[title=my.cnf的设置]
    [client]\\
    default-character-set = utf8\\

    [mysqld]\\
    default-storage-engine = INNODB\\
    character-set-server = utf8\\
    collation-server = utf8\_general\_ci
  \end{tcolorbox}

\end{frame}

\begin{frame}[fragile]{MySQL测试}
   利用命令行登录MySQL:mysql -u root -p

   查看MySQL的变量设置:\\
  
   \begin{tcolorbox}[title={mysql> show variables like '\%char\%';}]
     \scriptsize
\begin{verbatim}
+--------------------------+----------------------------+
| Variable_name            | Value                      |
+--------------------------+----------------------------+
| character_set_client     | utf8                       |
| character_set_connection | utf8                       |
| character_set_database   | utf8                       |
| character_set_filesystem | binary                     |
| character_set_results    | utf8                       |
| character_set_server     | utf8                       |
| character_set_system     | utf8                       |
| character_sets_dir       | /usr/share/mysql/charsets/ |
+--------------------------+----------------------------+
8 rows in set (0.00 sec)
\end{verbatim}
   \end{tcolorbox}

\end{frame}


\begin{frame}[fragile]{在MySQL中创建数据库和表}
\begin{verbatim}
create database test;
use test;

CREATE TABLE `users` (
    `id` int(11) NOT NULL AUTO_INCREMENT,
    `email` varchar(255) COLLATE utf8_bin NOT NULL,
    `password` varchar(255) COLLATE utf8_bin NOT NULL,
    PRIMARY KEY (`id`)
) ENGINE=InnoDB DEFAULT CHARSET=utf8 COLLATE=utf8_bin
AUTO_INCREMENT=1 ;
\end{verbatim}
\end{frame}

\begin{frame}[fragile]{MySQL驱动}
  \begin{easylist}
    & MySQL服务器以独立的进程运行,并通过网络对外服务
    & 需要支持Python的MySQL驱动来连接到MySQL服务器
    & Drviers
    && PyMySQL(我们选择PyMySQL连接MySQL数据库)
    &&& Pypi package: pymysql
    &&& import pymysql
    && MySQL Connector
    &&& Pypi Package: mysql-connector-python
    &&& import mysql.connector
  \end{easylist}
\end{frame}


\begin{frame}[fragile]{PyMySQL}
  \begin{easylist}
    & 主页:\url{https://github.com/PyMySQL/PyMySQL}
    & 安装:pip install PyMySQL

  \end{easylist}
\end{frame}

\begin{frame}[fragile, allowframebreaks]{PyMySQL使用示例}
  \lstinputlisting[keywordstyle=\ttfamily]{src/ch9/mysql.py}
\end{frame}

%\begin{frame}[fragile]{练习}
%  实现一个Web程序,在网页上输入一个url地址,提交后,通过浏览器显示该地址所包含的所有图片。
%\end{frame}


\begin{frame}[fragile]{科研人员在线社交媒体行为分析}
  \begin{easylist}
    & 编写一个Web应用程序,记录每个高校在社交媒体上的活动信息
    && 第一步完成基本信息的收集管理
    & 基本信息包括
    && 学校名称, 学院, 专业, 姓名, 性别, 出生年, 简介, 微博UID, 微博注册日期, 最
    后访问日期
  \end{easylist}

  \begin{easylist}
    & 基本信息管理功能:
    && 基本信息录入
    && 重复检测(根据学校、学院、姓名判断重复,或者根据微博UID判断重复)
    && 检索统计:
  \end{easylist}
\end{frame}

%%% Local Variables:
%%% mode: latex
%%% TeX-master: "../python"
%%% End:
