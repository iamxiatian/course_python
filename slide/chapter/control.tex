\section{控制流}

\begin{frame}[fragile]{CH3 控制流}
  \begin{easylist} \easyitem
    & if
    & while
    & for
    & break
    & continue
  \end{easylist}
\end{frame}

\begin{frame}[fragile]{补充:函数的简单定义}
  \begin{easylist}
    & 函数可通过def来定义,语法:
    \begin{python}
def function_name(param1, param2, ...):
    function_suite
    \end{python}
    & 例子
    \begin{python}
def test(name, age):
    print("{0} is {1} years old".format(name, age))

test('Mike', 20) #test function
    \end{python}
    & 函数的详细内容下次课讲解
  \end{easylist}

\end{frame}

\begin{frame}[fragile]{if}
  \begin{python}
if boolean_expression1:
    suite1
elif boolean_expression2:
    suite2
elif boolean_expression3:
    suite...
else:
    suite...
  \end{python}

  \begin{easylist} \easyitem
    & 可以有0到多个elif语句
    & 可以有0到1个else语句
  \end{easylist}
\end{frame}

\begin{frame}[fragile]{pass的应用}
  \begin{easylist}
    & 如果需要考虑某个特定的情况,但该情况出现时又不需要做什么,可以用pass 

  \begin{python}
if 5>2:
    pass
  \end{python}

    && pass也可以用在while,函数定义def等地方,共同的作用是保持语法缩进的正确性
  \end{easylist}
\end{frame}

\begin{frame}[fragile]{if实现的条件表达式}
  \begin{easylist}
    & 对于如下语句

    \begin{python}
speed = 200 #any value for test
if speed > 150:
    train = 'Harmony'
else:
    train = 'Normal'
    \end{python}

    & 可以缩写为一句:
    \begin{python}
train = 'Harmony' if speed > 150 else 'Normal'
    \end{python}
    
  \end{easylist}
\end{frame}


\begin{frame}[fragile]{循环 --- while}
  \begin{easylist}
    & 语法
    \begin{python}
while boolean_expression:
    while_suite
else:
    else_suite      
    \end{python}
    & else分支
    && 本身是可选的组成部分
    && 只要循环是正常终止,else分支总会执行
    && 由于break, 返回语句或异常导致跳出循环,则else不会执行
    && else的上述特点对于for循环,及try...except都是一样的 
  \end{easylist}

\end{frame}


\begin{frame}[fragile]{while示例}
  \begin{python}
def list_find(lst, target):
    index = 0
    while index<len(lst):
        if lst[index] == target:
            break
        index += 1
    else:
        index = -1
    return index

list_find(['ruc', 'irm'], 'irm') #??
list_find(['ruc', 'irm'], 'irm2') #??
  \end{python}
\end{frame}


\begin{frame}[fragile]{循环 --- for}
  \begin{easylist}
    & 语法
    \begin{python}
for expression in iterable:
    for_suite
else:
    else_suite      
    \end{python}

    && expression或者是一个单独的变量,或者是一个变量序列,一般以元组形式给出
    && 如果将元组或列表用于expression,则其中的每一数据项都会拆分到表达式的项
  \end{easylist}

\end{frame}

\begin{frame}[fragile]{for示例}
  \begin{python}
def list_find(lst, target):
    for index, x in enumerate(lst):    
        if lst[index] == target:
            break
    else:
        index = -1
    return index

list_find(['ruc', 'irm'], 'irm') #??
list_find(['ruc', 'irm'], 'irm2') #??
  \end{python}

  \begin{easylist}
    & 考虑以上代码中x的含义是什么
  \end{easylist}

\end{frame}


\begin{frame}[fragile]{break}
  \begin{easylist}
    & 提前终止循环语句的执行,如前面的例子
  \end{easylist}
\end{frame}

\begin{frame}[fragile]{continue}
  \begin{easylist}
    & 终止当前循环体内的后续语句执行,重新执行下一轮次的循环
    & 例如:统计一个数字列表中大于10的元素之和
    \begin{python}
def countall(lst):
    count = 0
    for n in lst:
        if n<=10: 
            continue
        count += n
    return count

countall([1,3,12,15]) # Result?
    \end{python}
  \end{easylist}
\end{frame}

\begin{frame}[fragile]{课堂练习}
  \begin{easylist}
    & 复习以前内容
    & 利用蒙特卡洛方法计算圆周率
    & 打印杨辉三角
  \end{easylist}
\end{frame}

\begin{frame}[fragile]{圆周率计算}
  \begin{python}
import random

def pi(count):
    total = 0
    in_circle = 0
    for i in range(count):
        total += 1
        x = random.uniform(-1,1)
        y = random.uniform(-1,1)
        distance = x**2 + y**2
        if(distance<=1):
            in_circle +=1
    return in_circle/total * 4

pi(1000000)    
  \end{python}
\end{frame}

\begin{frame}[plain]
  \begin{center}
    \Simley{1}{10}{1}

    \Huge ---END---
  \end{center}
\end{frame}


%%% Local Variables:
%%% mode: latex
%%% TeX-master: "../python"
%%% End:
