\section{Python标准库}

\begin{frame}[fragile]{CH6 Python标准库}
  \begin{easylist} \easyitem
    & 内置电池:Battery Included
    && 用于形容Python标准库
    & 标准模块
    && string 
    && io \& sys
    && optparse
    && math
    && random
  \end{easylist}
\end{frame}

\subsection{字符串标准模块:string}
\begin{frame}[fragile]{字符串标准模块:string}
  \begin{easylist}
    & 提供了字符串相关的一些常量 

    \pyinline{string.ascii\_letters} 

    \pyinline{string.ascii\_lowercase} 

    \pyinline{string.ascii\_uppercase} 

    \pyinline{string.digits} 

    \pyinline{string.hexdigits} 

    \pyinline{string.punctuation}

  \end{easylist}
\end{frame}


\subsection{IO输出相关模块}
\begin{frame}[fragile]{IO输出相关标准模块}
  \begin{easylist}
    & 标准输出
    && sys.stdout
    & 字符串输出
    && io.StringIO
    & 以下输出方式是等价的
  \end{easylist}

  \begin{python}
    import sys,io
    print("hello")
    print("hello", file=sys.stdout)
    sys.stdout.write("hello\n") #python3 中会同时输出字符串的数量
  \end{python}
\end{frame}

\begin{frame}[fragile]{StringIO}
  \begin{easylist}
    & 如果要把输出重定向到字符串中,在需要时再获取,可以StringIO
    && 注意:Python 2.x和3.x在输出时的行为不同
  \end{easylist}

  \begin{python}
    import sys, io
    out = io.StringIO()
    sys.stdout = out
    out.write('how are you')
    print('hello')
    print('guys')
    sys.stdout = sys.__stdout__
    print(out.getvalue())
  \end{python}
\end{frame}


\subsection{命令行参数模块}
\begin{frame}[fragile, allowframebreaks]{命令行参数模块:optparse}
  \begin{easylist}
    & 如何指定脚本运行的命令行参数
    && 例如,Linux命令 ls --l
    && Python提供的optparse模块对命令行参数的解析处理提供了良好的支持
  \end{easylist}

  \lstinputlisting[keywordstyle=\ttfamily]{src/ch6/cmdopt.py}
\end{frame}


\begin{frame}[fragile]{分析}
  \begin{easylist}
    & 说明
    && 长短参数名称
    && 参数对应的变量名称及获取方式
    && metavar参数:提醒用户该命令行参数所期待的参数,参数中的字符串会自动变为大写
    && parser.parse\_args()
    &&& 解析后得到的options拥有两个属性:filename和verbose

    & 假设文件保存到cmdopt.py,执行:
    && python3 cmdopt.py -f cmdopt.py,观察输出结果
    && python3 cmdopt.py -h
  \end{easylist}

\end{frame}


\subsection{数学标准模块:math}
\begin{frame}[fragile]{数学标准模块:math}
  \begin{easylist}
    & math.exp(x)
    \[e^x\]
    & math.sqrt(x)
    \[\sqrt{x}\]
    & math.pi
    & math.e
    & math.log
    && \pyinline{math.log(math.e)}
    && \pyinline{math.log2(2)}
    && \pyinline{math.log10(10)}
    && \pyinline{math.log1p(math.e - 1)}
  \end{easylist}
\end{frame}


\subsection{随机数模块:random}
\begin{frame}[fragile]{随机数标准模块:random}
  \begin{easylist}
    & 生成$[0, 1)$之间的随机数
    && \pyinline{import random}
    && \pyinline{random.random()}
    & random.gauss(mu, sigma)生成一个均值为mu,标准值为sigma的符合高斯分布的随机
    数
    && \pyinline{random.gauss(0, 1)}
  \end{easylist}
\end{frame}

\begin{frame}[fragile]{高斯分布模拟}
  \begin{easylist}
    & 通过random.gauss()生成一组随机数
    & 通过柱状图显示结果
  \end{easylist}
  \pause
  \begin{python}
    import matplotlib.pyplot as plt
    import random

    data = [random.gauss(0,1) for x in range(10000)]
    plt.hist(data, bins = 50)
    plt.show()
  \end{python}
\end{frame}

\begin{frame}[fragile]{Result}
  \includegraphics[width=0.8\textwidth]{figure/gauss1.png}
\end{frame}


\begin{frame}[fragile]{高斯分布模拟}
  \begin{easylist}
    & 通过高斯分布的概率密度函数生成(x, y)对,利用散点图显示结果
    \[ f(x) = \dfrac{1}{\sigma \sqrt{2 \pi}} e^{-\dfrac{(x-\mu)^2}{2 \sigma^2}} \]
  \end{easylist}
\end{frame}

\begin{frame}[fragile]{高斯分布模拟代码}
  \lstinputlisting[keywordstyle=\ttfamily]{src/ch6/gauss.py}
\end{frame}

\begin{frame}[fragile]{Result}
  \includegraphics[width=0.8\textwidth]{figure/gauss2.png}
\end{frame}


\subsection{日志处理模块: logging}
\begin{frame}[fragile]{日志处理标准库:logging}
  \begin{easylist}
    & 日志的作用
    
    & 日志的级别
    && DEBUG:详细的信息,通常只出现在诊断问题上
    && INFO:确认一切按预期运行
    && WARNING:一个迹象表明,一些意想不到的事情发生了,或表明一些问题在不久的将来(例如。磁盘空间低”)。这个软件还能按预期工作。
    && ERROR:个更严重的问题,软件没能执行一些功能
    && CRITICAL:一个严重的错误,这表明程序本身可能无法继续运行
    
    & 优先级关系
    && CRITICAL > ERROR > WARNING > INFO > DEBUG
  \end{easylist}
\end{frame}


\begin{frame}[fragile]{日志示例}
  
  \begin{python}
import logging

logging.debug('debug message')
logging.info('info message')
logging.warning('warning message')
logging.error('error message')
logging.critical('critical message')
  \end{python}

\begin{easylist}
  & 默认输出到控制台,级别在warning及以上的信息会输出,
  & 可以通过basicConfig设置输出方式和记录级别
\end{easylist}
\end{frame}

\begin{frame}[fragile]{日志示例:输出到文件}
  \begin{python}
import os
import logging


FILE = os.getcwd()
logging.basicConfig(filename=os.path.join(FILE, 'log.txt'),
                    level=logging.DEBUG)
logging.debug('some debug messages...')
logging.info('some info messages...')
logging.warning('some warning messages...')
  \end{python}
\end{frame}


% \subsection{正则表达式模块:re}
% \begin{frame}[fragile]{正则表达模块:re}
%   ~
% \end{frame}


% \subsection{os与os.path标准模块}
% \begin{frame}[fragile]{os与os.path标准模块}
%   ~
% \end{frame}

\begin{frame}[fragile]{第3方扩展:Requests}
  \begin{easylist}
    & 通过Requests包可以更方便地实现Web数据的采集
    && Install: pip install requests
  \end{easylist}

  \begin{python}
import requests

url = 'http://download.pchome.net/wallpaper/zhiwu/'
response = requests.get(url)
print(response.text)    
  \end{python}
\end{frame}

\begin{frame}[fragile]{BeautifulSoup}
  \begin{easylist}
    & 怎么解析抓取到的网页,例如,如何抽取网页中的图片?
    & BeautifulSoup --- Html Parser
    && Install: pip install beautifulsoup4
  \end{easylist}
\end{frame}

\begin{frame}[fragile]{Example}
  \begin{python}
import requests

url = 'http://download.pchome.net/wallpaper/zhiwu/'
response = requests.get(url)
print(response.text)

from bs4 import BeautifulSoup as soup
doc = soup(response.text, 'html')
images = [element.get('src') for element in doc.find_all('img')]    
  \end{python}
\end{frame}

\begin{frame}[fragile]{课堂练习}
  \begin{easylist}
    & 编写程序,实现对任意指定网页,下载该网页包含的所有图片到指定的文件夹当中
  \end{easylist}

\end{frame}

\begin{frame}[fragile]{Thanks}
  ~
\end{frame}

%%% Local Variables:
%%% mode: latex
%%% TeX-master: "../python"
%%% End:
