\documentclass[table, t,13pt]{beamer} 
\usepackage{graphicx}
\usepackage{xeCJK,fontspec,xunicode,xltxtra, fancybox}

%\usepackage[timeinterval=1]{tdclock}
%box tools
\usepackage{framed, color}
\usepackage{colortbl, booktabs, multirow, makecell, longtable}

\usepackage{soul} %for strikeout

\usepackage{amsmath, amsfonts, amssymb} %amssymb for varnothing symbol

\usepackage{caption, algorithm}
\usepackage[noend]{algpseudocode}

\usepackage{textpos}
\usepackage{tikz, flowchart} % tikz绘图
\usetikzlibrary{decorations.pathreplacing}
\usetikzlibrary{decorations.markings}
\usetikzlibrary{calc, arrows, arrows.meta, shapes, shapes.geometric, positioning}
\usetikzlibrary{topaths}
\usetikzlibrary{mindmap}
\usetikzlibrary{matrix}
\pgfmathsetmacro{\myinnersep}{2}% inner sep in mm
\tikzset{
%    box/.style={draw,%
%        inner sep=\myinnersep,%
%        outer sep=0,%
%    minimum width=5mm,%
%    minimum height=\heightof{Cap}+2*\myinnersep*1mm,%
%    align=center},
    database/.style={
      cylinder,
      cylinder uses custom fill,
      %cylinder body fill=green!5,
      %cylinder end fill=green!5,
      shape border rotate=90,
      aspect=0.25,
      inner sep=0.2cm,
      draw
    },
  multidocument/.style={
    shape=tape,
    draw,
    fill=white,
    tape bend top=none,
    double copy shadow},
  manual input/.style={
    shape=trapezium,
    draw,
    shape border rotate=90,
    trapezium left angle=90,
    trapezium right angle=80}
}


\makeatletter
\pgfdeclareshape{document}{
	\inheritsavedanchors[from=rectangle] % this is nearly a rectangle
	\inheritanchorborder[from=rectangle]
	\inheritanchor[from=rectangle]{center}
	\inheritanchor[from=rectangle]{north}
	\inheritanchor[from=rectangle]{south}
	\inheritanchor[from=rectangle]{west}
	\inheritanchor[from=rectangle]{east}
	% ... and possibly more
	\backgroundpath{% this is new
		% store lower right in xa/ya and upper right in xb/yb
		\southwest \pgf@xa=\pgf@x \pgf@ya=\pgf@y
		\northeast \pgf@xb=\pgf@x \pgf@yb=\pgf@y
		% compute corner of ‘‘flipped page’’
		\pgf@xc=\pgf@xb \advance\pgf@xc by-10pt % this should be a parameter
		\pgf@yc=\pgf@yb \advance\pgf@yc by-10pt
		% construct main path
		\pgfpathmoveto{\pgfpoint{\pgf@xa}{\pgf@ya}}
		\pgfpathlineto{\pgfpoint{\pgf@xa}{\pgf@yb}}
		\pgfpathlineto{\pgfpoint{\pgf@xc}{\pgf@yb}}
		\pgfpathlineto{\pgfpoint{\pgf@xb}{\pgf@yc}}
		\pgfpathlineto{\pgfpoint{\pgf@xb}{\pgf@ya}}
		\pgfpathclose
		% add little corner
		\pgfpathmoveto{\pgfpoint{\pgf@xc}{\pgf@yb}}
		\pgfpathlineto{\pgfpoint{\pgf@xc}{\pgf@yc}}
		\pgfpathlineto{\pgfpoint{\pgf@xb}{\pgf@yc}}
		\pgfpathlineto{\pgfpoint{\pgf@xc}{\pgf@yc}}
	}
}
\makeatother

\newcommand{\Simley}[3]{%
\begin{tikzpicture}[scale=0.11]
    \newcommand*{\SmileyRadius}{#2}%
    \draw [fill=brown!10] (0,0) circle (\SmileyRadius)% outside circle
        %node [yshift=-0.22*\SmileyRadius cm] {\tiny #1}% uncomment this to see the smile factor
        ;  

    \pgfmathsetmacro{\eyeX}{0.5*\SmileyRadius*cos(30)}
    \pgfmathsetmacro{\eyeY}{0.5*\SmileyRadius*sin(30)}
    \draw [fill=cyan,draw=none] (\eyeX,\eyeY) circle (#3 cm);
    \draw [fill=cyan,draw=none] (-\eyeX,\eyeY) circle (#3 cm);

    \pgfmathsetmacro{\xScale}{2*\eyeX/180}
    \pgfmathsetmacro{\yScale}{1.0*\eyeY}
    \draw[color=red, domain=-\eyeX:\eyeX]   
        plot ({\x},{
            -0.1+#1*0.15 % shift the smiley as smile decreases
            -#1*1.75*\yScale*(sin((\x+\eyeX)/\xScale))-\eyeY});
\end{tikzpicture}%
}%
%  \Simley{1}{10}{1}
%  \Simley{0.5}{10}{1}
%  \Simley{0}{10}{1}
%  \Simley{-0.5}{10}{1}
%  \Simley{-1}{10}{1}


\usepackage{pgfplots}
\usepgfplotslibrary{external}  %缓存tikz的结果
\tikzset{
    external/system call={%
    xelatex \tikzexternalcheckshellescape
    -halt-on-error -interaction=batchmode --shell-escape
    -jobname "\image" "\texsource"}}
%\tikzexternalize


\usepackage[tikz]{bclogo}  %see http://mirrors.ctan.org/graphics/bclogo/README
\DeclareGraphicsRule{.mps}{eps}{*}{} %解决xelatex处理bclogo时的mps问题
\newcommand{\infobox}[2]{ 
    \begin{bclogo}[couleur=yellow!10, logo=\bcfleur, ombre=true]{#1}
    #2
    \end{bclogo}
}
\newcommand{\warnbox}[2]{ 
    \begin{bclogo}[couleur=yellow!10, logo=\bctakecare, ombre=true]{#1}
    #2
    \end{bclogo}
}
\newcommand{\kpt}{{\color{red}$\ast$}}
\newcommand{\h}{{\color{orange!70}$\bullet ~ $}}
\newcommand{\hh}{{\color{orange!70}$~~ \ast ~ $}}

\newcommand{\newsec}{\color{red} $\Rightarrow$}

%\usetheme[height=8mm]{Madrid} % Favorite theme is Madrid! 
%\usecolortheme[RGB={130,120,232}]{structure} 
%\usetheme{Boadilla} 
%\usecolortheme{default} 

\usetheme{Madrid} 
\usecolortheme{crane} 

\setbeamertemplate{items}[ball] 
\setbeamertemplate{blocks}[rounded][shadow=true] 

\definecolor{red(ncs)}{rgb}{0.77, 0.01, 0.2}
\definecolor{champagne}{rgb}{0.97, 0.91, 0.81}
\definecolor{coolblack}{rgb}{0.0, 0.18, 0.39}
\definecolor{vanilla}{rgb}{0.95, 0.9, 0.67}

\usepackage[ampersand]{easylist}
\newcommand\easyitem{\ListProperties(Hide=100, Hang=true, Progressive=3ex,
  Style*=\color{orange!70}$\bullet$ ,
  Style2*=\color{orange!70}$\ast$ ,
  Style3*=\color{orange!70}$\circ$ ,
  Style4*=\tiny$\blacksquare$, Space=-.5em, Space*=-.5em)}

\parskip=3mm
\parindent=15pt
\linespread{1.1}


%自定义的一些命令,方便使用
\newcommand*\circled[1]{\tikz[baseline=(char.base)]{
  \node[shape=circle,draw,inner sep=1.5pt] (char) {#1};}}


% \newcommand*{\hei}{\fontfamily{FZLanTingHeiS-H-GB}\selectfont}
%\DeclareTextFontCommand{\texthei}{\hei}

\setCJKfamilyfont{FZHei}{FZLanTingHeiS-EL-GB}  
\newcommand{\cjkbold}{\color[rgb]{0.29, 0.0, 0.51} \CJKfamily{FZHei}}  %http://latexcolor.com/

\setCJKmainfont{FZLanTingHeiS-EL-GB} %方正字体,也可以改成:微软雅黑
\setCJKsansfont{FZLanTingHeiS-EL-GB} 
\setCJKmonofont{Consolas}


% % 设置英文字体%==================================================
%\defaultfontfeatures{Scale=MatchLowercase} % 这个参数保证 serif、sans-serif 和
% %monospace 字体在小写时大小匹配
%\setmainfont[Mapping=tex-text]{Ubuntu} % 使用 XeTeX 的 text-mapping 方案,正确显示 LaTeX 样式的双引号(`` '')%

%\setsansfont[Mapping=tex-text]{Consolas}
%\setmonofont{Ubuntu Light}

% FONTS
\defaultfontfeatures{Mapping=tex-text}
\setsansfont[  %\setmainfont[
SmallCapsFont = Fontin-SmallCaps.otf,
BoldFont = Fontin-Bold.otf,
ItalicFont = Fontin-Italic.otf
]
{Fontin.otf}



\newcommand{\cjkem}{\CJKfamily{FZHeiR}} 
\renewcommand{\em}[1]{\color{red} #1}

\XeTeXlinebreaklocale "zh"  
\XeTeXlinebreakskip = 0pt plus 1pt 

\usepackage{listings,lstautogobble}
\input{lststyle}

\lstnewenvironment{python}{\renewcommand\lstlistingname{Python} %
\lstset{language=python,basicstyle=\rmfamily\normalsize, prebreak=, %
  autogobble=true, keywordstyle=\ttfamily, %
  escapechar=|,
  numberstyle=\tiny\color{orange!90}\ttfamily, showstringspaces=false} } %
{}

\lstnewenvironment{bash}{\renewcommand\lstlistingname{Bash} %
\lstset{language=bash,basicstyle=\rmfamily\normalsize, prebreak=, %
  autogobble=true, keywordstyle=\ttfamily, numberstyle=\tiny\color{green!70}\ttfamily}} %
{}

\newcommand{\pyinline}[1]{\small{\color{blue!50!green}$>>>$}
  \lstinline[language=python, basicstyle=\rmfamily\normalsize, keywordstyle={\ttfamily\color{blue!50!black}}]!#1!}

\hypersetup{
  pdftitle={Python程序设计},
  pdfsubject={Python},
  pdfkeywords={Python},
  pdfproducer={LaTeX},
  pdfcreator={XeLaTeX}
}


%\setbeamercolor{title}{bg=red(ncs), fg=white}
%\setbeamercolor{frametitle}{fg=red(ncs),bg=gray!10!white}
\setbeamercolor{title}{fg=coolblack, bg=vanilla}
\setbeamercolor{frametitle}{fg=coolblack, bg=vanilla!0}

\setbeamercolor{palette primary}{fg=black, bg=gray!15!white}
\setbeamercolor{palette secondary}{fg=black, bg=gray!10!white}
\setbeamercolor{palette tertiary}{fg=black, bg=gray!15!white}


\addtobeamertemplate{frametitle}{}{%
\begin{textblock*}{1.0\paperwidth}(-.001\textwidth,0cm)
%\tikz{\draw[orange!95, line width=1.1] (-2cm,0cm) -- (0.5\textwidth,0cm);\draw[orange!95,yshift=-0.55] (0.5\textwidth,0cm) -- (0.7\textwidth,0cm);}
\end{textblock*}
\begin{textblock*}{100mm}(.88\textwidth,-1cm)
\includegraphics[height=1.2cm,width=1.2cm]{ruc_logo.png}
\end{textblock*}
}

%gets rid of bottom navigation bars
\setbeamertemplate{footline}[page number]{}

%gets rid of navigation symbols
\setbeamertemplate{navigation symbols}{}

\setbeamercolor{item projected}{bg=orange!70!orange,fg=white}
\setbeamertemplate{enumerate items}[circle]

\begin{document}

%\logo{\includegraphics[width=1.0cm,height=1.0cm]{figure/ruc.jpg}}
\title{Python程序设计}
\author{Xia Tian \\ Email: xiat(at)ruc.edu.cn}
\institute{Renmin University of China}
\date{\today{}}
%\date[\initclock\tdtime]{\today}  
\frame{\titlepage}

\chapter{引言}

\section{什么是Python}

如果你打算让计算机做更多的事情,而不仅仅局限于使用他人已经编好的可运行程序,例如,
以某种你希望的方式对照片文件重新命名,或者信手写一个数独游戏以消磨时间,甚至是在
科学兴趣探索中,进行一些复杂的数字运算,那么Python将是一种非常适合你需求的高级编
程语言,其语法简单,安装测试方便,功能又异常强大,并拥有高效率的高层数据结构和面
向对象特征。付出时间成本投资于Python语言,你一定会取得满意的回报。

Python语言是一种少有的既简单又强大的高级编程语言,注重问题的解决而非编程语言的语
法和结构,正如其官方所介绍:

\begin{enumerate}
\item Python is a programming language that lets you work more quickly and
integrate your systems more effectively.

(Python是一种能够让你工作更便捷、系统集成更高效的编程语言)

\item Python is a clear and powerful object-oriented programming language,
comparable to Perl, Ruby, Scheme, or Java.

(相比于Perl、Ruby、Scheme或Java,Python是一种简单、强大的面向对象的编程语言)
\end{enumerate}

顺便说一句,Python语言的作者Guido van Rossum是根据英国广播公司的节目``Monty
Python's Flying Circus''(蟒蛇飞行马戏)给这个语言命名的,并非他本人特别喜欢蛇缠起它
们的长身躯碾死动物觅食。

目前,Python正处于有2.x版本到3.x版本过渡的过程之中,Python 3又称为Python
3000或Py3K,作为新一代Python语言,Python 3以上版本去除了之前版本中积累的一些老问
题,并使语言更为清晰,但另一方面,由于Python2.x是如此的成功,在大量系统之中得到了
广泛应用,出于移植成本考虑,短时间之内还无法将所有Python代码都升级为Python 3版本,
因此,Python 2和Python 3还将长时间并存\footnote{Python 2.7之后,Python 2.x版本将
  不会再有重大更新,新项目建议使用Python 3.x,到底选择Python2还是Python3,可阅
  读:\url{https://wiki.python.org/moin/Python2orPython3} },考虑到未来发展,本教
程将采用Python 3的语言规范进行讲解\footnote{如果你已有大量的 Python 2.x 代码,也
  可以使用工具将 2.x 转换成 3.x 的源代码,参
  考\url{http://docs.python.org/3.5/library/2to3.html}}。



\section{Python的安装}
如果你已经安装了 Python 2.x ,那么就没必要卸载后再安装 Python 3.x ,实际上,可以将它
们同时安装在电脑里。

\subsection{Linux用户}
如果你正在使用一个Linux的发行版,比如Ubuntu或Fedora,默认情况下,系统里面已经安
装了Python,要测试系统中是否已经安装了Python以及Python的版本,可以打开一个shell
程序,输入如下命令:
\shellbox{\$ python -V

Python 2.7.10
}

注意,参数V为大写形式,如果写成了小写的v,则代表以详细追踪方式显示Python的启动时
的执行内容,而不是显示Python的版本。

2016年以来发布的Linux版本,通常在装有python 2的同时,也安装有python3,以Ubuntu为
例,在命令行上运行python3 -V,可以运行python3,并得到其版本信息,类似于如下:

\shellbox{
\$ python3 -V

Python 3.4.3+ (default, Oct 14 2015, 16:03:50) 
}

\subsection{Mac用户}

Mac OS X 用户 会 发 现已 经 在 系统中 安 装了 Python 。打开 Terminal.app 运行
python -V ,接着参考上面关于 Linux 部分的建议。

\subsection{Windows用户}

访问 http://www.python.org 网站下载最新版,在写本书的时候是 3.0 beta 1 。仅
有 12.8MB ,与大多其它的语言或软件相比,是非常紧凑的。安装与其它的 Win-
dows 软件一样。

有趣的是,大多的 Python 下载是来自 Windows 用户的。当然,这并不能说明问
题,因为几乎所有的 Linux 用户已经在安装系统的时候默认安装了 Python 。


\section{如何运行Python程序}
我们将看一下如何用 Python 编写运行一个传统的“Hello World”程序。通过它,你将学会如
何编写、保存和运行 Python 程序。

有两种使用 Python 运行你的程序的方式 —— 使用交互式的带提示符的解释器或使用源文件。
我们将学习这两种方法。

\subsection{使用带提示符的解释器}
在 shell 提示符下,键入python命令启动解释器。对 Windows 用户,如果你已经配置好
了PATH变量,那么就可在命令行中启动解释器。

如果使用 IDLE ,点击 开始 $\rightarrow$ 程序 $\rightarrow$ Python 3.0
$\rightarrow$ IDLE (Python GUI)。键入 print('Hello World') ,按回车键。将会看
到 Hello World 字样的输出。

注意, Python会在下一行立即给出你输出!对于刚键入的Python语句print('Hello World'),
我们使用print(不要惊讶,这里的打印默认是``打印''到显示器上,即在显示器上输出结果)来
打印你提供给它的值,这里提供的文本是``Hello World'' ,它被立即打印在屏幕上。

至于如何退出解释器提示符,如果你使用的是Linux/BSD shell,那么按``Ctrl-d''退出提示
符,如果是在Windows命令行中,则按``Ctrl-z'',再按回车键退出。

\subsection{运行Python源代码文件}




\section{工作环境的配置}

工欲善其事必先利其器,搭建一个好的工作环境,对于提高生产效率是十分重要的。下面,
我们从源代码编辑器和增强命令行交互工具两个方面对Python工作环境进行介绍。

\subsection{源代码编辑器的选择}

支持Python的编辑器有很多,在一个由多个工具组成的环境中,最好只关注与Python代码编
写相关的编辑器。实际上,简单的代码编辑器和集成开发环境之间的边界并不明显,一些简
单的编辑器也会提供与系统交互和扩展的机制。对于代码编写来说,一个配置齐全的源代码
编辑器可以降低一些重复性工作,有效辅助代码的编写。

多年以来,Python源代码编辑器的最佳选择是vim或Emacs,初次接触这两个编辑器会觉得其
操作并不友好,甚至感觉完全无法使用,这是因为这两个编辑器都引入了大量键盘快捷键,
只有经过一段时间的熟悉之后,才会体会到它们的强大之处,如果你觉得自己还年轻,那么
投入精力去掌握它们,你的付出会让你终生受用。

Vim在多数Linux的发行版和苹果的笔记本操作系统中,默认都已经安装\footnote{部分系统
  默认安装的是vi,而vim是vi的升级版本,兼容vi的所有指令,并拥有大量新特性,如果系
  统默认安装的是vi,建议读者进一步安装vim。},可以直接使用,而Emacs通常需要单独安
装。相比于Vim或Emacs,新潮的开发人员或初学者,可能更喜欢以鼠标操作为主的编辑器,
其中,Sublime Text(http://www.sublimetext.com/)是目前非常流行的编辑器,对Python
的语法支持也很好。

综上,初学者可以先需用Sublime Text作为Python的源代码编辑器,并逐步学习使
用Vim或Emacs,提高熟练程度。

\subsection{扩展Python shell}

\subsubsection{ipython}

\subsubsection{bpython}
bpython是一个不错的Python解释器界面,它将集成开发环境的常见语法提示等常见功能封
装在在一个简单、轻量级的包里,可以在终端窗口里面直接运行,保持python shell的简洁、
快速和实用特点。

bpython可以利用pip快速安装:

\shellbox{
\$pip install bpython
}

bpython的功能十分丰富,包括:

\begin{itemize}
\item 内置的语法高亮 – 使用Pygments排版你敲出的代码,并合理配色
\item 根据你的行为,显示自动补全的建议
\item 为任何Python函数列出所期望的参数 – 可以显示你调用的任何函数的参数列表
\item “Rewind”功能会调出内存里的最后一行代码并重新执行
\item 可以将你输入的代码送到pastebin
\item 可以将你输入的代码保存到一个文件
\item 自动缩进
\item 支持Python 3
\end{itemize}

%%% Local Variables:
%%% mode: latex
%%% TeX-master: "../python"
%%% End:

\section{Python数据类型}

\begin{frame}[fragile]{CH2 数据类型}
  \begin{easylist} \easyitem
    & 2.1 标识符与关键字
    & 2.2 基本数据类型
    && 2.2.1 整数
    && 2.2.2 布尔
    && 2.2.3 浮点
    && 2.2.4 字符
    & 2.3 组合数据类型
    && 2.3.1 序列类型:元组与列表
    && 2.3.2 集合
    && 2.3.3 字典(映射)
  \end{easylist}
\end{frame}

\subsection{2.1 标识符与关键字}

\begin{frame}[fragile]{2.1 标识符与关键字}
  \begin{easylist}
    & 标识符:任意长度的非空字符序列
    && 要求
    &&& 引导字符:Unicode编码字母、ASCII字符、\_,不能是数字!
    &&& 不能是关键字
  \end{easylist}

  \begin{table}
    \begin{center}
      \begin{tabular}{|l | l | l |  l| l|}
        \toprule \hline
        and & as & assert & break & class \\ \hline
        continue & def & del & eif & else \\ \hline
        except & False & finally & for & from \\ \hline
        global & if & import & in & is \\ \hline
        lambda & None & nonlocal & not & or \\ \hline
        pass & raise & return & True & try \\ \hline
        while & with & yield & & \\ \hline
        \bottomrule
      \end{tabular}
    \end{center}
    \caption{Python关键字列表}
  \end{table}
\end{frame}

\begin{frame}[fragile]{标识符命名约定}
  \begin{easylist}
    & 尽量不要使用Python内置函数名与异常名, 如int, float...
    & 避免使用名称的开始和结尾都是下划线
    && 在Python解释器中,输入如下语句,观察输出:

    $>>>$ max.\_\_doc\_\_

    注意:前后各是两个下划线"\_"
  \end{easylist}
\end{frame}

\begin{frame}[fragile]{测试}
  $>>>$ hello-world = 1 \\ Error!

  $>>>$ 5miles = 5 \\ Error!

  $>>>$ int = 5 \\ 正常运行,但不建议

  $>>>$ hello\_world = 'Hello world!' \\ OK
\end{frame}


\subsection{2.2 基本数据类型}
\begin{frame}[fragile]{2.2 基本数据类型}
  \begin{easylist}
    &  基本数据类型
    && 整数
    && 布尔
    && 浮点
    && 字符
  \end{easylist}
\end{frame}


\subsubsection{2.2.1 整数}
\begin{frame}[fragile]{2.2.1 整数}
  \begin{easylist}
    & 默认为10进制形式 \\
    $>>>$ 12345

    & 二进制形式:0bxxxx \\
    $>>>$ 0b1111 (15)

    & 八进制形式:0oxxxx \\
    $>>>$ 0o10 (8)

    & 十六进制形式:0xABCD \\
    $>>>$ 0xFF (255)

    & 前导字符大小写均可以,如0XFF
  \end{easylist}
\end{frame}


\begin{frame}[fragile]{数值型操作符与函数}
  \begin{center}
    \begin{tabular}{l | l}
      \toprule
      语法 & 描述 \\ 
      \midrule
      $x // y$ & 整除 \\ \hline
      $x \% y $ & 取模 \\ \hline
      $x ** y$ & x的y次幂 \\ \hline
      abs(x) & 取绝对值 \\ \hline
      pow(x, y) & x的y次幂 \\ \hline
      round(x, n) & 四舍五入,n指小数位保留几位 \\
      \bottomrule
    \end{tabular}
  \end{center}
\end{frame}

\begin{frame}[fragile]{整数转换函数}
  \begin{center}
    \begin{tabular}{l | l}
      \toprule
      语法 & 描述 \\
      \midrule
      bin(i) & 返回整数i的二进制形式 \\ \hline
      hex(i) & 返回整数i的十六进制形式 \\ \hline
      int(x) & 将x转换为整数 \\ \hline
      oct(i) & 返回i的八进制形式 \\ 
      \bottomrule
    \end{tabular}
  \end{center}
\end{frame}

\begin{frame}[fragile]{整数位移操作符}
  \begin{center}
    \begin{tabular}{l | l}
      \toprule
      语法 & 描述 \\ 
      \midrule
      $i | j$  & 逻辑OR运算 \\ \hline
      $i \mathbin{\char`\^} j$ & XOR \\ \hline
      $i \& j$ & AND \\ \hline
      $i<<j$ & i左移j位,类似于$i*(2**j)$,但不带溢出检查 \\ \hline 
      $i>>j$ & i右移j位,类似于$i//(2**j)$,但不带溢出检查 \\ \hline
      $\textasciitilde i$ & 反转i的每一位 \\
      \bottomrule
    \end{tabular}
  \end{center}
\end{frame}

\begin{frame}[fragile]{位逻辑操作练习}
  $>>>$ i = 0b1010 \\
  $>>>$ j = 0b11000 \\

  $>>>$ $print(i,j,i|j, i\&j, i>>2, i<<2, \textasciitilde i)$

  手工计算一下结果是多少?并利用Python解释器进行验证,注意,可以利用bin()函数,
  查看结果的二进制形式。

  \pause

  输出结果:10 24 26 8 2 40 -11

  \pause

  \begin{easylist}
    & \small 计算机内部通常采用补码表示整数,补码对于正数就是本身,负数补码等于源
    码的反码加一,正数取反后在数值上体现为$-|x+1|$,负数的是$|x|-1$。

    & \small 如果要表示多个布尔值,可以利用一个整数进行表示,如文件的属性有``读、写和执
    行''三种,可以用一个三位的二进制数字表示。
  \end{easylist}
\end{frame}


\begin{frame}[fragile, allowframebreaks]{原码 $\rightarrow$ 反码 $\rightarrow$ 补码}
  \h 无符号数字:只有正数,没有负数的概念。

  \begin{center}
    \begin{tabular}{| c | c |}
      \hline
      ~~~十进制~~~ & ~~~~~~~二进制~~~~~~~ \\ \hline
      0 & 0000 \\ \hline
      1 & 0001 \\ \hline
      2 & 0010 \\ \hline
      3 & 0011 \\ \hline
      4 & 0100 \\ \hline
      $\vdots$ & $\vdots$ \\ \hline
    \end{tabular}
  \end{center}

    \newpage
    ~\\
    \begin{easylist}
      & 神说,要有光,就有了光,要有正数和负数,就有了正数和负数
      && 为表示正数和负数,人们发明了``原码'':
      &&& 左边第一位存放符号,正用0来表示,负用1来表示
    \end{easylist}

    \begin{center}
      \begin{columns}[totalwidth=0.5\textwidth,t]
        \begin{column}{.3\textwidth}
          \centering
          \begin{tabular}{| c | c |}
            \hline
            ~ & 正数 \\ \hline
            0 & 0000 \\ \hline
            1 & 0001 \\ \hline
            2 & 0010 \\ \hline
            3 & 0011 \\ \hline
            4 & 0100 \\ \hline
            5 & 0101 \\ \hline
            6 & 0110 \\ \hline
            7 & 0111 \\ \hline
          \end{tabular}
        \end{column}
        
        \begin{column}{.3\textwidth}
          \centering
          \begin{tabular}{| c | c |}
            \hline
            ~ & 负数 \\ \hline
            -0 & 1000 \\ \hline
            -1 & 1001 \\ \hline
            -2 & 1010 \\ \hline
            -3 & 1011 \\ \hline
            -4 & 1100 \\ \hline
            -5 & 1101 \\ \hline
            -6 & 1110 \\ \hline
            -7 & 1111 \\ \hline
          \end{tabular}
        \end{column}
      \end{columns}
    \end{center}

    \newpage
    ~ \\
    \begin{easylist}
      & 易于理解,但不利于计算机处理
      && $1+(-1) = 0001_b + 1001_b = 1010_b = -2=$
      && 存在正0($0000_b$)和负0($1000_b$)
      && 正负相加不等于0
    \end{easylist}

    \newpage
    ~\\
    \begin{easylist}
      & 反码
      && 用来处理负数的,符号位置不变,其余位置相反
    \end{easylist}

    \begin{center}
      \begin{columns}[totalwidth=0.5\textwidth,t]
        \begin{column}{.3\textwidth}
         \centering
          正数:\\
          \begin{tabular}{| c | c |}
            \hline
            ~ & 正数 \\ \hline
            0 & 0000 \\ \hline
            1 & 0001 \\ \hline
            2 & 0010 \\ \hline
            3 & 0011 \\ \hline
            4 & 0100 \\ \hline
            5 & 0101 \\ \hline
            6 & 0110 \\ \hline
            7 & 0111 \\ \hline
          \end{tabular}
        \end{column}
        
        \begin{column}{.3\textwidth}
          \centering
          \color{red} 反码:\\
          \begin{tabular}{| c | c |}
            \hline
            ~ & 负数 \\ \hline
            -0 & 1111 \\ \hline
            -1 & 1110 \\ \hline
            -2 & 1101 \\ \hline
            -3 & 1100 \\ \hline
            -4 & 1011 \\ \hline
            -5 & 1010 \\ \hline
            -6 & 1001 \\ \hline
            -7 & 1000 \\ \hline
          \end{tabular}
        \end{column}

        \begin{column}{.3\textwidth}
          \centering
          原码:\\
          \begin{tabular}{| c | c |}
            \hline
            ~ & 负数 \\ \hline
            -0 & 1000 \\ \hline
            -1 & 1001 \\ \hline
            -2 & 1010 \\ \hline
            -3 & 1011 \\ \hline
            -4 & 1100 \\ \hline
            -5 & 1101 \\ \hline
            -6 & 1110 \\ \hline
            -7 & 1111 \\ \hline
          \end{tabular}
        \end{column}
      \end{columns}
    \end{center}
    
    \newpage
    ~\\
    \begin{easylist}
      & 原码变成反码后,解决了正负相加等于$0$的问题
      & 过去的$+1$和$-1$相加,变成了$0001_b+1101_b=1111_b$,刚好反码表示方式中,
      $1111_b$象征$-0$
      & 新问题:存在两个零,$+0$和$-0$
      & 从原来"反码"的基础上,补充一个新的代码,负数加1,因此,$-0$由$1111_b$变
      为$10000_b$,舍去溢出的高位,变为$0000_b$
    \end{easylist}
    
    \newpage
    ~\\
    \begin{center}
      \begin{columns}[t]
        \begin{column}{.25\textwidth}
         \centering
          正数:\\
          \begin{tabular}{| c | c |}
            \hline
            ~ & 正数 \\ \hline
            0 & 0000 \\ \hline
            1 & 0001 \\ \hline
            2 & 0010 \\ \hline
            3 & 0011 \\ \hline
            4 & 0100 \\ \hline
            5 & 0101 \\ \hline
            6 & 0110 \\ \hline
            7 & 0111 \\ \hline
          \end{tabular}
        \end{column}
        
        \begin{column}{.25\textwidth}
          \centering
          \color{blue} 补码:\\
          \begin{tabular}{| c | c |}
            \hline
            ~ & 负数 \\ \hline
            -0 & 0000 \\ \hline
            -1 & 1111 \\ \hline
            -2 & 1110 \\ \hline
            -3 & 1101 \\ \hline
            -4 & 1100 \\ \hline
            -5 & 1011 \\ \hline
            -6 & 1010 \\ \hline
            -7 & 1001 \\ \hline
            -8 & 1000 \\ \hline
          \end{tabular}
        \end{column}

        \begin{column}{.25\textwidth}
          \centering
          \color{red} 反码:\\
          \begin{tabular}{| c | c |}
            \hline
            ~ & 负数 \\ \hline
            -0 & 1111 \\ \hline
            -1 & 1110 \\ \hline
            -2 & 1101 \\ \hline
            -3 & 1100 \\ \hline
            -4 & 1011 \\ \hline
            -5 & 1010 \\ \hline
            -6 & 1001 \\ \hline
            -7 & 1000 \\ \hline
          \end{tabular}
        \end{column}

        \begin{column}{.25\textwidth}
          \centering
          原码:\\
          \begin{tabular}{| c | c |}
            \hline
            ~ & 负数 \\ \hline
            -0 & 1000 \\ \hline
            -1 & 1001 \\ \hline
            -2 & 1010 \\ \hline
            -3 & 1011 \\ \hline
            -4 & 1100 \\ \hline
            -5 & 1101 \\ \hline
            -6 & 1110 \\ \hline
            -7 & 1111 \\ \hline
          \end{tabular}
        \end{column}
      \end{columns}
    \end{center}
    
    \begin{easylist}
      & 解决了$+0$和$-0$同时存在的问题
      & 另外"正负数相加等于0"的问题
      & 多了一个$-8$
    \end{easylist}

\end{frame}




\subsubsection{2.2.2 布尔类型}
\begin{frame}[fragile]{2.2.2 布尔类型}
  ~ \\
  \begin{columns}
    \begin{column}{.3\textwidth}
      基本用法:

      \pyinline{t=True} 

      \pyinline{f = False} 

      \pyinline{t and f} \\
      False 

      \pyinline{t and True} \\
      True 
    \end{column}
    \pause
    \begin{column}{.6\textwidth}
      布尔类型可以看作是一种特殊的整数类型,False为0,True为1,
      0为False, 非0值为True,例如:

      \vspace{0.5cm}
      \pyinline{False + 1} \\ $1$

      \pyinline{True - 2} \\ $-1$
    \end{column}
  \end{columns}
\end{frame}


\subsubsection{2.2.3 浮点类型}
\begin{frame}[fragile]{2.2.3 浮点类型}
  \textbf{float}:  计算机以二进制表示浮点数,是近似表示
  
  \pyinline{0.0, 5.4, 1e-2} \\
  (0.0, 5.4, 0.01)
  

  \pyinline{x = 5.59} \\
  \pyinline{int(x)} \\5 \\    
  \pyinline{s=x.hex()}\\'0x1.65c28f5c28f5cp+2', 此处指数以p表示,而不是e,why? \\
  \pyinline{y=float.fromhex(s)}
\end{frame}

\begin{frame}[fragile]{其他数字类型及操作}
  \begin{easylist}
    & 复数 \\
    \pyinline{z = 1 + 2j}
    & 扩展包
    && math \\
    ~\pyinline{import math} \\
    ~\pyinline{math.pi} \\
    ~\pyinline{math.______}
    && decimal:精度可控 

    \begin{python}
      import decimal
      a = decimal.Decimal(1234)
      b = decimal.Decimal("1234567890000.1234567")
      a + b
    \end{python}
    Result: Decimal('1234567891234.1234567')
  \end{easylist}
\end{frame}

\subsection{2.2.4 字符串}
\begin{frame}[fragile]{2.2.4 字符串}
  \begin{easylist}
    & 可以使用单引号或双引号,但两端必须相同
    & 三个引号包含的字符串
    && 可以直接在里面使用换行,而不需要进行转义处理
    & 转义符
    && 用于处理特殊字符,如回车、换行、引号等
    &&& $\backslash r$, $\backslash t$, $\backslash n$, $\backslash '$,
    $\backslash "$, $\cdots$
    &&& $\backslash xhh$: 对应十六进制数字的字符
    && 例子 \\
    ~\pyinline{s = 'hello\\x20world\\n\\tI\\'m fine.'} \\
    ~\pyinline{print(s)} \\
    \pause
    输出结果:\\
    hello world\\
    ~~~~I'm fine.
  \end{easylist}
\end{frame}

\begin{frame}[fragile]{常见函数}
  \begin{easylist}
    & Python通过ord和chr两个内置函数,用于字符与ASCII码之间的转换
    && ord()函数 \\
    求特定字符的Unicode编码,以十进制表示返回结果 
    && chr()函数\\
    返回整数所对应的字符 
  \end{easylist}

  \pyinline{ord('A')}  \\65 \\
  \pyinline{hex(ord('A'))} \\'0x41' \\
  \pyinline{hex(ord('中'))} \\'0x4e2d' \\
  \pyinline{chr(0x4e2d)} \\ '中'
\end{frame}

\begin{frame}[fragile]{字符串比较}
  内部安装字符串的ASCII码值进行比较

  \pyinline{"RUC" > "IRM"} \\ True

  \pyinline{"China" > "Canda"} \\True
\end{frame}

\begin{frame}[fragile]{字符串分片与步距}
  \begin{center}
    \begin{tikzpicture}[c/.style={minimum width=1.2cm, minimum height=1cm, draw}]
      \draw node[c] (1) {W} node[c, right=0 of 1] (2) {h} node[c, right=0 of 2] (3) {o} node[c, right=0 of 3] (4) { } node[c, right=0 of 4] (5) {a}  node[c, right=0 of 5] (6) {m}  node[c, right=0 of 6] (7) { }  node[c, right=0 of 7] (8) {I}  node[c, right=0 of 8] (9) {?};
      \draw node[above=0 of 1] {s[-9]} node[above=0 of 2] {s[-8]} node[above=0 of 3] {s[-7]} node[above=0 of 4] {s[-6]} node[above=0 of 5] {s[-5]} node[above=0 of 6] {s[-4]} node[above=0 of 7] {s[-3]} node[above=0 of 8] {s[-2]} node[above=0 of 9] {s[-1]};
      \draw node[below=0 of 1] {s[0]} node[below=0 of 2] {s[1]} node[below=0 of 3] {s[2]} node[below=0 of 4] {s[3]} node[below=0 of 5] {s[4]} node[below=0 of 6] {s[5]} node[below=0 of 7] {s[6]} node[below=0 of 8] {s[7]} node[below=0 of 9] {s[8]};
    \end{tikzpicture}
  \end{center}

  \begin{easylist}
    & 语法
    && s[start]
    && s[start:end]
    && s[start:end:step]
  \end{easylist}

  \pyinline{s = 'Who am I?'} \\
  \pyinline{s[0:6:2], s[0:3]} \\
  'Woa', 'Who'
\end{frame}


\begin{frame}[fragile]{字符串步距}
  步距默认为1,但可以为负

  \pyinline{s[-1:]} \\ '?'

  \pyinline{s[-1::-1]} \\'?I ma ohW'
\end{frame}


\begin{frame}[fragile,allowframebreaks]{字符串操作方法}
  \begin{center}
    \scriptsize
    \begin{longtable}{p{0.3\textwidth} | p{0.6\textwidth}}
      \toprule
      语法 & 描述 \\ 
      \midrule
      s.capitalize() & 返回s的副本,并将首字母变为大小 \\ \hline
      s.center(width,char) & 返回s中间部分的一个子字符串,长度为width,并使用空格或可选的char(长度为1的字符串)进行填充。参考str.ljust()、str.rjust()与str.format() \\ \hline
      s.count(t,start,end) & 返回字符串s中(或在s的start:end分片中)子字符串t出现的次数 \\ \hline
      s.encode(encoding,err) & 返回一个bytes对象,该对象使用默认的编码格式或制定的编码格式来表示该字符串 \\ \hline
      s.endswith(x,start,end) & 如果s(或在s的start:end分片中)一字符串x(或元组中的任意字符串)结尾,就返回true,否则返回False,参考str.find() \\ \hline
      s.expandtabs(size) & 返回s的一个副本,其中的制表符使用8个货指定数量的空格替换 \\ \hline
      s.find(t,star,end) & 返回t在s中(或在s的start:end分片中)的最左位置,如果没有找到,就返回-1;使用str.rfind()则可以发现相应的最右边位置:参考str.index() \\ \hline
      s.formal(...) & 返回按给定参数进行格式化后的字符串副本,这一方法及其参数将在下一小节进行讲解 \\ \hline
      s.index(t,start,end) & 返回t在s中的最左边位置(或在s的start:end分片中),如果没有找到,就产生ValuError异常。使用sr.rindex()可以从右面开始搜索。参见str.find() \\ \hline
      s.isalnum() & 如果s非空,并且其中的每个字符串都是字母数字的,就返回True \\ \hline
      s.isalpha() & 如果s非空,并且其中每个字符都是字母的,就返回Ture \\ \hline
      s.isdecimal() & 如果s非空,并且其中的每个字符都是Uicode的基数为10的数字,就返回True, \\ \hline
      s.isdigit() & 如果s非空,并且每个字符都是一个ASCII数字,就返回Ture \\ \hline
      s.isidentifier() & 如果s非空,并且是一个邮箱的标识符,就返回Ture \\ \hline
      s.islower() & 如果s至少有一个可小写的字符,并且所有可小写的字符都是小写的,就返回True,参见str.isupper() \\ \hline
      s.isnumeric() & 如果s非空,并且其中的每个字符都是数值型的Uicode字符,比如数字或小数,就返回True \\ \hline
      s.isprintable() & 如果s非空,并且其中的每一个字符被认为死可打印的,包括空格,但不包括换行,就返回True \\ \hline
      s.isspace() & 如果s非空,并且其中的每一个字符都是空白字符,就返回Ture \\ \hline
      s.istitle() & 如果s是一个非空的首字母大写的字符串,就返回True,参见str.title() \\ \hline
      s.isupper() & 如果s至少有一个可大写的字符,并且所有可大写的字符都是大写的,就返回True,参见str.islower() \\ \hline
      s.join(seq) & 返回序列seq中每个项连接起来的后果,并以s(可以为空)在每两项之间分隔 \\ \hline
      s.ljust(width,char) & 返回长度为width的字符串(使用空格或可选的char(长度为1的字符串)进行填充)中左对齐的字符串s的一个副本,使用str.rjust()可以右对齐,str.center()可以中间对齐,参考str.format() \\ \hline
      s.lower() & 将s中的字符变为小写,参见str.upper() \\ \hline
      s.maketrans() & 与str.translate()类似,参见正文谅解详细资料 \\ \hline
      s.partition(t) & 返回包含3个字符串的元组————字符串s在t的最左边之前的部分、t、字符串s在t之后的部分。如果t不在s内,则返回s与两个空字符串。使用str.rpartition()可以在t最右边部分进行分区 \\ \hline
      s.replace(t,u,n) & 返回s的一个副本,其中每个(或最多n个,如果给定)字符串t使用u替换 \\ \hline
      s.split(t,n) & 返回一个字符串列表。要求在字符串t处至多分割n次,如果没有给定n,就分割尽可能多次,如果t没有给定,就在空白处分割。使用str.rsplit()可以从右边进行分割————只有在给定n并且n小于可能分割的最大次数时才能起作用 \\ \hline
      s.splitlines(f) & 返回在行终结符处进行分割产生的列表,并剥离行终结符(除非f为True) \\ \hline
      s.starswith(x,start,end) & 如果s(或s的start:end分片)以字符串x开始(或以元组x中的任意字符串开始),就返回True,否则返回False,参考str.endswith() \\ \hline
      s.strip(chars) & 返回s的一个副本,并将开始处与结尾处的空白字符(或字符串chars中的字符)移除,str.lstrip()仅剥离起始处的相应字符,str.rstrip()只剥离结尾处的相应字符 \\ \hline
      s.swapcase() & 返回s的辅副本,并将其中大写字符变为小写,小写字符变大写,参考str.lower()与str.upper() \\ \hline
      s.title() & 返回s的副本,并将每个单词的首字母变为大写,其他字母都变为小写,参考str.istitle() \\ \hline
      s.translate() & 与str.maketrans()类似,参考正文了解详细资料 \\ \hline
      s.upper() & 返回s的大写化版本,参考str.lower() \\ \hline
      s.zfill(w) & 返回s的副本,如果比w短,就在开始处添加0,使其长度为w \\ \hline
    \end{longtable}
  \end{center}
\end{frame}

\begin{frame}[fragile]{join函数}
  \pyinline{countries=["China", "USA", "Japan"]} 

  \pyinline{" ".join(countries)}  \\
  'China USA Japan' 

  \pyinline{" <-> ".join(countries)} \\
  'China <-> USA <-> Japan'
\end{frame}

\begin{frame}[fragile]{字符串复制 \emph{*}}
  \pyinline{s = 'ab' * 3} \\
  'ababab'

  \pyinline{s *= 2} \\
  'abababababab'
\end{frame}

\begin{frame}[fragile]{format()函数}
  \pyinline{'{0} is {1} years old.'.format("Tom", 5)} \\
  'Tom is 5 years old.'

\end{frame}

\subsection{2.3 组合数据类型}
\begin{frame}[fragile]{2.3 组合数据类型}
  \begin{easylist}
    & 列表
    & 元组
    & 字典
    & 集合
  \end{easylist}
\end{frame}



\begin{frame}[fragile]{列表}
  \h list是一种有序的集合,可以随时添加和删除其中的元素

  \pyinline{classmates = ['Michael', 'Bob', 'Tracy']} 

  \pyinline{classmates} \\
  $['Michael', 'Bob', 'Tracy']$

\end{frame}

\begin{frame}[fragile]{列表——索引}
  \h 用索引来访问list中每一个位置的元素,索引是从$0$开始的,最后一个元素的索引是$-1$

  \pyinline{classmates[0]} \\  
  'Michael'

  \pyinline{classmates[1]} \\  
  'Bob'  

  \pyinline{classmates[-1]} \\
  'Tracy'

  \pyinline{classmates[-2]} \\
  'Bob'
\end{frame}


\begin{frame}[fragile]{列表——追加}
    \h list是一个可变的有序表,可以往list中追加元素到末尾

    \pyinline{classmates.append('Adam')} 

    \pyinline{classmates} \\  
    $['Michael', 'Bob', 'Tracy', 'Adam']$
    

    \h 可以把元素插入到指定的位置,比如索引号为1的位置

    \pyinline{classmates.insert(1, 'Jack')} 

    \pyinline{classmates} \\ 
    $['Michael', 'Jack', 'Bob', 'Tracy', 'Adam']$
\end{frame}

\begin{frame}[fragile]{列表——删除}
  \h 用pop()方法删除list末尾的元素

  \pyinline{classmates.pop()} \\ 
  'Adam'

  \pyinline{classmates} \\ 
  $['Michael', 'Jack', 'Bob', 'Tracy']$


  \h 用pop(i)方法删除指定位置的元素

  \pyinline{classmates.pop(1)} \\ 
  'Jack'

  \pyinline{classmates} \\ 
  $['Michael', 'Bob', 'Tracy']$
\end{frame}

\begin{frame}[fragile]{列表——替换}
  \h 赋值给对应的索引位置把某个元素替换成别的元素

  \pyinline{classmates[1] = 'Sarah'} 

  \pyinline{classmates} \\ 
  $['Michael', 'Sarah', 'Tracy']$
\end{frame}


\begin{frame}[fragile]{列表——数据类型}
  \h list里面的元素的数据类型可以不同

  \pyinline{L = ['Apple', 123, True]} 


  \h list元素也可以是另一个list

  \pyinline{p = ['asp', 'php']} 

  \pyinline{s = ['python', 'java', p, 'scheme']} 

  \pyinline{s[2][1]} \\ 
  'php'
\end{frame}

\begin{frame}[fragile]{元组}
  \h 元组tuple是一种有序的集合,一旦初始化就不能修改,使代码更安全

  \pyinline{classmates = ('Michael', 'Bob', 'Tracy')} 
  
\end{frame}

\begin{frame}[fragile]{元组——属性}
  \h tuple不能改变,不可以追加、删除、替换

  \pyinline{t = (1, 2)} 

  \pyinline{t} \\  
  (1, 2)

  
  tuple的“可变性”

  \pyinline{t = ('a', 'b', ['A', 'B'])} 

  \pyinline{t[2][0] = 'X'} 

  \pyinline{t[2][1] = 'Y'} 

  \pyinline{t} \\
  ('a', 'b', ['X', 'Y'])
\end{frame}


\begin{frame}[fragile]{字典}
  \h 字典是一种可变的无序的数据组合类型,使用键-值(key-value)储存,进行极快的
  查找 

  \h 键必须是唯一的,必须是可哈希的

  \pyinline{d = {'Michael': 95, 'Bob': 75, 'Tracy': 85}} 

  \pyinline{d['Michael']} \\  
  95
  
\end{frame}

\begin{frame}[fragile]{字典——原理}
  \begin{easylist}
    & 给定一个名字,比如'Michael',dict在内部就可以直接计算出Michael对应的存放成绩的
    “页码” 
    & 也就是95这个数字存放的内存地址,直接取出来,所以速度非常快
    & 在放进去的时候,必须根据key算出value的存放位置,取的时候才能根据key直接拿到
    value
  \end{easylist}
\end{frame}

\begin{frame}[fragile]{字典——对应性}
  \h 一个key只能对应一个value,多次对一个key放入value,后面的值会把前面的值冲掉

  \pyinline{>>> d['Jack'] = 90}

  \pyinline{d['Jack']} \\ 
  90

  \pyinline{>>> d['Jack'] = 88} 

  \pyinline{d['Jack']} \\ 
  88

\end{frame}

\begin{frame}[fragile]{字典——更新和删除}
  \h 字典是可变的,所以可以用update进行更新 

  \pyinline{d = {'Michael': 95, 'Bob': 75, 'Tracy': 85}} 

  \pyinline{d1 = {'Jack':80}} 

  \pyinline{d.updata(d1)} 

  \pyinline{d} 
  $\{'Jack':80 ,'Michael': 95, 'Bob': 75, 'Tracy': 85\}$


  \h 如果要删除一个key,用pop(key)方法
  
  \pyinline{d.pop('Bob')} \\
  75

  \pyinline{d} \\
  $\{'Jack':80 ,'Michael': 95, 'Tracy': 85\}$
\end{frame}

\begin{frame}[fragile]{集合}
    \h 集合是一个无序的、不重复的元素集,包含两种类型:可变集合(set)和不可变集合
    (frozen set)
    
    ~ (1) 可变集合(set):一组key的集合,不储存value,可添加和删除元素,非可哈希的,
    不能用作字典的键,也不能做其他集合的元素

    \pyinline{s = set([1, 1, 2, 2, 3, 3])} 
  
    \pyinline{s} \\  
    {1, 2, 3}
    
    ~ (2) 不可变集合(frozenset):不可添加和删除元素,可哈希的,能用作字典的键,能做其他集合的元素
\end{frame}

\begin{frame}[fragile]{集合——添加和删除}
    \h set集合是可变的,可以用add(key)进行添加 

    ~ \pyinline{s.add(4)} 

    ~ \pyinline{s} \\ 
    {1, 2, 3, 4}

    \h 如果要删除一个key,用remove(key)方法 

    ~ \pyinline{s.remove(4)} 

    ~ \pyinline{s} \\
    {1, 2, 3}
\end{frame}

\begin{frame}[fragile]{集合——交集和并集}
  \h 两个set可以做数学意义上的交集、并集等操作

  \pyinline{s1 = set([1, 2, 3])} 

  \pyinline{s2 = set([2, 3, 4])} 

  \pyinline{s1 \& s2} \\ 
  {2, 3}

  \pyinline{s1 | s2} \\
  {1, 2, 3, 4}
  
\end{frame}

\begin{frame}[fragile]{练习}
  \begin{easylist}
    & 给定一个字符串变量s,编写Python代码统计字符串中每个字符的出现频度。
    && 例如:
    
    $s = 'hello'$,则输出:

    e:1, h:1, l:2, o:1
  \end{easylist}

 \begin{python}
s = 'hello world'
d = {}
for ch in s:
     if d.get(ch)==None:
         d[ch] = 1
     else:
         d[ch] += 1
 \end{python}
\end{frame}


\begin{frame}[plain]
  \begin{center}
    \Simley{1}{10}{1}

    \Huge ---END---
  \end{center}
\end{frame}

%%% Local Variables:
%%% mode: latex
%%% TeX-master: "../python"
%%% End:

\section{控制流}

\begin{frame}[fragile]{CH3 控制流}
  \begin{easylist} \easyitem
    & if
    & while
    & for
    & break
    & continue
  \end{easylist}
\end{frame}

\begin{frame}[fragile]{补充:函数的简单定义}
  \begin{easylist}
    & 函数可通过def来定义,语法:
    \begin{python}
def function_name(param1, param2, ...):
    function_suite
    \end{python}
    & 例子
    \begin{python}
def test(name, age):
    print("{0} is {1} years old".format(name, age))

test('Mike', 20) #test function
    \end{python}
    & 函数的详细内容下次课讲解
  \end{easylist}

\end{frame}

\begin{frame}[fragile]{if}
  \begin{python}
if boolean_expression1:
    suite1
elif boolean_expression2:
    suite2
elif boolean_expression3:
    suite...
else:
    suite...
  \end{python}

  \begin{easylist} \easyitem
    & 可以有0到多个elif语句
    & 可以有0到1个else语句
  \end{easylist}
\end{frame}

\begin{frame}[fragile]{pass的应用}
  \begin{easylist}
    & 如果需要考虑某个特定的情况,但该情况出现时又不需要做什么,可以用pass 

  \begin{python}
if 5>2:
    pass
  \end{python}

    && pass也可以用在while,函数定义def等地方,共同的作用是保持语法缩进的正确性
  \end{easylist}
\end{frame}

\begin{frame}[fragile]{if实现的条件表达式}
  \begin{easylist}
    & 对于如下语句

    \begin{python}
speed = 200 #any value for test
if speed > 150:
    train = 'Harmony'
else:
    train = 'Normal'
    \end{python}

    & 可以缩写为一句:
    \begin{python}
train = 'Harmony' if speed > 150 else 'Normal'
    \end{python}
    
  \end{easylist}
\end{frame}


\begin{frame}[fragile]{循环 --- while}
  \begin{easylist}
    & 语法
    \begin{python}
while boolean_expression:
    while_suite
else:
    else_suite      
    \end{python}
    & else分支
    && 本身是可选的组成部分
    && 只要循环是正常终止,else分支总会执行
    && 由于break, 返回语句或异常导致跳出循环,则else不会执行
    && else的上述特点对于for循环,及try...except都是一样的 
  \end{easylist}

\end{frame}


\begin{frame}[fragile]{while示例}
  \begin{python}
def list_find(lst, target):
    index = 0
    while index<len(lst):
        if lst[index] == target:
            break
        index += 1
    else:
        index = -1
    return index

list_find(['ruc', 'irm'], 'irm') #??
list_find(['ruc', 'irm'], 'irm2') #??
  \end{python}
\end{frame}


\begin{frame}[fragile]{循环 --- for}
  \begin{easylist}
    & 语法
    \begin{python}
for expression in iterable:
    for_suite
else:
    else_suite      
    \end{python}

    && expression或者是一个单独的变量,或者是一个变量序列,一般以元组形式给出
    && 如果将元组或列表用于expression,则其中的每一数据项都会拆分到表达式的项
  \end{easylist}

\end{frame}

\begin{frame}[fragile]{for示例}
  \begin{python}
def list_find(lst, target):
    for index, x in enumerate(lst):    
        if lst[index] == target:
            break
    else:
        index = -1
    return index

list_find(['ruc', 'irm'], 'irm') #??
list_find(['ruc', 'irm'], 'irm2') #??
  \end{python}

  \begin{easylist}
    & 考虑以上代码中x的含义是什么
  \end{easylist}

\end{frame}


\begin{frame}[fragile]{break}
  \begin{easylist}
    & 提前终止循环语句的执行,如前面的例子
  \end{easylist}
\end{frame}

\begin{frame}[fragile]{continue}
  \begin{easylist}
    & 终止当前循环体内的后续语句执行,重新执行下一轮次的循环
    & 例如:统计一个数字列表中大于10的元素之和
    \begin{python}
def countall(lst):
    count = 0
    for n in lst:
        if n<=10: 
            continue
        count += n
    return count

countall([1,3,12,15]) # Result?
    \end{python}
  \end{easylist}
\end{frame}

\begin{frame}[fragile]{课堂练习}
  \begin{easylist}
    & 复习以前内容
    & 利用蒙特卡洛方法计算圆周率
    & 打印杨辉三角
  \end{easylist}
\end{frame}

\begin{frame}[fragile]{圆周率计算}
  \begin{python}
import random

def pi(count):
    total = 0
    in_circle = 0
    for i in range(count):
        total += 1
        x = random.uniform(-1,1)
        y = random.uniform(-1,1)
        distance = x**2 + y**2
        if(distance<=1):
            in_circle +=1
    return in_circle/total * 4

pi(1000000)    
  \end{python}
\end{frame}

\begin{frame}[plain]
  \begin{center}
    \Simley{1}{10}{1}

    \Huge ---END---
  \end{center}
\end{frame}


%%% Local Variables:
%%% mode: latex
%%% TeX-master: "../python"
%%% End:

\section{函数}

\begin{frame}[fragile]{CH4 函数}
  \begin{enumerate}
     \item 函数的创建
     \item 函数参数
     %\item 参数魔法
     \item 作用域
     \item 名称与docstring
     %\item 注解
     \item 递归
  \end{enumerate}
\end{frame}

\begin{frame}[fragile]{函数介绍}
  \begin{easylist} \easyitem
    & 函数的类型
    && 全局函数
    && 局部函数
    && lambda函数: 特殊表达式,比普通函数有更多限制
    && 方法
    & 根据功能的实现者不同
    && 内置函数
    && 第三方函数
    && 自己编写的函数
  \end{easylist}
\end{frame}

\subsection{函数的创建}
\begin{frame}[fragile]{函数的创建}
  \begin{python}
    def functionName(parameters):
        suite
  \end{python}

  \begin{easylist}
    & parameters是可选的,如果有多个,则用英文逗号分隔
    & 函数参数可以有默认值
    & 每个函数都有返回值,如果未明确指定返回,则返回None
  \end{easylist}
\end{frame}


\begin{frame}[fragile]{函数创建示例}
  计算前top\_n个元素的平均值
  \begin{python}
    def avg_top(lst, top_n):
        total = 0
        for e in lst[:top_n]:
            total += e
        print(total/top_n)

    avg_top([1,3,9,7,6], 4)
  \end{python}
\end{frame}

\subsection{函数参数}
\begin{frame}[fragile]{函数参数}
  \begin{easylist}
    & 参数可以有默认值
  \end{easylist}

  \begin{python}
    def avg_top(lst, top_n=4):
        total = 0
        for e in lst[:top_n]:
            total += e
        print(total/top_n)

    avg_top([1,3,9,7,6])
  \end{python}
\end{frame}

\begin{frame}[fragile]{形式参数与实际参数}
  \begin{easylist}
    & 写在def语句中函数名称后面的变量叫函数的形式参数
    & 函数调用时提供的值称为实际参数
    & 字符串、数字等普通类型默认为传值调用
    & 列表、元组等默认为传递引用方式
  \end{easylist}
\end{frame}

\begin{frame}[fragile]{传值调用示例}
  \begin{python}
    def change(name):
        name = 'Summer'

    name = 'Spring'
    change(name)
    print(name)
  \end{python}

  此时,输出的name为Spring还是Summer?
\end{frame}


\begin{frame}[fragile]{传递引用调用示例}
  \begin{python}
    def change_list(lst):
        lst.append('Summer')
    
    x = ['Spring']
    change_list(x)
    print(x)
  \end{python}

  此时,输出的x结果是['Spring', 'Summer'],还是['Spring']?
\end{frame}


\begin{frame}[fragile]{位置参数与关键字参数}
  \begin{easylist}
    & 函数默认按照声明位置依次传入参数,即位置参数
    & 当参数顺序难以记忆时,可以使用关键字参数方式,即指定参数名称和参数值的方式
    传递参数
  \end{easylist}

  \begin{python}
    def hello(name, greeting):
        print("%s, %s!" % (name, greeting))
    
    hello("悟空", "欢迎你")
    hello(greeting="欢迎你", name="悟空" )
    hello(name="悟空", greeting="欢迎你")
  \end{python}
  
  后面三条语句的输出结果分别是什么?
\end{frame}


\begin{frame}[fragile]{收集参数}
  \begin{easylist}
    & 有时候允许用户提供任意数量的参数非常有用,如实现任意数量的数字的加法
    & Python允许在一个形式参数名称前面加上星号*,表示收集该位置开始的任意数量的
    参数
    & 任意多个关键字参数的收集:**修饰符 (自学)
  \end{easylist}

  \begin{python}
    def add(*numbers):
        total = 0
        for n in numbers:
            total += n
        return total

    add(1,2,3)
  \end{python}
\end{frame}


\subsection{Lambda函数}
\begin{frame}[fragile]{Lambda函数}
  \begin{easylist}
    & 一种快速定义单行的最小函数
    & 按照如下语法创建的匿名函数
    && lambda parameters: expression
    & 特点
    && parameters是可选的
    && epxression不能包含分支或循环,但可以使用条件表达式
    && 不能包含return, yield语句
    && 结果为一个匿名函数
  \end{easylist}
\end{frame}

\begin{frame}[fragile]{Lambda函数示例}
  \begin{python}
def f(x,y):
    return x*y

g = lambda x,y: x*y
print(f(3,4))
print(g(3,4))
  \end{python}

  \begin{easylist}
    & 上例中函数f和g的效果相同,但g的定义更快捷
    & 区别:def 是语句而lambda是表达式
  \end{easylist}
\end{frame}


%\subsection{参数魔法}


\subsection{作用域}
\begin{frame}[fragile]{作用域}
  \begin{easylist}
    & 全局变量
    && 函数内部可以直接调用之前声明的全局变量,但不建议使用这种方式
    & 局部变量
    && 在函数(类)内部定义的变量
  \end{easylist}

  \begin{python}
    def f(): 
        x = 10
    
    x = 1
    f()
    print(x)
  \end{python}
  结果为:?
\end{frame}

\begin{frame}[fragile]{在函数内部绑定全局变量}
  \begin{easylist}
    & 通过global关键字在函数内部将变量绑定为全局变量
  \end{easylist}

  \begin{python}
    def f():   
        global x
        x = 10
    
    x = 1
    f()
    print(x)
  \end{python}

  结果为:10
\end{frame}



\subsection{名称与docstring}

\begin{frame}[fragile]{参数的命名建议}
  \begin{easylist}
    & 对函数及其参数合理命名有助于代码理解
    && 使用命名框架,保持一致性
    && 所有名称避免使用缩略(除非是标准化且广泛使用的)
    && 合理地使用变量与参数名
    &&& x: 适合作为坐标参数
    &&& i: 适合用于简单的循环计数
    && 函数名与方法名应可以表现其行为或返回值
    &&& E.g. 查找在列表中的位置
    \begin{python}
      def find(l, s, i=0)
      def linear_search(l, s, i = 0)
      def first_index_of(sorted_name_list, name, start=0) #GOOD!
    \end{python}
  \end{easylist}
\end{frame}

\begin{frame}[fragile]{docstring}
  \begin{easylist}
    & 对于上面的函数first\_index\_of,如果没有找到对应的name该怎么处理,是返回$-1$,
    还是产生异常?这类信息可以通过docstring提供给函数使用者
    & docstring第一行是对函数的一个简短的描述,紧接着一个空白行,然后是完整描述,
    如果是交互式输入再执行的程序,还会给出一些示例
    & 例子
  \end{easylist}

\end{frame}

\begin{frame}[fragile]{docstring示例}
  \tiny
  \lstinputlisting[keywordstyle=\ttfamily,
  basicstyle=\rmfamily\small]{src/finder.py} 
\end{frame}

\subsection{注解}

\subsection{递归}
\begin{frame}[fragile]{递归}
  \begin{easylist}
    & 函数不仅可以调用其他函数,还可以调用自身
    & 递归的典型特点是一个函数不断调用自身,当到达指定条件时,再逐级返回
    & 普通的幂函数实现
  \end{easylist}

  \begin{python}
def power(x, n):
    result = 1
    for i in range(n):
        result = result*x
    return result

power(2,3) #应输出8
  \end{python}
\end{frame}

\begin{frame}[fragile]{幂函数的递归实现}
  \begin{easylist}
    & 对于任意数字x,power(x,0) = 1
    & 对于任意大于0的数n来说,power(x,n)是x乘以power(x, n-1)的结果
  \end{easylist}

  \begin{python}
def power(x, n):
    if(n==0): 
        return 1
    else:
        return power(x, n-1)*x
    
power(2,3)#应输出8
  \end{python}
\end{frame}


\begin{frame}[fragile]{课堂练习}
  \begin{easylist}
    & 利用递归算法解决汉诺塔(Hanoi Tower)问题
  \end{easylist}

  \begin{center}
    \includegraphics[width=0.6\textwidth]{figure/hanoi.jpg}
  \end{center}

\end{frame}


\begin{frame}[fragile]{Hanoi Tower}
  \begin{python}
def hanoi(a, b, c, n):
    if n==1:
        print("move {0} from {1} to {2}".format(n, a, c))
    else:
        hanoi(a, c, b, n-1)
        print("move {0} from {1} to {2}".format(n, a, c))
        hanoi(b, a, c, n-1)
        
hanoi('a', 'b', 'c', 3)    
  \end{python}
\end{frame}


\begin{frame}[fragile]{求组合$n \choose m$}
  c(n,m)=c(n-1,m-1)+c(n-1,m)
等式左边表示从n个元素中选取m个元素,而等式右边表示这一个过程的另一种实现方法:任意选择n中的某个备选元素为特殊元素,从n中选m个元素可以由此特殊元素的分成两类情况,即m个被选择元素包含了特殊元素和m个被选择元素不包含该特殊元素。
\end{frame}


\subsection{函数式编程}
\begin{frame}[fragile]{函数式编程}
  \begin{easylist}
    & 面向过程的程序设计
    && 把复杂任务分解成简单的任务,简单任务通常以函数表示
    & 函数式编程(Functional Programming)
    && 也可以归结到面向过程的程序设计,但其思想更接近数学计算。
    && 就是越低级的语言,越贴近计算机,抽象程度低,执行效率高,比如C语言;越高级
    的语言,越贴近计算,抽象程度高,执行效率低,比如Lisp语言。
    && 函数式编程就是一种抽象程度很高的编程范式,纯粹的函数式编程语言编写的函数
    没有变量。任意一个函数,只要输入是确定的,输出就是确定的,这种没有副作用的函
    数称为纯函数。
    && 允许把函数本身作为参数传入另一个函数,还允许返回一个函数!
  \end{easylist}
\end{frame}

\begin{frame}[fragile]{高阶函数:Higher-order function}
  \begin{easylist}
    & 变量可以指向函数
    & 函数名也是变量
    & 传入函数
    & 
  \end{easylist}

\end{frame}


\begin{frame}[plain]
  \begin{center}
    \Simley{1}{10}{1}

    \Huge ---END---
  \end{center}
\end{frame}

%%% Local Variables:
%%% mode: latex
%%% TeX-master: "../python"
%%% End:

\section{模块}

\begin{frame}[fragile]{CH5 模块}
  \begin{enumerate}
     \item 模块介绍
     \item 模块的导入方式
     \item 模块的作用域
     \item 模块的测试
     \item 模块导入的路径搜索
     \item 包
  \end{enumerate}
\end{frame}

\begin{frame}[fragile]{模块介绍}
  \begin{easylist} \easyitem
    & 随着程序代码越写越多,在一个文件里代码就会越来越长,越来越不容易维护。
    & 为了编写可维护的代码,把函数分组放到不同的文件里,这样,每个文件包含的代码
    就相对较少,很多编程语言都采用这种组织代码的方式。
    & 在Python中,一个.py文件就称之为一个模块(Module)。
    && 模块:把多个函数组织到一起,方便其他程序调用
    && 提高了代码的可维护性
    && 编写代码不必从零开始。当一个模块编写完毕,就可以被其他地方引用。

    & 之前我们编写的程序也保存在.py文件中,程序和模块的区别在于:
    && 程序的设计目标是运行
    && 模块的设计目标是由其他程序导入并使用
  \end{easylist}
\end{frame}


\begin{frame}[fragile]{模块的导入方式}
  \begin{easylist}
    & import importable
    & import importable1, importable2, ..., importableN
    & import importable as preferred\_name
    & from importable import *
  \end{easylist}
\end{frame}

\begin{frame}[fragile]{标准库中的模块使用示例}
  \begin{python}
import sys
from pprint import pprint
pprint(sys.path)
  \end{python}

['', \\
 '/usr/lib/python3.4', \\
 '/usr/lib/python3.4/plat-x86\_64-linux-gnu', \\
 '/usr/lib/python3.4/lib-dynload', \\
 '/usr/local/lib/python3.4/dist-packages', \\
 '/usr/lib/python3/dist-packages']

\end{frame}


\begin{frame}[fragile, allowframebreaks]{自定义模块}
  \lstinputlisting[keywordstyle=\ttfamily,
  basicstyle=\rmfamily\normalsize]{src/ch5/hello.py}
  
  \newpage
  ~\\
  \begin{easylist}
    & 行1的注释可以让hello.py文件直接在Unix/Linux/Mac上运行
    & 行2的注释表示.py文件本身使用标准UTF-8编码
    & 第4到6行是一个字符串,表示模块的文档注释,任何模块代码的第一个字符串都被视
    为模块的文档注释
    & 第8行导入了引用的模块
    & 第10行使用\_\_author\_\_变量把作者写进去
  \end{easylist}

\end{frame}


\begin{frame}[fragile]{如何运行}
  \begin{easylist}
    & 方式1:
    && 保存到hello.py文件中
    && 进入命令行,通过cd命令进入hello.py文件所在的目录
    &&& python3 hello.py
    &&& python3 hello.py Tom    
    & 方式2:
    && 启动python交互环境
    &&& \pyinline{import hello}
    &&& \pyinline{hello.sayHi()}
    &&& \pyinline{help(hello)}
    & 观察sys.argv是否包含了模块对应的文件名称
  \end{easylist}

\end{frame}


\begin{frame}[fragile]{模块的作用域}
  \begin{easylist}
    & 在一个模块中,我们可能会定义很多函数和变量,但有的函数和变量我们希望给别人
    使用,有的函数和变量我们希望仅仅在模块内部使用。
    && 通过\_前缀定义的函数和变量只能在模块内部访问
    && 其他函数和变量则是公开可访问的
    && \_\_xxx\_\_ 这样的变量可以被直接引用,但通常有特殊含义
    && 如\_\_name\_\_,  \_\_author\_\_
    & private函数和变量“不应该”被直接引用,而不是“不能”被直接引用,是因为Python
    并没有一种方法可以完全限制访问private函数或变量
  \end{easylist}
\end{frame}


\begin{frame}[fragile, allowframebreaks]{作用域示例: hello2.py}
    \lstinputlisting[keywordstyle=\ttfamily,
  basicstyle=\rmfamily\normalsize]{src/ch5/hello2.py}
 
 \begin{easylist}
    & 请分别用命令行和Python交互环境进行测试
    & 问题:能够在交互环境中通过hello2.\_sayInChinese()访问私有方法?
    & 实验:添加代码,使得程序能够根据命令行传入的参数,决定源代码26行处是调用
  \_sayInChinese()还是\_sayInEnglish()
  && 假设命令行传入的第一个有效参数用于指定语言,中文对应为zh,英文对应为en
  \end{easylist}
  
\end{frame}


\begin{frame}[fragile, allowframebreaks]{hello\_lang.py}
    \lstinputlisting[keywordstyle=\ttfamily,
  basicstyle=\rmfamily\normalsize]{src/ch5/hello_lang.py}
 
 \begin{easylist}
    & python3 hello\_lang.py zh
    & python3 hello\_lang.py en
    & python3 hello\_lang.py zh Tom
    & python3 hello\_lang.py en Tom
  \end{easylist}
\end{frame}


\begin{frame}[fragile]{模块的测试}
  \begin{easylist}
    & 模块本身用于定义函数、类及其他一些内容
    & 在模块中添加一些检查模块本身是否正常工作的测试代码非常有用
    \lstinputlisting[keywordstyle=\ttfamily,
    basicstyle=\rmfamily\normalsize]{src/ch5/hello3.py}   
    & 打开Python交互环境测试
    && \pyinline{import hello3}
    && \pyinline{hello3.hello()}
  \end{easylist}
\end{frame}


\begin{frame}[fragile]{\_\_name\_\_}
  \begin{easylist}
    & \pyinline{hello3.__name__}
    & \pyinline{__name__}
    & 因此,在测试模块时,可以通过如下方式:
    \begin{python}
      if __name__ == '__main__':
          test_suite...
    \end{python}
    && 此时,如果将模块作为独立的程序,条件判断将会满足,继续执行测试代码
    && 如果是import引入模块,则条件表达式不成立,测试代码被忽略
  \end{easylist}
\end{frame}


\begin{frame}[fragile]{如何让Python找到自定义的模块}
  \begin{enumerate}
  \item 在源代码目录下执行python
  \item  设置sys.path
  \end{enumerate}

  \lstinputlisting[keywordstyle=\ttfamily,
    basicstyle=\rmfamily\normalsize]{src/ch5/path_test.py} 
\end{frame}

\subsection{包}
\begin{frame}[fragile]{包的处理}
  \begin{easylist}
    & 包是一个有层次的文件目录结构,由模块和子包组成。
    && 为平坦的名称空间加入了有层次的组织结构
    && 允许程序员把有联系的模块组织到一起
    && 允许分发者使用目录结构而非一大堆文件
    && 有助于解决模块名称冲突问题
  \end{easylist}

\end{frame}


\begin{frame}[fragile]{\_\_init\_\_.py文件}
  \begin{easylist}
    & python的每个模块的包中,都有一个\_\_init\_\_.py文件,有了这个文件,我们才
    能导入这个目录下的module
    && 该文件可以为空
    && 我们在导入一个包时,实际上导入了它的\_\_init\_\_.py文件
  \end{easylist}

\end{frame}

\begin{frame}[fragile]{包的示例}
  graphics/ \\
  ~~~~\_\_init\_\_.py \\
  ~~~~primitive/ \\
  ~~~~~~~~\_\_init\_\_.py \\
  ~~~~~~~~line.py \\
  ~~~~~~~~fill.py \\
  ~~~~~~~~text.py \\
  ~~~~formats/ \\
  ~~~~~~~~\_\_init\_\_.py \\
  ~~~~~~~~png.py \\
  ~~~~~~~~jpg.py \\
\end{frame}

\begin{frame}[fragile]{以上包结构的引用方式}
  \begin{python}
    import graphics.primitive.line
    from graphics.primitive import line
    from graphics.primitive import *
    import graphics.formats.jpg as jpg
  \end{python}
  
  \begin{easylist}
    & 第1行在使用line中的方法时,只能用全名称引用:
    && graphics.primitive.line.xxxx
    & 第2行和第3行的方式,则可以直接使用如下方式
    && line.xxxx, fill.xxxx
    && jpg.xxxx
  \end{easylist}

\end{frame}

\begin{frame}[fragile]{练习}
  \begin{easylist}
    & 实现以上包结构
    && line.py, fill.py 和 text.py中分别写一个draw()方法,在该方法中输入一个简单
    的字符串
    && 在png.py和jpg.py中添加open()方法和close()方法,方法本身只输出一个提示字符
    串即可
  \end{easylist}
\end{frame}


\begin{frame}[fragile]{\_\_init\_\_.py中的\_\_all\_\_}
  \begin{easylist}
    & 通过from p1.p2 import *,可以一次性引入包里面的所有子模块
    & 如果想限定默认引入的子模块集合,可以通过设置\_\_init\_\_.py,例如:
    && 在\_\_init\_\_.py中添加如下内容:\\
    \_\_all\_\_ = ['line', 'fill']
    && from graphics.primitive import *
    && 请完善代码并测试能够直接调用line.xxx()和text.xxxx(),并分析原因
  \end{easylist}
\end{frame}


\begin{frame}[plain]
  \begin{center}
    \Simley{1}{10}{1}

    \Huge ---END---
  \end{center}
\end{frame}

%%% Local Variables:
%%% mode: latex
%%% TeX-master: "../python"
%%% End:

\section{Python标准库}

\begin{frame}[fragile]{CH6 Python标准库}
  \begin{easylist} \easyitem
    & 内置电池:Battery Included
    && 用于形容Python标准库
    & 标准模块
    && string 
    && io \& sys
    && optparse
    && math
    && random
  \end{easylist}
\end{frame}

\subsection{字符串标准模块:string}
\begin{frame}[fragile]{字符串标准模块:string}
  \begin{easylist}
    & 提供了字符串相关的一些常量 

    \pyinline{string.ascii\_letters} 

    \pyinline{string.ascii\_lowercase} 

    \pyinline{string.ascii\_uppercase} 

    \pyinline{string.digits} 

    \pyinline{string.hexdigits} 

    \pyinline{string.punctuation}

  \end{easylist}
\end{frame}


\subsection{IO输出相关模块}
\begin{frame}[fragile]{IO输出相关标准模块}
  \begin{easylist}
    & 标准输出
    && sys.stdout
    & 字符串输出
    && io.StringIO
    & 以下输出方式是等价的
  \end{easylist}

  \begin{python}
    import sys,io
    print("hello")
    print("hello", file=sys.stdout)
    sys.stdout.write("hello\n") #python3 中会同时输出字符串的数量
  \end{python}
\end{frame}

\begin{frame}[fragile]{StringIO}
  \begin{easylist}
    & 如果要把输出重定向到字符串中,在需要时再获取,可以StringIO
    && 注意:Python 2.x和3.x在输出时的行为不同
  \end{easylist}

  \begin{python}
    import sys, io
    out = io.StringIO()
    sys.stdout = out
    out.write('how are you')
    print('hello')
    print('guys')
    sys.stdout = sys.__stdout__
    print(out.getvalue())
  \end{python}
\end{frame}


\subsection{命令行参数模块}
\begin{frame}[fragile, allowframebreaks]{命令行参数模块:optparse}
  \begin{easylist}
    & 如何指定脚本运行的命令行参数
    && 例如,Linux命令 ls --l
    && Python提供的optparse模块对命令行参数的解析处理提供了良好的支持
  \end{easylist}

  \lstinputlisting[keywordstyle=\ttfamily]{src/ch6/cmdopt.py}
\end{frame}


\begin{frame}[fragile]{分析}
  \begin{easylist}
    & 说明
    && 长短参数名称
    && 参数对应的变量名称及获取方式
    && metavar参数:提醒用户该命令行参数所期待的参数,参数中的字符串会自动变为大写
    && parser.parse\_args()
    &&& 解析后得到的options拥有两个属性:filename和verbose

    & 假设文件保存到cmdopt.py,执行:
    && python3 cmdopt.py -f cmdopt.py,观察输出结果
    && python3 cmdopt.py -h
  \end{easylist}

\end{frame}


\subsection{数学标准模块:math}
\begin{frame}[fragile]{数学标准模块:math}
  \begin{easylist}
    & math.exp(x)
    \[e^x\]
    & math.sqrt(x)
    \[\sqrt{x}\]
    & math.pi
    & math.e
    & math.log
    && \pyinline{math.log(math.e)}
    && \pyinline{math.log2(2)}
    && \pyinline{math.log10(10)}
    && \pyinline{math.log1p(math.e - 1)}
  \end{easylist}
\end{frame}


\subsection{随机数模块:random}
\begin{frame}[fragile]{随机数标准模块:random}
  \begin{easylist}
    & 生成$[0, 1)$之间的随机数
    && \pyinline{import random}
    && \pyinline{random.random()}
    & random.gauss(mu, sigma)生成一个均值为mu,标准值为sigma的符合高斯分布的随机
    数
    && \pyinline{random.gauss(0, 1)}
  \end{easylist}
\end{frame}

\begin{frame}[fragile]{高斯分布模拟}
  \begin{easylist}
    & 通过random.gauss()生成一组随机数
    & 通过柱状图显示结果
  \end{easylist}
  \pause
  \begin{python}
    import matplotlib.pyplot as plt
    import random

    data = [random.gauss(0,1) for x in range(10000)]
    plt.hist(data, bins = 50)
    plt.show()
  \end{python}
\end{frame}

\begin{frame}[fragile]{Result}
  \includegraphics[width=0.8\textwidth]{figure/gauss1.png}
\end{frame}


\begin{frame}[fragile]{高斯分布模拟}
  \begin{easylist}
    & 通过高斯分布的概率密度函数生成(x, y)对,利用散点图显示结果
    \[ f(x) = \dfrac{1}{\sigma \sqrt{2 \pi}} e^{-\dfrac{(x-\mu)^2}{2 \sigma^2}} \]
  \end{easylist}
\end{frame}

\begin{frame}[fragile]{高斯分布模拟代码}
  \lstinputlisting[keywordstyle=\ttfamily]{src/ch6/gauss.py}
\end{frame}

\begin{frame}[fragile]{Result}
  \includegraphics[width=0.8\textwidth]{figure/gauss2.png}
\end{frame}


\subsection{日志处理模块: logging}
\begin{frame}[fragile]{日志处理标准库:logging}
  \begin{easylist}
    & 日志的作用
    
    & 日志的级别
    && DEBUG:详细的信息,通常只出现在诊断问题上
    && INFO:确认一切按预期运行
    && WARNING:一个迹象表明,一些意想不到的事情发生了,或表明一些问题在不久的将来(例如。磁盘空间低”)。这个软件还能按预期工作。
    && ERROR:个更严重的问题,软件没能执行一些功能
    && CRITICAL:一个严重的错误,这表明程序本身可能无法继续运行
    
    & 优先级关系
    && CRITICAL > ERROR > WARNING > INFO > DEBUG
  \end{easylist}
\end{frame}


\begin{frame}[fragile]{日志示例}
  
  \begin{python}
import logging

logging.debug('debug message')
logging.info('info message')
logging.warning('warning message')
logging.error('error message')
logging.critical('critical message')
  \end{python}

\begin{easylist}
  & 默认输出到控制台,级别在warning及以上的信息会输出,
  & 可以通过basicConfig设置输出方式和记录级别
\end{easylist}
\end{frame}

\begin{frame}[fragile]{日志示例:输出到文件}
  \begin{python}
import os
import logging


FILE = os.getcwd()
logging.basicConfig(filename=os.path.join(FILE, 'log.txt'),
                    level=logging.DEBUG)
logging.debug('some debug messages...')
logging.info('some info messages...')
logging.warning('some warning messages...')
  \end{python}
\end{frame}


% \subsection{正则表达式模块:re}
% \begin{frame}[fragile]{正则表达模块:re}
%   ~
% \end{frame}


% \subsection{os与os.path标准模块}
% \begin{frame}[fragile]{os与os.path标准模块}
%   ~
% \end{frame}

\begin{frame}[fragile]{第3方扩展:Requests}
  \begin{easylist}
    & 通过Requests包可以更方便地实现Web数据的采集
    && Install: pip install requests
  \end{easylist}

  \begin{python}
import requests

url = 'http://download.pchome.net/wallpaper/zhiwu/'
response = requests.get(url)
print(response.text)    
  \end{python}
\end{frame}

\begin{frame}[fragile]{BeautifulSoup}
  \begin{easylist}
    & 怎么解析抓取到的网页,例如,如何抽取网页中的图片?
    & BeautifulSoup --- Html Parser
    && Install: pip install beautifulsoup4
  \end{easylist}
\end{frame}

\begin{frame}[fragile]{Example}
  \begin{python}
import requests

url = 'http://download.pchome.net/wallpaper/zhiwu/'
response = requests.get(url)
print(response.text)

from bs4 import BeautifulSoup as soup
doc = soup(response.text, 'html')
images = [element.get('src') for element in doc.find_all('img')]    
  \end{python}
\end{frame}

\begin{frame}[fragile]{课堂练习}
  \begin{easylist}
    & 编写程序,实现对任意指定网页,下载该网页包含的所有图片到指定的文件夹当中
  \end{easylist}

\end{frame}

\begin{frame}[fragile]{Thanks}
  ~
\end{frame}

%%% Local Variables:
%%% mode: latex
%%% TeX-master: "../python"
%%% End:

\section{CH7 面向对象编程}

\begin{frame}[fragile]{CH7 面向对象编程}
  \begin{easylist} \easyitem
    & 简介
    & 类和实例
    & 数据封装
    & 继承和多态
    & Duck Typing
    & 获取对象信息
    & 实例属性和类属性
    & 高级特性
    & 小结
  \end{easylist}
\end{frame}

\subsection{简介}
\begin{frame}[fragile]{\newsec 面向对象编程简介}
  \begin{easylist}
    & 面向对象编程OOP(Object Oriented Programming)
    && 一种程序设计思想,OOP把对象作为程序的基本单元,一个对象包含了数据和操作
    数据的函数。
    && 面向过程的程序设计把计算机程序视为一系列的命令集合,即一组函数的顺序执
    行
    && OOP通过把大块函数切割为小块函数来降低系统的复杂度
    && OOP把计算机程序看做是一组对象的集合,程序执行就是一些列消息在各个对象之
    间传递,给对象发消息实际上就是调用对象对应的关联函数
    & OOP特点
    && 封装、继承、多态
    & 类(Class)与实例(Instance)
  \end{easylist}
\end{frame}

\begin{frame}[fragile]{举例: 面向过程方式}
  处理学生成绩并打印输出,面向过程设计方式:

  \begin{python}
std1 = {'name': '韩寒', 'score':85}    
std2 = {'name': '黄锤', 'score':90}

def print_score(std):
    print('%s: %s' % (std['name'], std['score']))

print_score(std1)
print_score(std2)
  \end{python}
\end{frame}

\begin{frame}[fragile]{举例: 面向对象方式}
  \begin{python}
class Student(object):
    def __init__(self, name, score):
        self.name = name
        self.score = score

    def print_score(self):
        print('%s: %s' % (self.name, self.score))

if __name__ == '__main__':
    han = Student('韩寒', 85)
    huang = Student('黄锤', 90)
    han.print_score()
    huang.print_score()
  \end{python}
\end{frame}


\begin{frame}[fragile]{例子解释}
  \begin{easylist}
    & class关键字
    & object
    & self
  \end{easylist}
\end{frame}


\subsection{类和实例}
\begin{frame}[fragile]{\newsec 类和实例}
  \begin{easylist}
    & 类是抽象的模板,比如Student类
    & 实例是根据类创建出来的一个个具体的“对象”
    & 每个对象都拥有相同的方法,但各自的数据可能不同。
  \end{easylist}
\end{frame}

\begin{frame}[fragile]{类的定义}
  通过class关键字定义类

  \begin{python}
class Student(object):
    pass
  \end{python}

  \begin{easylist}
    & class后面紧接着是类名
    & 类名通常是大写开头的单词
    & 类名后紧接着是(object),表示该类是从哪个类继承下来的
    && object是所有类最终都会继承的类
  \end{easylist}
\end{frame}


\begin{frame}[fragile]{类的实例化}
  类名+()实现, 例如:

  \begin{python}
    xiaoming = Student('小明', 85)
    print(xiaoming)    
    print(Student)
  \end{python}

  <\_\_main\_\_.Student object at 0x7f16e99ee710> \\
  <class '\_\_main\_\_.Student'>
\end{frame}

\begin{frame}[fragile]{实例的变量绑定}
  可以自由地给一个实例变量绑定属性和方法,比如,给实例xiaoming绑定一个name属性,
  和speak()方法:

\begin{python}
def speak(something):
    print('speak:' , something)

xiaoming.name = 'xiaoming'
xiaoming.speak = speak
print(xiaoming.name)
xiaoming.speak('hello')
\end{python}
\end{frame}


\begin{frame}[fragile]{实例的初始化}
  通过定义特殊的\_\_init\_\_方法,在创建实例的时候,可以绑定期望的属性

\begin{python}
  class Student(object):

    def __init__(self, name, score):
        self.name = name
        self.score = score
\end{python}

\begin{easylist}
  & \_\_init\_\_方法的第一个参数永远是self,表示创建的实例本身
  & 在\_\_init\_\_方法内部,可以把各种属性绑定到self,因为self就指向创建的实例本
  身。

\end{easylist}

\end{frame}


\begin{frame}[fragile]{\_\_init\_\_()解释}
  \begin{easylist}
    & 有了\_\_init\_\_方法,在创建实例的时候,就不能传入空的参数了,必须传入与该
    方法匹配的参数,但self不需要传,Python解释器自己会把实例变量传进去
  
    \pyinline{xiaoli = Student('小李', 80)}
    & 和普通的函数相比,在类中定义的函数只有一点不同,就是第一个参数永远是实例变
    量self,并且,调用时,不用传递该参数。除此之外,类的方法和普通函数没有什么区
    别,所以,仍然可以用默认参数、可变参数、关键字参数和命名关键字参数。

  \end{easylist}
\end{frame}

\subsection{数据封装}
\begin{frame}[fragile]{\newsec 数据封装}
  \begin{easylist}
    & 实例中拥有的属性、方法,被封装到实例中
    & 在实例中使用本身的属性或方法,用self.xxx方式
    & 在类中定义方法时,第一个参数必须为self
    & 在外部调用方法时,无需指定self,类会自动把self传入
    & 封装的优点是调用简单,无需关心内部实现细节
  \end{easylist}
\end{frame}


\subsection{继承和多态}
\begin{frame}[fragile]{\newsec 继承和多态}
  \begin{block}{}
    在OOP程序设计中定义一个class的时候,可以从某个现有的class继承,新的class称为
    子类(Subclass),而被继承的class称为基类、父类或超类(Base class、Super
    class)。
  \end{block}

  \begin{python}
    class Postgraduate(Student):
        '''post graduate student'''
        pass
  \end{python}

  \color{blue}继承可以把父类的所有功能都直接拿过来,这样就不必从零做起,子类只需要新增自己特
  有的方法,也可以把父类不适合的方法覆盖重写(多态)。

\end{frame}


\begin{frame}[fragile]{多态}
  \begin{easylist}
    & 多态即多种状态
    & 同一个实体同时具有多种形式, 对基类的引用指向子类的对象
    && 例子中,子类和父类都存在相同的sayHi()方法,此时,子类的sayHi()覆盖了父类的
    sayHi(),在代码运行的时候,总是会调用子类的sayHi()
  \end{easylist}
\end{frame}


\begin{frame}[fragile, allowframebreaks]{多态示例}
  \lstinputlisting[keywordstyle=\ttfamily]{src/ch7/student.py}
\end{frame}


\begin{frame}[fragile]{isinstance}
  \begin{easylist}
    & 可以用isinstance判断实例的类型
  \end{easylist}

\begin{python}
  wang = Postgraduate('王二小', 80)
  isinstance(wang, Postgraduate) #True or False?
  isinstance(wang, Student) #True or False?
\end{python}
\end{frame}


\subsection{Duck Typing}
\begin{frame}[fragile]{\newsec Duck Typing\footnote{\url{https://en.wikipedia.org/wiki/Duck\_typing}}}
  \begin{easylist}
    & 对于静态语言(如Java),如果需要传入Student类型,则传入的对象必须是Student类
    型或者它的子类  
    & 对于Python这样的动态语言,只需要保证传入的对象有sayHi()方法就可以
    & 这种设计方法称之为鸭子类型:
    && \color{blue} 一个对象有效的语义,不是由继承自特定的类或实现特定的接口,而是由当前方法
    和属性的集合决定。
  \end{easylist}

\begin{python}
class Robot():
    def sayHi(self):
        print("I'm robot")
        
meet(Robot())    
\end{python}
\end{frame}

\begin{frame}[fragile]{Duck Typing}
  \begin{block}{Duck Test}
    {\huge "} When I see a bird that walks like a duck and swims like a duck and quacks
    like a duck, I call that bird a duck.{\huge "}

    \begin{flushright}
      \scriptsize{--- Indiana poet James Whitcomb Riley (1849--–1916)}
    \end{flushright}
  \end{block}
\end{frame}

\begin{frame}[fragile, allowframebreaks]{Duck Typing示例}
  \lstinputlisting{src/ch7/duck.py}
\end{frame}


\subsection{获取对象信息}
\begin{frame}[fragile]{\newsec 获取对象信息}
  \begin{easylist}
    & 当我们拿到一个对象的引用时,如何知道这个对象是什么类型、有哪些方法呢?
    && isinstance(): 判断对象是否为某个类的实例 
    && type(): 判断对象的类型
    && dir(): 获得对象的所有属性和方法
  \end{easylist}
\end{frame}

\begin{frame}[fragile]{type()}
  \begin{python}
    type(123) == type(456) 
    type(123) == int 
    type('hello') == str 

    import types
    type(abs) == types.BuiltinFunctionType
    type(lambda x:x) == types.LambdaType
    type((x for x in range(10)))==types.GeneratorType

    def hi():
        pass
    type(hi) == types.FunctionType
  \end{python}
\end{frame}

\begin{frame}[fragile]{dir()}
  \begin{easylist}
    & 获得一个对象的所有属性和方法,可以使用dir()函数,它返回一个包含字符串的list
  \end{easylist}

  \begin{python}
    wang = Postgraduate('王小二', 80)
    dir(wang)

    ['__class__', '__delattr__', '__dict__', '__dir__', '__doc__', '__eq__', '__format__', '__ge__', '__getattribute__', '__gt__', '__hash__', '__init__', '__le__', '__lt__', '__module__', '__ne__', '__new__', '__reduce__', '__reduce_ex__', '__repr__', '__setattr__', '__sizeof__', '__str__', '__subclasshook__', '__weakref__', 'name', 'print_score', 'sayHi', 'score']
  \end{python}
\end{frame}


\subsection{实例属性和类属性}
\begin{frame}[fragile]{\newsec 实例属性和类属性}
  \begin{easylist}
    & 实例属性隶属于实例
    && 通过self或实例绑定属性
    & 类属性隶属于类,其所有实例均可以访问到
    && 直接在class中定义的属性
  \end{easylist}
\end{frame}

\begin{frame}[fragile, allowframebreaks]{实例属性和类属性示例}
  \begin{python}
#实例属性示例
class Student(object):
    def __init__(self, name):
        self.name = name #通过self绑定属性
        
s = Student('Bob')
s.score = 90 #通过实例绑定属性
print("%s : %s" % (s.name,  s.score))
\end{python}

\newpage

\begin{python}
#类属性示例
class Student(object):
    name = 'Student'
\end{python}

  $>>>$ s = Student() \# 创建实例s \\
  $>>>$ print(s.name) \# 打印name属性,因为实例并没有name属性,所以会继续查找class的
  name属性 \\
  Student \\
  $>>>$ print(Student.name) \# 打印类的name属性 \\
  Student \\
  $>>>$ s.name = 'Michael' \# 给实例绑定name属性 \\

  \newpage

  $>>>$ print(s.name) \# 由于实例属性优先级比类属性高,因此,它会屏蔽掉类的name属性 \\
  Michael \\
  $>>>$ print(Student.name) \# 但是类属性并未消失,用Student.name仍然可以访问 \\
  Student \\
  $>>>$ del s.name \# 如果删除实例的name属性 \\
  $>>>$ print(s.name) \# 再次调用s.name,由于实例的name属性没有找到,类的name属性就显
  示出来了 \\
  Student    
\end{frame}


\subsection{小结}
\begin{frame}[fragile]{\newsec 面向对象编程小结}
  \begin{easylist}
    & 类是创建实例的模板,而实例则是一个一个具体的对象,各个实例拥有的数据都互相独立,互不影响;

    & 方法就是与实例绑定的函数,和普通函数不同,方法可以直接访问实例的数据;
    & 通过在实例上调用方法,我们就直接操作了对象内部的数据,但无需知道方法内部的实现细节。
    & 和静态语言不同,Python允许对实例变量绑定任何数据
    && 也就是说,对于两个实例变量,虽然它们都是同一个类的不同实例,但拥有的变量
    名称都可能不同
    & Duck Typing
    & 实例属性与类属性
  \end{easylist}
\end{frame}

\begin{frame}[fragile]{练习}
  \begin{easylist}
    & 假设磁盘中存在一个学生成绩的文本文件,每行格式如下:
    && 123,成工,高等数学,80
    && 123,成工,线性代数,85
    && 124,王小二,线性代数,80
    && $\cdots$
    & 利用OOP设计思想,实现以下功能:
    && 按照学号或姓名查找满足条件的学生的所有成绩信息,并输出其平均分
    && 根据课程名称输出所有学生的成绩,并输出平均成绩、最高分和最低分
  \end{easylist}

\end{frame}


\begin{frame}[fragile, allowframebreaks]{Test}
  \begin{python}
class Query():
    def __init__(self, filename):
        slef.filename = filename
        
    def _parse_line(self, line):
        return 123, 'zhang', 'math', 80
        
    def find_by_number(self, num):
        scores = {}
        f = open(self.filename, 'r')
        for line in f:
            n, name, course, score = self._parse_line(line)
            if n == num:
                scores[course] = score
                
        total = 0
        for course, score in scores.items():
            print(course, '==>', score)
            total += score
        print('avg', total/len(scores))    

    if __name__ == '__main__':
        q = Query('some_file.txt')
        q.find_by_number(123)
  \end{python}
\end{frame}



\subsection{Python面向对象的高级特性}
\begin{frame}[fragile]{Python面向对象的高级特性}
  \begin{easylist}
    & \_\_slots\_\_
    & @property
    & 多重继承
    & 定制类
    & 枚举类
    & 元类
  \end{easylist}
\end{frame}


\subsubsection{\_\_slots\_\_}
\begin{frame}[fragile, allowframebreaks]{\_\_slots\_\_的引入}
  实例绑定与类的绑定示例
  \begin{python}
class Student(object):
    pass

s = Student()
s.name = 'Lucy'
print(s.name) # 输出Lucy

def set_age(self, age):
    self.age = age

from types import MethodType
s.set_age = MethodType(set_age, s) # 给实例绑定方法
s.set_age(25) # 调用实例方法
print(s.age) # 测试结果

s2 = Student() # 创建新的实例
s2.set_age(25) # 尝试调用方法
  \end{python}

  调用s2.set\_age(25)会给出错误提示:对象Student没有属性set\_age,此时,为了给所有
  实例都绑定方法,可以给class绑定方法,如下:
\begin{python}
def set_score(self, score):
    self.score = score

Student.set_score = set_score # 在类级别上绑定方法

s.set_score(100)
print(s.score)
s2.set_score(99)
print(s2.score)
\end{python}

\begin{easylist}
  & 新问题:
  && 如果要限制对实例的属性随意赋值,该怎么处理?
  && \_\_slots\_\_
\end{easylist}
\end{frame}



\begin{frame}[fragile]{\_\_slots\_\_}
  \_\_slots\_\_规定了class能添加的属性

\begin{python}
class Student(object):
    __slots__ = ('name', 'age')

s = Student()
s.age = 25
s.score = 90 # Error!
\end{python}

AttributeError: 'Student' object has no attribute 'score'

注意:\_\_slots\_\_定义的属性仅对当前类实例起作用,对继承的子类是不起作用的
\end{frame}


\subsubsection{@property}
\begin{frame}[fragile, allowframebreaks]{@property的引入}
  \begin{easylist}
    & 在绑定属性时,如果我们直接把属性暴露出去,虽然写起来很简单,但是,没办法检
    查参数
    && 例如,成绩score,我们希望把成绩的范围限制到0--100之间
    && 思路:增加get\_score()和set\_score()对进行读取和修改
  \end{easylist}

  \begin{python}
class Student(object):
    def get_score(self):
        return self._score

    def set_score(self, value):
        if not isinstance(value, int):
            raise ValueError('score必须是整数!')    
        if value < 0 or value > 100:
            raise ValueError('score必须是一个0到100之间的数字')
        self._score = value
  \end{python}

  \newpage
  \begin{python}
s = Student()
s.set_score(60)
print(s.get_score())
s.set_score(999) # Error!
  \end{python}

  \begin{easylist}
    & 上面的调用方法又略显复杂
    & 能否仍然使用s.score = 某个数字, 同时还能判断赋予的数值是否满足要求?
    & Python的@property装饰器负责把一个方法变成属性调用
  \end{easylist}
\end{frame}

\begin{frame}[fragile, allowframebreaks]{@property示例}
  \begin{python}
class Student(object):
    |\color{red}@property|
    def score(self):
        return self._score

    |\color{red}@score.setter|
    def score(self, value):
        if not isinstance(value, int):
            raise ValueError('score必须是整数!')    
        if value < 0 or value > 100:
            raise ValueError('score必须是一个0到100之间的数字')
        self._score = value   
\end{python}

\newpage
\begin{python}
s = Student()
s.score = 60
print(s.score)
|\color{red}s.score = 999| # Error!
\end{python}
\end{frame}

\subsubsection{多重继承}
\begin{frame}[fragile]{多重继承}
  \begin{easylist}
    & 继承是面向对象编程的一个重要的方式,因为通过继承,子类就可以扩展父类的功能。
    & 假设我们要实现以下4种动物:
    && Dog - 狗
    && Bat - 蝙蝠
    && Parrot - 鹦鹉
    && Ostrich - 鸵鸟
  \end{easylist}

\end{frame}

\begin{frame}[fragile]{设计思路}
  \begin{columns}
    \begin{column}{.5\textwidth}
      按照哺乳动物和鸟类归类:


    \end{column}
    
    \begin{column}{.5\textwidth}
      按照“能跑”和“能飞”来归类
      
    \end{column}
  \end{columns}
\end{frame}

\begin{frame}[fragile]{设计思路}
  \begin{python}
class Animal(object):
    pass

# 大类:
class Mammal(Animal):
    pass

class Bird(Animal):
    pass

# 各种动物:
class Dog(Mammal):
    pass

class Bat(Mammal):
    pass

class Parrot(Bird):
    pass

class Ostrich(Bird):
    pass    
  \end{python}
\end{frame}


\begin{frame}[fragile]{定义Runnable和Flyable}
  \begin{python}
class Runnable(object):
    def run(self):
        print('Running...')

class Flyable(object):
    def fly(self):
        print('Flying...')
  \end{python}
\end{frame}

\begin{frame}[fragile]{多重继承示例}
  \begin{python}
class Dog(Mammal, Runnable):
    pass

class Bat(Mammal, Flyable):
    pass    
  \end{python}

  \begin{easylist}
    & 通过多重继承,一个子类就可以同时获得多个父类的所有功能。
  \end{easylist}
\end{frame}

\begin{frame}[fragile]{MixIn}
  \begin{easylist}
    & 在设计类的继承关系时,通常,主线都是单一继承下来的
    && 如:Ostrich继承自Bird
    & 如果需要“混入”额外的功能,通过多重继承就可以实现
    && 比如,让Ostrich除了继承自Bird外,再同时继承Runnable
    & 这种设计方法通常称之为MixIn
    && 为了更好地看出继承关系,通常采用MixIn作为功能性的父类名称的后缀
  \end{easylist}
\end{frame}

\begin{frame}[fragile]{MixIn实例}
  \begin{python}
class RunnableMixIn(object):
    def run(self):
        print('Running...')

class FlyableMixIn(object):
    def fly(self):
        print('Flying...')

class Dog(Mammal, RunnableMixIn):
    pass    
  \end{python}
\end{frame}

\subsubsection{对类进行定制}
\begin{frame}[fragile]{对类进行定制}
  \begin{easylist}
    & \_\_len\_\_
    & \_\_str\_\_
    & \_\_repr\_\_
    & \_\_iter\_\_ 与 \_\_next\_\_
    & \_\_getitem\_\_
    & \_\_getattr\_\_
    & \_\_call\_\_
  \end{easylist}
\end{frame}

\begin{frame}[fragile]{\_\_len\_\_}
  \begin{python}
    class Panda(object):
    def __init__(self):
        pass

    def __len__(self):
        return 10

panda = Panda()
len(panda)
  \end{python}
\end{frame}

\begin{frame}[fragile]{\_\_str\_\_与\_\_repr\_\_}
  \begin{python}
class Panda(object):
    def __init__(self):
        pass

    def __str__(self):
        return '熊猫'

panda = Panda()
print(panda) # 熊猫
panda    # <__main__.Panda at 0x7f20e414db38>
  \end{python}
  
  \begin{easylist}
    & 如何解决非print时输出地址的问题?
  \end{easylist}
\end{frame}


\begin{frame}[fragile]{\_\_str\_\_与\_\_repr\_\_}
  \begin{python}
class Panda(object):
    def __init__(self):
        pass

    def __str__(self):
        return '熊猫'

    __repr__ = __str__
panda = Panda()
print(panda) # 熊猫
panda    # 熊猫
  \end{python}  
\end{frame}


\begin{frame}[fragile]{\_\_iter\_\_ 与 \_\_next\_\_}
  \begin{easylist}
    & 方便通过for循环对对象维持的数据进行遍历
    && 例如:假设拥有一个斐波那契数列的类Fib,通过以下语句循环输出:
    \begin{python}
for n in Fib(10):
     print(n)
    \end{python}
    && 我们希望输出1, 1, 2, 3, 5, 8, 13, 21, 34, 55
    && 怎么实现Fib? 
  \end{easylist}
\end{frame}

\begin{frame}[fragile, allowframebreaks]{Fib}
  \lstinputlisting{src/ch7/fib.py}
\end{frame}

\begin{frame}[fragile]{扩展练习}
  \begin{easylist}
    & 用Python编写程序,对本机指定目录下的文件进行扫描,把所有指定后缀名的文档
    名称统一输出到一个文本文件中。
  \end{easylist}
\end{frame}


\begin{frame}[fragile]{END}
  ~
\end{frame}

%%% Local Variables:
%%% mode: latex
%%% TeX-master: "../python"
%%% End:

\section{CH8 异常、调试与测试}

\begin{frame}[fragile]{CH8 异常、调试与测试}
  \begin{easylist} \easyitem
    & 异常
    && Python的异常处理使用方法
    && Python的异常继承关系
    && 利用raise抛出异常
    & 调试
    && print调试法
    && logging调试法
    && assert
    && pdb
    & 单元测试
  \end{easylist}
\end{frame}

\subsection{异常处理}
\subsubsection{异常处理使用方法}
\begin{frame}[fragile]{异常处理}
  \begin{easylist}
    & 程序在编写过程中,有大量情况需要考虑
    && 除数是否为0
    && 打开文件时,需要判断文件是否存在,有无权限
    && ...
    & 异常可以简化这一处理过程
    & Python的错误处理机制
    && try...except...finally...
  \end{easylist}
\end{frame}

\begin{frame}[fragile]{try}
  \begin{python}
try:
    print('try...')
    r = 10 / 0
    print('result:', r)
except ZeroDivisionError as e:
    print('except:', e)
finally:
    print('finally...')
print('END')    
  \end{python}

  try... \\
  except: division by zero \\
  finally... \\
  END \\
\end{frame}

\begin{frame}[fragile]{try}
  \begin{python}
try:
    print('try...')
    r = 10 / 2
    print('result:', r)
except ZeroDivisionError as e:
    print('except:', e)
finally:
    print('finally...')
print('END')    
  \end{python}

  try... \\
  result: 5.0 \\
  finally... \\
  END \\  
\end{frame}

\begin{frame}[fragile]{Python异常处理的规则}
  \begin{easylist}
    & 遇到第一个满足条件的异常,执行该异常下的语句,忽略后续的其它异常
    & finally永远会被执行
  \end{easylist}

\end{frame}

\subsubsection{异常继承关系}
\begin{frame}[fragile]{Python异常的继承关系}
  \begin{easylist}
    & Python的错误也是class,都继承自BaseException
    && 在使用except时需要注意的是,它不但捕获该类型的错误,还把其子类也“一网打尽”。
  \end{easylist}

  \begin{python}
lst = [x for x in range(10)]
try:
    n = lst[15]
except LookupError as e:
    print('LookupError ', e)
except IndexError as e:
    print('IndexError ', e)
  \end{python}
\end{frame}


\begin{frame}[fragile]{Python异常继承关系示例}
  \begin{center}
    \includegraphics[width=0.8\textwidth]{figure/exception-hierarchy.png}
  \end{center}
  \url{https://docs.python.org/3/library/exceptions.html#exception-hierarchy}
\end{frame}

\subsubsection{利用raise抛出异常}
\begin{frame}[fragile]{抛出异常}
  \begin{easylist}
    & 可以用raise语句来引发一个异常。异常/错误对象必须有一个名字,且它们应是Error或Exception类的子类。
  \end{easylist}

\end{frame}


\begin{frame}[fragile, allowframebreaks]{raise示例}
  \begin{python}
class Student(object):
    |\color{red}@property|
    def score(self):
        return self._score

    |\color{red}@score.setter|
    def score(self, value):
        if not isinstance(value, int):
            raise ValueError('score必须是整数!')    
        if value < 0 or value > 100:
            raise ValueException('score必须在0到100之间')
        self._score = value   
\end{python}

\newpage
\begin{python}
s = Student()
s.score = 60
print(s.score)
|\color{red}s.score = 999| # Error!
\end{python}
\end{frame}


\begin{frame}[fragile, allowframebreaks]{异常可以自定义}
  \begin{easylist}
    & 例如:把以上的score判断条件不满足时,抛出的异常更改为自定义异常
  \end{easylist}

  \begin{python}
class ScoreException(Exception):
    def __init__(self, msg):
        self.msg = msg
        
    def __str__(self):
        return 'ScoreException: ' + repr(self.msg)
    
class Student(object):
    @property
    def score(self):
        return self._score

    @score.setter
    def score(self, value):
        if not isinstance(value, int):
            raise ScoreException('score必须是整数!')    
        if value < 0 or value > 100:
            raise ScoreException('score必须在0到100之间')
        self._score = value   
        
try:       
    s = Student()
    s.score = 60
    print(s.score)
    s.score = 999
except ScoreException as e:
    print(e)    
  \end{python}
\end{frame}

\begin{frame}[fragile]{异常总结}
  \begin{easylist}
    & Python内置的try...except...finally用来处理错误十分方便
    && 出错时,会分析错误信息并定位错误发生的代码位置更为关键
    & 程序也可以主动抛出错误,让调用者来处理相应的错误
    && 应在文档中写清楚可能会抛出哪些错误,以及错误产生的原因
  \end{easylist}
\end{frame}


\subsection{调试}
\begin{frame}[fragile]{调试}
  \large Bug and Debug
  \begin{columns}[onlytextwidth,T]
    \begin{column}{0.5\textwidth}
      \begin{block}{\small 格蕾丝·赫柏(Grace Murray Hopper)}
        \scriptsize{赫柏是一位为美国海军工作的电脑专家。1945年的一天,赫柏
        对Harvard Mark II设置好17000个继电器进行编程后,技术人员在进行整机运行时,
        它突然停止了工作。于是他们爬上去找原因,发现这台巨大的计算机内部一组继电
        器的触点之间有一只飞蛾,这显然是由于飞蛾受光和热的吸引,飞到了触点上,然
        后被高电压击死。所以在报告中,赫柏用胶条贴上飞蛾,并把“bug”来表示"一个在
        电脑程序里的错误"。}
      \end{block}
    \end{column}
    \begin{column}{0.45\textwidth}
      \includegraphics[width=0.85\textwidth]{figure/bug.jpg}
    \end{column}
  \end{columns}

  \begin{easylist}
    & 程序一次编写就能成功运行的概率很小,通常会有各种各样的错误需要调试,因此,
    掌握错误的调试方法非常重要。
  \end{easylist}
\end{frame}

\begin{frame}[fragile]{常用的调试方法}
  \begin{easylist}
    & 观察出错的提示信息
    & 利用print函数输出信息,观察输出结果和预期结果是否一致
    & 通过日志logging替代print
    & 利用assert断言
    & pdb
  \end{easylist}
\end{frame}

\subsubsection{利用print进行调试}
\begin{frame}[fragile]{利用print进行调试}
  \begin{python}
books = ['Python', 'XML', 'Information Retrieval']
for book in books:
    print(book)

print('Run here!') 
if book.price > 50:
    print('High price.')
  \end{python}

以上代码会抛出异常信息:

Traceback (most recent call last): \\
~~File "bug.py", line 5, in <module> \\
~~~~if book.price > 50: \\
AttributeError: 'str' object has no attribute 'price'\\

如果我们觉得前三行代码不太可能出问题,而问题很可能在后面,那我们可以在后面加入一
行print语句,观察程序能否正常运行到该位置,来缩小异常发生的范围
\end{frame}

\subsubsection{利用logging进行调试}
\begin{frame}[fragile]{利用logging进行调试}
  print()的结果默认输出到控制台上,不够灵活和方便,可以使用logging进行控制

  \begin{python}
import logging
logging.basicConfig(level=logging.INFO)

books = ['Python', 'XML', 'Information Retrieval']
for book in books:
    logging.info(book)

logging.debug('Run here!') 
if book.price > 50:
    logging.warn('High price.')
  \end{python}
\end{frame}

\subsubsection{assert调试}
\begin{frame}[fragile]{assert调试}
  assert断言用于判断给定的逻辑表达式是否成立,如果不成立,就会抛出AssertionError
  异常

\begin{python}
books = ['Python', 'XML', 'Information Retrieval']
assert len(books) == 3
\end{python}

启动Python解释器时可以用-O参数来关闭assert,此时的assert语句可看作是pass
\end{frame}

\subsubsection{pdb调试}
\begin{frame}[fragile]{pdb}
  Python的调试器,可以单步运行Python脚本,查看当前运行的代码,查看变量值

  bug.py: 
  \begin{python}
books = ['Python', 'XML', 'Information Retrieval']
for book in books:
    print(book)

if book.price > 50:
    print('High price.')
  \end{python}

  运行:pdb bug.py \\
  或者:python -m pdb bug.py

\end{frame}

\begin{frame}[fragile]{pdb}
  \begin{easylist}
    & n: 执行下一条语句
    & p xxx: 查看变量xxx的当前值
    & l: 列出
  \end{easylist}
\end{frame}

\begin{frame}[fragile]{pdb.set\_trace()}
  在可能出错的地方放置pdb.set\_trace(),程序运行到该条语句时,会暂停并进入pdb调试环
境,此时,可以用p指令查看变量,或者用c继续运行

  bug2.py: 
  \begin{python}
import pdb
books = ['Python', 'XML', 'Information Retrieval']
for book in books:
    print(book)

pdb.set_trace()
if book.price > 50:
    print('High price.')
  \end{python}
\end{frame}


\subsection{测试}

\begin{frame}[fragile]{测试}
  \begin{easylist}
    & 代码规范检查
    & 单元测试
  \end{easylist}
\end{frame}

\begin{frame}[fragile]{代码规范检查}
  \begin{easylist}
    & PEP8:Style Guide for Python Code
    && https://www.python.org/dev/peps/pep-0008/
    & 工具:\href{https://www.pylint.org/}{pylint}
    && star your python code!
    && Coding Standard
    &&& checking line-code's length,
    &&& checking if variable names are well-formed according to your coding standard
    &&& checking if imported modules are used
    && Error detection
    &&& checking if declared interfaces are truly implemented
    &&& checking if modules are imported
    &&& and much more ...
  \end{easylist}
\end{frame}

\begin{frame}[fragile]{pylint}
  \begin{easylist}
    & 安装
    && sudo pip install pylint
    & 使用
    && pylint some\_file.py
    && 最终输出综合评分结果,例如:
  \end{easylist}

  \begin{tcolorbox}[colback=green!5,colframe=green!10!black,title=pylint输出结果片断]
    \textit{
      Global evaluation\\
      -----------------\\
      Your code has been rated at 8.38/10\\ }
  \end{tcolorbox}

\end{frame}

\subsubsection{单元测试}
\begin{frame}[fragile]{单元测试}
\begin{easylist}
  & 单元测试是用来对一个模块、一个函数或者一个类来进行正确性检验的测试工作。
  & TDD:Test-Driven Development
  && 敏捷开发中的一项常用技术和设计方法,即通过测试来推动整个开发的进行,在明确
  需要开发的功能后,首先思考如何对功能进行测试,进而完成测试用例代码的编写,然后
  实现具体产品功能,满足之前设计的测试用例。


\end{easylist}

  

  
\end{frame}


\begin{frame}[plain]{}
  \begin{center}
    ~ \\
    \Huge ---END---
  \end{center}
\end{frame}

\section{Web处理}

\begin{frame}[fragile]{Web服务器}
  \begin{easylist} \easyitem
    & 最简单的Python Web服务器
    & Flask
  \end{easylist}
\end{frame}

\begin{frame}[fragile]{Simple Http Server}
  \begin{easylist}

    & 最轻便的Web服务器\footnote{参数m的作用参考:
      \url{http://www.tuicool.com/articles/jMzqYzF}}:

    & 启动方式:
    && python -m http.server
  \end{easylist}
\end{frame}

\begin{frame}[fragile]{Flask}
  \begin{easylist}
    & Flask是一种简便的基于Python语言的Web应用程序开发框架
    && Flask is a microframework for Python based on Werkzeug, Jinja 2 and good intentions. And before you ask: It's BSD licensed!
    & 相关中文文档可在线参考:
    &&
    \scriptsize{\url{http://dormousehole.readthedocs.io/en/latest/quickstart.html}
    }
    & 安装:
    && pip install Flask
    & 文档
    && http://flask.pocoo.org/docs/0.10/.latex/Flask.pdf
  \end{easylist}
\end{frame}


\begin{frame}[fragile]{Flask简单例子}
  \lstinputlisting[keywordstyle=\ttfamily]{src/web/flask1.py}
\end{frame}

\begin{frame}[fragile, allowframebreaks]{Flask简单例子}
  \lstinputlisting[keywordstyle=\ttfamily]{src/web/flask2.py}
\end{frame}

\begin{frame}[fragile]{练习}
  实现一个Web程序,在网页上输入一个url地址,提交后,通过浏览器显示该地址所包含的所有图片。
\end{frame}


\begin{frame}[fragile]{科研人员在线社交媒体行为分析}
  \begin{easylist}
    & 编写一个Web应用程序,记录每个高校在社交媒体上的活动信息
    && 第一步完成基本信息的收集管理
    & 基本信息包括
    && 学校名称, 学院, 专业, 姓名, 性别, 出生年, 简介, 微博UID, 微博注册日期, 最
    后访问日期
  \end{easylist}

  \begin{easylist}
    & 基本信息管理功能:
    && 基本信息录入
    && 重复检测(根据学校、学院、姓名判断重复,或者根据微博UID判断重复)
    && 检索统计:
  \end{easylist}
\end{frame}

%%% Local Variables:
%%% mode: latex
%%% TeX-master: "../python"
%%% End:

\end{document}


%%% Local Variables:
%%% mode: latex
%%% TeX-master: t
%%% End:
